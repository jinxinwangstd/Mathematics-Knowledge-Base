\documentclass[onecolumn]{ctexart}
\usepackage[utf8]{inputenc}
\usepackage{amsmath}
\usepackage{amssymb}
\usepackage{amsthm}
\usepackage{geometry}
\usepackage{graphicx}
\usepackage{float}
\usepackage{xcolor}
\usepackage{listings}
\usepackage{indentfirst}
\usepackage{bm}
\usepackage{tikz}
\usetikzlibrary{shapes,arrows}
\geometry{a4paper,scale=0.8}

\newtheorem{definition}{Definition}
\newtheorem{theorem}{Theorem}
\newtheorem{proposition}{Proposition}
\newtheorem{lemma}{Lemma}
\newtheorem{corollary}{Corollary}
\newtheorem{remark}{Remark}
\newtheorem{example}{Example}

\title{Notes of "The Basic Theorems of Differential Calculus"}
\author{Jinxin Wang}
\date{}

\begin{document}

\maketitle

\section{Fermat's Lemma and Rolle's Theorem}

\begin{definition}[Local Maximum/Minimum and Local Maximum/Minimum Value]
  
\end{definition}

\begin{definition}[Strict Local Maximum/Minimum and Strict Local Maximum/Minimum Value]
  
\end{definition}

\begin{definition}[Local Extrema and Local Extreme Values]
  
\end{definition}

\begin{example}
  
\end{example}

\begin{example}
  
\end{example}

\begin{example}[Riemann's Function]
  
\end{example}

\begin{definition}[Interior Extremum]
  
\end{definition}

\begin{lemma}[Fermat's Lemma]
  
\end{lemma}

\begin{proposition}[Rolle's Theorem]
  
\end{proposition}

\section{The Theorem of Lagrange and Cauchy on Finite Increments}

\subsection{Lagrange's Finite-Increment Theorem}

\begin{theorem}[Lagrange's Finite-Increment Theorem]
  If a function $f: \lbrack a,b \rbrack \to \mathbb{R}$ is continous on a closed 
  interval $\lbrack a,b \rbrack$ and differentiable on the open interval 
  $(a, b)$, then there exists a point $\xi \in (a, b)$ such that
  \begin{equation}
    f(b) - f(a) = f'(\xi)(b - a)
  \end{equation} 
\end{theorem}
\begin{proof}
  Hint:
  \begin{itemize}
    \item Construct the auxiliary function $F(x) = f(x) - 
    \frac{f(b) - f(a)}{b - a}$.
    \item Use Rolle's Theorem.
  \end{itemize}
\end{proof}
\begin{remark}
  Constructing an auxiliary function is a useful method especially in proving 
  theorems of differential calculus. The auxiliary function in this proof can 
  be understood as leveling the function values of the two endpoints of the 
  interval.
\end{remark}
\begin{remark}
  Another geometrical proof of the Lagrange's Theorem is to rotate the 
  coordinate axes so that the new $x'$-axis is parallel to the chord joining the 
  points $(a, f(a))$ and $(b, f(b))$.
\end{remark}
\begin{remark}
  In geometric language the theorem means that at some point $(\xi, f(\xi)), 
  \xi \in (a, b)$, the tangent to the graph of the function is parallel to the 
  chord joining the points $(a, f(a))$ and $(b, f(b))$.
\end{remark}
\begin{remark}
  Lagrange's Theorem is important in that it connects the increment of a 
  function over a finite interval with the derivative of the function on that 
  interval. Until now it is the first tool to achieve that.
\end{remark}

\subsection{Corollaries (Applications) of Lagrange's Theorem}

\begin{corollary}[Criterion for Monotonicity of a Function]
  If the derivative of a function is nonnegative (resp. positive) at every point 
  of an open interval, then the function is nondecreasing (resp. increasing) on 
  that interval.
\end{corollary}
\begin{proof}
  Hint: $\forall x_1, x_2 \in (a, b)$, we have $f(x_1) - f(x_2) = 
  f'(\xi)(x_1 - x_2)$ according to the Lagrange's Theorem.
\end{proof}
\begin{remark}
  Notice that the intervals discussed in this corollary are open intervals. For 
  a closed interval, the prerequisite is that the function is continuous on the 
  closed interval and differentiable on the open interval (i.e. excluding the 
  two endpoints), then the relationship between the monotonicity and the sign of 
  the derivative is the same. It is unclear why this corollary chooses open 
  intervals as its discussed objects.
\end{remark}
\begin{remark}
  The sufficient condition of increasing function can be relaxed to that the 
  derivative of the function is positive on the open interval with only finite 
  points where the derivative is $0$.
\end{remark}
\begin{proof}
  Hint: Given an open interval $(a, b)$ where $f'(x) > 0$ is true on the 
  interval except only one point $c \in (a, b)$ where $f'(c) = 0$, we consider 
  $(a, c) \cup (c, b)$. Using Lagrange's Theorem it is clear that $f(x)$ is 
  increasing on both $\lbrack a, c \rbrack$ and $\lbrack c, b \rbrack$.
\end{proof}
\begin{remark}
  It is natural that an analogous assertion can be made about the nonincreasing 
  (resp. decreasing) nature of a function with a nonpositive (resp. negative) 
  derivative.
\end{remark}
\begin{remark}
  If a numerical-valued function $f(x)$ on some interval $I$ has a derivative 
  that is always positive or always negative, then the function is continuous 
  and monotonic on $I$, and has an inverse function $f^{-1}$ that is defined on 
  the interval $I' = f(I)$a and is differentiable on it.
\end{remark}

\begin{corollary}[Criterion for a Function to Be Constant]
  A function that is continous on a close interval $\lbrack a, b \rbrack$ is 
  constant on it if and only if the derivative equals to $0$ at every point of 
  the open interval $(a, b)$ (i.e. excluding the two endpoints).
\end{corollary}
\begin{proof}
  Hint: Apply Lagrange's Theorem.
\end{proof}
\begin{remark}
  If $F'_1(x) = F'_2(x)$ on an interval, then the difference $F_1(x) - F_2(x)$ 
  on that interval is constant.
\end{remark}

\subsection{Cauchy's Finite-Increment Theorem}

\begin{proposition}[Cauchy's Finite-Increment Theorem]
  Suppose $x = x(t)$ and $y = y(t)$ are continuous on a closed interval $\lbrack 
  \alpha, \beta \rbrack$ and differentiable on the open interval $(\alpha, 
  \beta)$, then there exists a point $\tau \in (\alpha, \beta)$ such that
  \begin{equation}
    x'(\tau)(y(\beta) - y(\alpha)) = y'(\tau)(x(\beta) - x(\alpha))
  \end{equation}

  If $x'(t) \neq 0$ is true on the open interval $(\alpha, \beta)$, then $x(a) 
  \neq x(b)$ and the following equation holds
  \begin{equation}
    \frac{y(\beta) - y(\alpha)}{x(\beta) - x(\alpha)} = \frac{y'(\tau)}{x'(\tau)}
  \end{equation}
\end{proposition}
\begin{proof}
  Hint:
  \begin{itemize}
    \item Construct the auxiliary function $F(t) = (x(\beta) - x(t))(y(\beta) - 
    y(\alpha)) - (y(\alpha) - y(t))(x(\beta) - x(\alpha))$, then $F(\alpha) = 
    F(\beta) = (x(\beta) - x(\alpha))(y(\beta) - y(\alpha))$. (An alternative 
    auxiliary function from Zorich: $F(x) = x(t)(y(\beta) - y(\alpha)) - 
    y(t)(x(\beta) - x(\alpha))$, then $F(\alpha) = F(\beta) = x(\alpha)y(\beta) - 
    x(\beta)y(\alpha)$).
    \item Use Rolle's Theorem.
  \end{itemize}
\end{proof}
\begin{remark}
  Interpretation in physics: the movement of a particle in a plane. Exception: 圆螺线。
\end{remark}
\begin{remark}
  From Cauchy's Theorem to Lagrange's Theorem: Set $x(t) = t$.
\end{remark}

\section{Taylor's Formula}

\subsection{Taylor Polynomial}

An intuition: if two functions have same derivatives of more orders (including 
the order $0$, i.e. the function itself, which is important) at a point $x = x_0$, 
then the values of the two functions in a neighborhood of $x_0$ are closer.

\begin{definition}[Taylor Polynomial of Order $n$]
  The following algebraic polynomial is the Taylor polynomial of order $n$ of 
  $f(x)$ at $x_0$:
  \begin{equation}
    P_n(x_0; x) = P_n(x) = f(x_0) + \frac{f'(x_0)}{1!}(x - x_0) + \cdots + \frac{f^{(n)}(x_0)}{n!}(x - x_0)^n
  \end{equation}
\end{definition}

\subsection{Remainder of Taylor Polynomial}

\begin{equation}
  f(x) - P_n(x_0; x) = r_n(x_0; x)
\end{equation}

\begin{equation}
  f(x) = f(x_0) + \frac{f'(x_0)}{1!}(x - x_0) + \cdots + \frac{f^{(n)}(x_0)}{n!}(x - x_0)^n + r_n(x_0; x)
\end{equation}

\begin{theorem}
  If the function $f$ is continuous on the closed interval with end-points $x_0$ 
  and $x$ ($\lbrack x, x_0 \rbrack$ or $\lbrack x_0, x \rbrack$), and its first 
  $n$ derivatives are continuous on this closed interval, and it has a derivative 
  of order $n+1$ at the interior points of this interval, then for any function 
  $\phi$ that is continuous on this close interval and has a nonzero derivative 
  at its interior points, there exists a point $\xi$ between $x_0$ and $x$ such 
  that
  \begin{equation}
    r_n(x_0; x) = \frac{\phi(x) - \phi(x_0)}{\phi'(\xi)n!}f^{(n+1)}(\xi)(x - \xi)^n
  \end{equation}
\end{theorem}
\begin{proof}
  Hint:
  \begin{itemize}
    \item Construct the auxiliary function 
    \[
      \begin{split}
        F(t) &= f(x) - P_n(x_0; x) \\
             &= f(x) - (f(t) + \frac{f'(t)}{1!}(x - t) + \frac{f''(t)}{2!}(x - t)^2 + \cdots + \frac{f^{(n)}(t)}{n!}(x - t)^n)
      \end{split}
    \]
    Then $F(x) = 0$ and $F(x_0) = r_n(x_0;x)$, which shows that our motivation 
    behind this auxiliary function is to make the finite increment of $F(t)$ 
    between $x_0$ and $x$ as $r_n(x_0;x)$.

    Notice that in $F(t)$ the variable is $t$ while $x$ is a constant specified 
    by the closed interval in the above theorem in the above theorem.
    \item Use Cauchy's finite increment theorem.
  \end{itemize}
\end{proof}
\begin{remark}
  Here the Cauchy's finite increment theorem comes on the scene because we need 
  a tool to characterize a finite difference between two functions at a point. 
  If the finite difference between two functions can be transformed into a 
  finite increment of a function, then we can use the Cauchy's finite increment 
  theorem.
\end{remark}

\begin{corollary}[Cauchy's Formula for the Remainder Term]
  Let $\phi(t) = x - t$, then the remainder becomes
  \begin{equation}
    r_n(x_0; x) = \frac{1}{n!}f^{(n+1)}(\xi)(x - \xi)^n(x - x_0)
  \end{equation}
\end{corollary}

\begin{corollary}[The Lagrange Form of the Remainder]
  Let $\phi(t) = (x - t)^{n+1}$, then the remainder becomes
  \begin{equation}
    r_n(x_0; x) = \frac{1}{(n+1)!}f^{(n+1)}(\xi)(x - x_0)^{n+1}
  \end{equation}
\end{corollary}

\subsection{Local Taylor Formula}

\begin{proposition}
  If there exists a polynomial
  \[
    P_n(x_0;x) = c_0 + c_1(x - x_0) + \cdots + c_n(x - x_0)^n 
  \]
  satisfying the following condition
  \[
    f(x) = P_n(x) + o((x - x_0)^n), \thickspace \textnormal{as} \thickspace x \to x_0, x \in E
  \],
  that polynomial is unique.
\end{proposition}
\begin{proof}
  
\end{proof}

\begin{proposition}[The Local Taylor Formula]
  Let $E$ be a closed interval having $x_0 \in \mathbb{R}$ as an endpoint. If 
  the function $f: E \to \mathbb{R}$ has derivatives $f'(x_0), \ldots, 
  f^{(n)}(x_0)$ up to order $n$ inclusive at the point $x_0$, then the following 
  representation holds:
  \begin{equation}
    f(x) = f(x_0) + \frac{f'(x_0)}{1!}(x - x_0) + \cdots + \frac{f^{(n)}(x_0)}{n!}(x - x_0)^n + o((x - x_0)^n), \thickspace \textnormal{as} \thickspace x \to x_0, x \in E
  \end{equation}
\end{proposition}
\begin{remark}
  The problem of the local approximation of a differentiable function is solved 
  by the Taylor polynomial of the appropriate order.
\end{remark}

\begin{lemma}
  If a function $\varphi: E \to \mathbb{R}$, defined on a closed interval $E$ 
  with endpoint $x_0$, is such that it has derivatives up to order $n$ inclusive 
  at $x_0$ and $\varphi(x_0) = \varphi'(x_0) = \cdots = \varphi^{(n)}(x_0) = 0$, 
  then $\varphi(x) = o((x - x_0)^n)$, as $x \to x_0, x \in E$.
\end{lemma}
\begin{proof}
  Hint: Proof by Induction.
\end{proof}
\begin{remark}
  
\end{remark}
\end{document}