\documentclass[onecolumn]{ctexart}
\usepackage[utf8]{inputenc}
\usepackage{amsmath}
\usepackage{amssymb}
\usepackage{amsthm}
\usepackage{geometry}
\usepackage{graphicx}
\usepackage{float}
\usepackage{xcolor}
\usepackage{listings}
\usepackage{indentfirst}
\usepackage{bm}
\usepackage{tikz}
\usetikzlibrary{shapes,arrows}
\geometry{a4paper,scale=0.8}

\newtheorem{definition}{Definition}
\newtheorem{theorem}{Theorem}
\newtheorem{proposition}{Proposition}
\newtheorem{lemma}{Lemma}
\newtheorem{corollary}{Corollary}
\newtheorem{remark}{Remark}
\newtheorem{example}{Example}

\title{Notes of "Differential Calculus Used to Study Functions"}
\author{Jinxin Wang}
\date{}

\begin{document}

\maketitle

\section{L'Hopital's Rule}

\begin{proposition}
  Suppose the function $f: (a,b) \to \mathbb{R}$ and $g: (a,b) \to \mathbb{R}$
  are differentiable on the open interval $(a, b)(-\infty \leq a < b \leq 
  +\infty)$ with $g'(x) \neq 0$ on $(a, b)$ and
  \[
    \frac{f'(x)}{g'(x)} = A \thickspace \textnormal{as} \thickspace x \to a+0 \thickspace (-\infty \leq A \leq +\infty)
  \]
  Then
  \[
    \frac{f(x)}{g(x)} = A \thickspace \textnormal{as} \thickspace x \to a+0   
  \]
  in each of the following two cases:
  \begin{description}
    \item[$1^0$] $(f(x) \to 0) \wedge (g(x) \to 0)$ as $x \to a+0$
    \item[$2^0$] $g(x) \to \infty$ as $x \to a+0$
  \end{description}

  A similar assertion holds as $x \to b-0$
\end{proposition}
\begin{proof}
  
\end{proof}

\end{document}