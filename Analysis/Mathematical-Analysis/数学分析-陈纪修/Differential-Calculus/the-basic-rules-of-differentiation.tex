\documentclass[onecolumn]{ctexart}
\usepackage[utf8]{inputenc}
\usepackage{amsmath}
\usepackage{amssymb}
\usepackage{amsthm}
\usepackage{geometry}
\usepackage{graphicx}
\usepackage{float}
\usepackage{xcolor}
\usepackage{listings}
\usepackage{indentfirst}
\usepackage{bm}
\usepackage{tikz}
\usetikzlibrary{shapes,arrows}
\geometry{a4paper,scale=0.8}

\newtheorem{definition}{Definition}
\newtheorem{theorem}{Theorem}
\newtheorem{proposition}{Proposition}
\newtheorem{lemma}{Lemma}
\newtheorem{corollary}{Corollary}
\newtheorem{remark}{Remark}
\newtheorem{example}{Example}

\title{Notes of "Basic Rules of Differentiation"}
\author{Jinxin Wang}
\date{}

\begin{document}

\maketitle

\section{Differentiation and the Arithmetic Operations}

\begin{theorem}[Rules of Differentiation of Results of Arithmetic Operations]
  
\end{theorem}

\begin{corollary}
  The derivative of a linear combination of differentiable functions equals the 
  same linear combination of the derivatives of these functions.
\end{corollary}

\begin{corollary}
  If the functions $f_1, f_2, \ldots, f_n$ is differentiable at $x$, then
  \[
    (f_1f_2 \cdots f_n)'(x) = f_1'(x)f_2(x) \cdots f_n(x) + f_1(x)f_2'(x) \cdots f_n(x) + \cdots + f_1(x)f_2(x) \cdots f_n'(x)
  \]
\end{corollary}

\begin{corollary}
  The rules of differentiation of a result of arithmetic operations can also be 
  written in terms of differentials:
  \begin{itemize}
    \item $d(f + g)(x)$
    \item $d(f \cdot g)(x)$
    \item $d(\frac{f}{g})(x)$
  \end{itemize}
\end{corollary}

\section{Differentiation of a Composite Function (Chain Rule)}

\begin{theorem}[Differentiation of a Composite Function]
  If the function $f: X \to Y \subset \mathbb{R}$ is differentiable at a point 
  $x \in X$ and the function $g: Y \to \mathbb{R}$ is differentiable at the 
  point $y = f(x) \in Y$, then the composite function $g \circ f: X \to 
  \mathbb{R}$ is differentiable at $x$, and the differential $d(g \circ f)(x): 
  T\mathbb{R}(x) \to T\mathbb{R}(g(f(x)))$ of their composition equals the 
  composition $dg(y) \circ df(x)$ of their differentials
  \[
    df(x): T\mathbb{R}(x) \to T\mathbb{R}(y = f(x)) \thickspace \textnormal{and} \thickspace dg(y = f(x)): T\mathbb{R}(y) \to T\mathbb{R}(g(y))
  \]
\end{theorem}
\begin{proof}
  Hint:
\end{proof}
\begin{remark}
  Let $z = g(y), y = f(x)$, one may consider using the following relation to 
  prove the rule of the differentiation of a composite function:
  \[
    \frac{\Delta z}{\Delta x} = \frac{\Delta z}{\Delta y} \frac{\Delta y}{\Delta x}
  \]
  \[
    \frac{dz}{dx} = \lim_{\Delta x \to 0} \frac{\Delta z}{\Delta x} = \lim_{\Delta x \to 0} \frac{\Delta z}{\Delta y} \frac{\Delta y}{\Delta x}
  \]
  The difficulty that appears in this method is that as $\Delta x \to 0$, 
  $\Delta y$ may be 0, which makes the above equation invalid.

  In the above proof, we use the following techniques to handle this difficulty:
  \begin{enumerate}
    \item First, we use differentials rather than derivatives to avoid division.
    \item Second, we make the differential of $g(y)$ valid at $\Delta y = 0$ by 
    defining $o(t) = 0$ when $t = 0$.
  \end{enumerate}
\end{remark}

\begin{corollary}
  The derivative $(g \circ f)'(x)$ of the composition of differentiable 
  real-valued functions equals the product $g'(f(x))\cdot f'(x)$ of the 
  derivatives of these functions computed at the corresponding points.
\end{corollary}

\begin{corollary}
  If the composition $(f_n \circ \cdots \circ f_2 \circ f_1)(x)$ of 
  differentiable functions $y_1 = f_1(x), y_2 = f_2(y_1), \ldots, y_n = 
  f_n(y_{n-1})$ exists, then
  \[
    (f_n \circ \cdots \circ f_2 \circ f_1)'(x) = f_n'(y_{n-1}) f_{n-1}'(y_{n-2}) \cdots f_2'(y_1) f_1'(x)
  \]
\end{corollary}

\section{Differentiation of an Inverse Function}

\begin{theorem}[The Derivative of an Inverse Function]
  Let the functions $f: X \to Y$ and $f^{-1}: Y \to X$ be mutually inverse and 
  continuous at points $x_0 \in X$ and $f(x_0) = y_0 \in Y$ respectively. If $f$ 
  is differentiable at $x_0$ and $f'(x_0) \neq 0$, then $f^{-1}$ is also 
  differentiable at the points $y_0$, and 
  \[
    (f^{-1})'(y_0) = f'(x_0)^{-1}
  \]
\end{theorem}
\begin{proof}
  Hint:
  \begin{itemize}
    \item $f$ and $f^{-1}$ are mutually inverse $\Rightarrow$ (($x = f^{-1}(y) 
    \neq x_0 = f^{-1}(y_0)) \Leftrightarrow (y = f(x) \neq y_0 = f(x_0)$))
    \item $f$ and $f^{-1}$ are continuous at $x_0$ and $y_0 = f(x_0)$ 
    respectively $\Rightarrow$ $(X \owns x \to x_0) \Leftrightarrow (Y \owns y 
    \to y_0)$
  \end{itemize}
\end{proof}

\section{Table of Derivatives of the Basic Elementary Functions}

\section{Differentiation of a Very Simple Implicit Function}

\begin{example}[The Law of Addition of Velocities]
  
\end{example}

\section{Higher-Order Derivatives}

\begin{definition}[Higher-Order Derivatives]
  By induction, if the derivative $f^{(n-1)}(x)$ of order $n - 1$ of $f$ has 
  been defined, then the derivative of order $n$ is defined by the formula
  \[
    f^{(n)}(x) := (f^{(n-1)})'(x)
  \]

  In convention, the derivative of order $n$ is denoted by $f^{(n)}(x)$ or $\frac{d^n f(x)}{dx^n}$
\end{definition}

\begin{example}[Leibniz's Formula]
  Let $u(x)$ and $v(x)$ be functions having derivatives up to order $n$ 
  inclusive on a common set $E$.
  \begin{equation}
    (uv)^{(n)} = \Sigma_{m=0}^n 
    \begin{pmatrix}
      n \\
      m \\
    \end{pmatrix} u^{(n-m)} v^{(m)}
  \end{equation}
  \begin{proof}
    Hint:
    \begin{itemize}
      \item By principle of induction.
      \item Use the formula of binomial coefficient: $
      \begin{pmatrix}
        n \\
        k \\
      \end{pmatrix} = 
      \begin{pmatrix}
        n-1 \\
        k-1 \\
      \end{pmatrix} + 
      \begin{pmatrix}
        n-1 \\
        k \\
      \end{pmatrix}$
    \end{itemize}
  \end{proof}
\end{example}

\begin{example}[Finding the Coefficients of the Talor's Formula]
  If $P_n(x) = c_0 + c_1 x + c_2 x^2 + \cdots + c_n x^n$, then
  \[
    c_0 = ,
  \]
  \[
    c_1 = ,
  \]
  \[
    c_2 = ,
  \]
  \[
    c_n
  \]
  Thus,
  \[
    P_n(x) = 
  \]
\end{example}

\begin{example}[Acceleration]
  
\end{example}

\begin{example}[The Second Derivative of a Simple Implicit Function]
  Let $y = y(t)$ and $x = x(t)$
  \[
    \begin{split}
      y_{x^n}^{(n)} &= \frac{d}{dx}(y_{x^{n-1}}^{(n-1)}) \\
                    &= \frac{d}{dt}(y_{x^{n-1}}^{(n-1)}) \frac{dt}{dx} \\
                    &= \frac{(y_{x^{n-1}}^{(n-1)})'_t}{x'_t}
    \end{split}
  \]
\end{example}
\end{document}