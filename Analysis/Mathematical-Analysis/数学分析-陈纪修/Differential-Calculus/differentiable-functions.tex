\documentclass[onecolumn]{ctexart}
\usepackage[utf8]{inputenc}
\usepackage{amsmath}
\usepackage{amssymb}
\usepackage{amsthm}
\usepackage{geometry}
\usepackage{graphicx}
\usepackage{float}
\usepackage{xcolor}
\usepackage{listings}
\usepackage{indentfirst}
\usepackage{bm}
\usepackage{tikz}
\usetikzlibrary{shapes,arrows}
\geometry{a4paper,scale=0.8}

\newtheorem{definition}{Definition}
\newtheorem{theorem}{Theorem}
\newtheorem{proposition}{Proposition}
\newtheorem{lemma}{Lemma}
\newtheorem{corollary}{Corollary}
\newtheorem{remark}{Remark}
\newtheorem{example}{Example}

\title{Notes of "Differentiable Functions"}
\author{Jinxin Wang}
\date{}

\begin{document}

\maketitle

\section{Statement of the Problem and Introductory Considerations}

Solving the Kepler problem of two bodies, which is to describe the motion of a 
planet relative to a star. With an appropriate coordinate system, we use two 
functions $x(t)$ and $y(t)$ to describe the motion of the planet.

\begin{equation} \label{eq:1}
  \vec{F} = m \vec{a}
\end{equation}

\begin{equation}
  \vec{F} = G\frac{Mm}{|\vec{r}|^3}\vec{r}
\end{equation}

Now our task is to define and compute the instaneous velocity of a motion 
governed by the law (\ref{eq:1}).

By the law of inertia, if no force is acted on a body, it moves in uniform 
motion. Hence if at time $t$ the external action on a body ceases, it will 
continue moving in a straight line with a certain velocity. Since forces are to 
change velocity instead of maintaining velocity, it is natural to consider the 
hypothetical velocity as the instaneous velocity at time $t$.

If the absolute magnitude of the rate of change of a quantity $f(t)$ is bounded 
over an time interval $[0, t]$, then as the time interval $\Delta t$ becomes 
smaller, the change of the quantity $\Delta f$ becomes smaller, which is 
equivalent to say that $f(t)$ is continuous.

Apparently the above analysis applies to the case of velocity where the rate of 
change of velocity is acceleration, whose absolute magnitude is bounded since 
the external force on a body is finite. Therefore, at all times $t$ close to 
some time $t_0$ the velocity $\vec{v}(t)$ of the body $m$ must be close to the 
value $\vec{v}(t_0)$ that we wish to determine.
\begin{remark}
  Notice that there seems to be a shift from a scalar $f(t)$ to a vector 
  $\vec{v}(t)$, and a question may arise that whether the previous analysis 
  about the bounded rate of change and continuity of a scalar function applies 
  to a vector function. Consider the nature of change, it must be true that when 
  a change of a quantity becomes smaller, the changed quantity is closer to the 
  original quantity, no matter whether the quantity is a scalar or a vector. 
  Therefore, the analysis about the bounded rate of change and continuity 
  applies to both cases.
\end{remark}

\section{Functions Differentiable at a Point}

\begin{definition}
  A function $f:E \to \mathbb{R}$ is differentiable at a point $a \in E$ that is 
  a limit point of $E$ if there exists a linear function $A(x - a)$ such that 
  $f(x) - f(a)$ can be represented as
  \begin{equation}
    f(x) - f(a) = A (x - a) + o(x - a) \thickspace \textnormal{as} \thickspace x \to a, x \in E
  \end{equation}
\end{definition}
\begin{corollary}
  If a function $f:E \to \mathbb{R}$ is differentiable at a point $a \in E$ that 
  is a limit point of $E$, then it is also continuous at the point $a \in E$.
\end{corollary}
\begin{remark}
  The conversion of the above corollary is not true. An counterexample is that 
  $f(x) = x^{\frac{1}{3}}$ is continuous at $x = 0$, but it is not 
  differentiable at $x = 0$.
\end{remark}

\begin{definition}
  The linear function $A (x - a)$ in the above definition is called the 
  differential of the function $f$ at $a$.
\end{definition}

\begin{definition}
  The number
  \begin{equation}
    f'(a) = \lim_{E \owns x \to a} \frac{f(x) - f(a)}{x - a}
  \end{equation}
  is called the derivative of the function $f$ at $a$.
\end{definition}

\begin{definition}
  A function $f:E \to \mathbb{R}$ is differentiable at a point $a \in E$ that is 
  a limit point of $E$ if
  \begin{equation}
    f(x + h) - f(h) = A(x)h + \alpha(x;h)
  \end{equation}
  where $h \mapsto A(x)h$ is a linear function in $h$ and $\alpha(x;h) = o(h)$ 
  as $h \to 0, x + h \in E$.
\end{definition}
\begin{remark}
  The difference between this definition of differentiability at a point and the 
  previous one is that this one takes into account that the quantity $A(x)$ and 
  $\alpha(x;h)$ may change in terms of differentiability at different points $x$, 
  so we write them as functions of $x$.
\end{remark}

\begin{definition}
  The function $h \mapsto A(x)h$ in the definition of differentiability at a 
  point, which is linear in $h$, is called the differential of the function $f:E 
  \to \mathbb{R}$ at the point $x \in E$ and is denoted $df(x)$ or $Df(x)$.
\end{definition}
\begin{remark}[The Principal Linear Part of the Increment of the Function]
  
\end{remark}
\begin{remark}[The Leibniz Notation of the Derivative]
  \begin{equation}
    df(x)(h) = f'(x)h
  \end{equation}
  \[
    dx(h) = 1 \cdot h = h
  \]
  \begin{equation}
    df(x)(h) = f'(x)dx(h)
  \end{equation}
  \begin{equation}
    df(x) = f'(x)dx
  \end{equation}
  \begin{equation}
    f'(x) = \frac{df(x)(h)}{dx(h)}
  \end{equation}
\end{remark}

\section{The Tangent Line; Geometric Meaning of the Derivative and Differential}

\begin{equation}
  f(x) = c_0 + c_1(x - x_0) + o(x - x_0) \thickspace \textnormal{as} \thickspace x \to x_0, x \in E
\end{equation}

\begin{proposition}
  A function $f:E \to \mathbb{R}$ that is continuous at a point $x_0 \in E$ that 
  is a limit point of $E \subset \mathbb{R}$ admits a linear approximation as 
  \begin{equation}
    f(x) = c_0 + c_1(x - x_0) + o(x - x_0) \thickspace \textnormal{as} \thickspace x \to x_0, x \in E
  \end{equation} 
  if and only if it is differentiable at the point.
\end{proposition}
\begin{remark}
  (TODO) What is the difference between this proposition and the definition of the 
  differentiability at a point?
\end{remark}

\begin{definition}[Analytic Definition of a Tangent Line]
  If a function $f:E \to \mathbb{R}$ is defined on a set $E \subset \mathbb{R}$ 
  and differentiable at a point $x_0 \in E$, the line defined by the equation
  \begin{equation}
    y - f(x_0) = f'(x_0)(x - x_0)
  \end{equation}
  is called the tangent to the graph of this function at the point $(x_0, f(x_0))$.
\end{definition}

\begin{definition}[Nth Order Contact at a Point between Two Mappings]
  If the mappings $f:E \to \mathbb{R}$ and $g:E \to \mathbb{R}$ are continuous 
  at a point $x_0 \in E$ that is a limit point of $E$ and $f(x) - g(x) = 
  o((x - x_0)^n)$ as $x \to x_0, x \in E$, we say that $f$ and $g$ have $n$th 
  order contact at $x_0$ (more precisely, contact of order at least $n$).

  For $n = 1$, we say that the mappings $f$ and $g$ are tangent to each other 
  at $x_0$.
\end{definition}
\begin{example}[Polynomial Approximation]
  
\end{example}

\section{The Role of the Coordinate System}

\begin{remark}[Geometric Definition of a Tangent Line]
  
\end{remark}
\end{document}
