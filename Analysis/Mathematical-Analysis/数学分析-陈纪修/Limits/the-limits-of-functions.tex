\documentclass[onecolumn]{ctexart}
\usepackage[utf8]{inputenc}
\usepackage{amsmath}
\usepackage{amssymb}
\usepackage{amsthm}
\usepackage{geometry}
\usepackage{graphicx}
\usepackage{float}
\usepackage{xcolor}
\usepackage{listings}
\usepackage{indentfirst}
\usepackage{bm}
\usepackage{tikz}
\usetikzlibrary{shapes,arrows}
\geometry{a4paper,scale=0.8}

\newtheorem{definition}{Definition}
\newtheorem{theorem}{Theorem}
\newtheorem{proposition}{Proposition}
\newtheorem{lemma}{Lemma}
\newtheorem{corollary}{Corollary}
\newtheorem{remark}{Remark}
\newtheorem{example}{Example}

\DeclareMathOperator{\rank}{rank}
\DeclareMathOperator{\sgn}{sgn}

\title{Notes of "The Limit of A Function"}
\author{Jinxin Wang}
\date{}

\begin{document}

\maketitle

\section{Definitions and Examples}

\begin{definition}[The Limit of a Function (Basic Type)]
  ($\epsilon$-$\delta$)
\end{definition}

\begin{remark}
  Based on the definition, we can see that limits are a kind of local 
  characteristic of a function. With that, when using the definition to prove 
  the limit of a function at $x = x_0$, we can discuss it within a certain 
  neighborhood $O(x_0, \delta_0)$.
\end{remark}

\begin{definition}[去心邻域]
  
\end{definition}

\begin{definition}[函数极限的邻域定义]
  
\end{definition}

\begin{example}[符号函数$\sgn x$]
  
\end{example}

\begin{proposition}[Heine's Proposition]
  
\end{proposition}

\begin{corollary}[Existence of the Limit of a Function by Limits of Sequences]
  
\end{corollary}

\section{Properties of the Limit of a Function}

\subsection{General Properties}

\begin{definition}[有界函数/局部有界函数,上有界函数/局部上有界函数,下有界函数/局部下有界函数]
  
\end{definition}

\begin{theorem}
  \begin{enumerate}
    \item (局部常值函数收敛)
    \item (局部有界)
    \item (在某一点处极限值唯一)
  \end{enumerate}
\end{theorem}

\subsection{Properties Involving Arithmetic}

\begin{definition}[两个函数的和、积与商]
  
\end{definition}

\begin{theorem}[四则运算中的函数极限]
  
\end{theorem}

\begin{definition}[无穷小函数]
  
\end{definition}

\begin{proposition}[无穷小函数的四则运算性质]
  
\end{proposition}

\subsection{Properties Involving Inequalities}

\begin{theorem}
  \begin{itemize}
    \item (局部保序性)
    \item (夹逼性)
  \end{itemize}
\end{theorem}

\begin{corollary}[局部保号性]
  
\end{corollary}

\begin{corollary}[极限值的不等关系]
  
\end{corollary}

\begin{example}[研究$\frac{\sin x}{x}$在$x=0$处的极限]
  \begin{equation}
    \lim_{x \to 0} \frac{\sin x}{x} = 1
  \end{equation}
\end{example}

\begin{example}[定义指数函数、对数函数和幂函数]
  
\end{example}

\section{The General Definition of the Limit of a Function}

\subsection{Definition and Examples of a Base}

\subsection{Limit over a Base}

\section{The Existence of the Limit of a Function}

\subsection{Cauchy's Criterion}

\begin{definition}[Oscillation]
  
\end{definition}

\begin{theorem}[Cauchy's Criterion on the Limit of a Function]
  
\end{theorem}

\subsection{The Limit of a Composite Function}

\begin{theorem}[Theorem of the Limit of a Composite Function]
  
\end{theorem}

\begin{example}
  \begin{equation}
    \lim_{x \to \infty} (1 + \frac{1}{x})^x = e
  \end{equation}
\end{example}

\subsection{The Limit of a Monotonic Function}

\begin{definition}[Monotonic Functions]
  
\end{definition}

\begin{theorem}[The Existence Criterion of the Limit of a Monotonic Function]
  
\end{theorem}

\subsection{Comparison of the Limiting Behaviors of Functions}

\begin{proposition}
  
\end{proposition}

\begin{remark}
  The above proposition can not be generalized to the limit of a sum of 
  functions.
\end{remark}

\section{方法与技巧}

\subsection{证明与研究函数极限}

\begin{itemize}
  \item 定义法
  \item 变量代换
\end{itemize}

\begin{remark}
  在研究函数极限时使用变量代换是否总是成立?这个问题使用数学语言来描述如下:\\
  Suppose that $\lim_{x \to a} g(x) = A$, $\lim_{x \to A} f(x) = B$, is it true that 
  \[
    \lim_{x \to a} f(g(x)) = \lim_{y \to A} f(y)
  \]
  In fact it is not always true. Here is a counterexample: Let $g(x) \equiv 0$, 
  and thus $\lim_{x \to 0} g(x) = 0$. Let $f(x) = 
  \begin{cases}
    1, x = 0 \\
    0, x \neq 0
  \end{cases}$. Then we have $\lim_{x \to 0} f(x) = 0$, $\lim_{x \to 0} f(g(x)) 
  = 1$. 这里的数学直观是$\lim_{x \to a} g(x) = A$决定了$y = g(x) \to A$的方式。它与
  $\lim_{x \to A} f(x) = B$中$x \to A$的不同可能导致结果的不同。

  Here are two propositions related to this problem:
  \begin{proposition}
    Suppose that $\lim_{x \to a} g(x) = A$, $\lim_{x \to A} f(x) = B$. If any of 
    the following conditions is true:
    \begin{itemize}
      \item $\exists \delta_0 > 0$ such that $\forall x \in O(a, \delta_0) 
      \setminus \{a\}$: $g(x) \neq A$.
      \item $\lim_{x \to A} f(x) = f(A)$.
      \item $A = \infty$, and $\lim_{x \to \infty} f(x)$ is defined.
    \end{itemize}
    then the following is true:
    \[
      \lim_{x \to a} f(g(x)) = \lim_{y \to A} f(y)
    \]
  \end{proposition}
  \begin{proof}
    Hint:
    \begin{itemize}
      \item Notice the difference between the conclusion of the definition of 
      $\lim_{x \to a} g(x) = A$, and the condition of the definition of 
      $\lim_{x \to A} f(x) = B$.
      \item Notice what change the fact of continuity brings to the definition 
      of $\lim_{x \to A} f(x) = B$.
    \end{itemize}
  \end{proof}

  \begin{proposition}
    If $\lim_{x \to a} g(x) = A$, $\lim_{x \to A} f(x) = B$, then exact one of 
    the following situation is true:
    \begin{itemize}
      \item $\lim_{x \to a} f(g(x)) = B$
      \item $\lim_{x \to a} f(g(x)) = g(A)$
      \item $\lim_{x \to a} f(g(x))$ is not defined
    \end{itemize}
  \end{proposition}
  \begin{proof}
    Hint: Using the first condition of the previous proposition to discuss 
    different kinds of $g(x)$.
  \end{proof}
\end{remark}

\end{document}