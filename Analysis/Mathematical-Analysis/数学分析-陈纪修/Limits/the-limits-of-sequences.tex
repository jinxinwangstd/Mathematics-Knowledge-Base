\documentclass[onecolumn]{ctexart}
\usepackage[utf8]{inputenc}
\usepackage{amsmath}
\usepackage{amssymb}
\usepackage{amsthm}
\usepackage{geometry}
\usepackage{graphicx}
\usepackage{float}
\usepackage{xcolor}
\usepackage{listings}
\usepackage{indentfirst}
\usepackage{bm}
\usepackage{tikz}
\usetikzlibrary{shapes,arrows}
\geometry{a4paper,scale=0.8}

\newtheorem{definition}{Definition}
\newtheorem{theorem}{Theorem}
\newtheorem{proposition}{Proposition}
\newtheorem{lemma}{Lemma}
\newtheorem{corollary}{Corollary}
\newtheorem{remark}{Remark}
\newtheorem{example}{Example}

\DeclareMathOperator{\rank}{rank}

\title{Notes of "The Limit of A Sequence"}
\author{Jinxin Wang}
\date{}

\begin{document}

\maketitle

\section{Definition of the Limit of a Sequence}

\begin{definition}[The Limit of a Sequence]
  ($\epsilon$-N) 

  (neighborhood-N)
\end{definition}

\begin{remark}
  The two definitions of the limit of a sequence is equivalent. The equivalence 
  relation between them relies on the definition of a neighborhood.
\end{remark}

\begin{definition}[Divergence]
  
\end{definition}

\section{Properties of the Limit of a Sequence}

\subsection{General Properties}

\begin{itemize}
  \item 有限点无关性:A finite number of terms of a sequence doesn't affect the 
  convergence of the sequence. (Proof: definition)
  \item 唯一性:The limit of a convergent sequence is unique. (Proof: 
  contradiction + definition)
  \item 有界性:A convergent sequence is bounded. (Proof: definition)
\end{itemize}

\subsection{Properties Involving Arithmetic Operations}

\begin{theorem}[极限的四则运算]
  
\end{theorem}

\subsection{Properties Involving Inequalities}

\begin{theorem}[保序性]
  
\end{theorem}

\begin{theorem}[夹逼性]
  
\end{theorem}

\section{Infinity}

\subsection{Definition of Infinity}

\begin{definition}[Infinity, Positive Infinity, Negative Infinity]
  
\end{definition}

\begin{corollary}[Relation between Infinity and Infinitesimal]
  
\end{corollary}

\begin{definition}[Not an Infinity]
  
\end{definition}

\subsection{Operations Involving Infinity}

\section{Existence of the Limit of a Sequence}

\subsection{Cauchy's Convergence Criterion}

\subsection{Existence of the Limit of a Monotonic Sequence}

\subsubsection{e}

\begin{proposition}
  The sequences $a_n = (1 + \frac{1}{n})^n$ and $b_n = (1 + \frac{1}{n})^{n+1}$ 
  are convergent, and they have the same limit values.
\end{proposition}
\begin{proof}
  
\end{proof}

\begin{definition}
  
\end{definition}

\begin{proposition}
  $e = \Sigma_{n=0}^{\infty} \frac{1}{n!} = 1 + 1 + \frac{1}{2!} + \frac{1}{3!} + \cdots + \frac{1}{n!} + \cdots$
\end{proposition}
\begin{proof}
  
\end{proof}

\begin{proposition}
  $e = \lim_{n \to \infty} \frac{n}{\sqrt[n]{n!}}$
\end{proposition}
\begin{proof}
  
\end{proof}

\subsubsection{pi}

\subsubsection{Euler Number}

\begin{theorem}[Weierstrass Theorem / 单调有界数列收敛定理]
  
\end{theorem}

\begin{corollary}
  当数列严格单调增加时,
\end{corollary}

\subsection{Subsequences and the Partial Limits}

\subsection{闭区间套定理}

\section{The Limit of Transformed Sequences}

\subsection{Undertermined Form}

\subsection{Toeplitz's Theorem}

\begin{theorem}
  Suppose there exists a sequence $\{t_{nk}\}$ such that 
  $\forall n, k \in \mathbb{N^+}$, $t_{nk} \geq 0$, $\Sigma_{k=1}^n t_{nk} = 1$, 
  $\lim_{n \to \infty} t_{nk} = 0$. If $\lim_{n \to \infty} a_n = a$, then 
  \[
    \lim_{n \to \infty} \Sigma_{k=1}^n t_{nk} a_k = a
  \]
\end{theorem}

\begin{remark}
  The condition in the Toeplitz' Theorem $\lim_{n \to \infty} t_{nk} = 0$ means 
  that for any given $k$, in other words $k$ is finite, $t_{nk}$ tends to $0$ 
  when $n$ tends to $\infty$. This is supported by the proof, since in the proof 
  we only need the first finite number of terms in the sequence $\{t_{nk}\}$ to 
  converge to $0$.
\end{remark}

\subsection{Stolz's Theorem}

\begin{theorem}[$\frac{0}{0}$ type]
  Suppose $\lim_{n \to \infty} a_n = 0$, $\lim_{n \to \infty} b_n = 0$, and 
  $\{a_n\}$ is decreasing. If 
  \[
    \lim_{n \to \infty} \frac{b_{n+1} - b_n}{a_{n+1} - a_n} = l
  \]
  then 
  \[
    \lim_{n \to \infty} \frac{b_n}{a_n} = l
  \]
\end{theorem}

\begin{theorem}[$\frac{*}{\infty}$ type]
  Suppose $\{a_n\}$ is increasing and $\lim_{n \to \infty} a_n = \infty$. If
  \[
    \lim_{n \to \infty} \frac{b_{n+1} - b_n}{a_{n+1} - a_n} = l
  \]
  then
  \[
    \lim_{n \to \infty} \frac{b_n}{a_n} = l   
  \]
\end{theorem}

\begin{proof}
  Method: By Toeplitz's Theorem
  \[
    t_nk = \{\frac{a_1}{a_n}, \frac{a_2 - a_1}{a_n}, \frac{a_3 - a_2}{a_n}, \cdots, \frac{a_n - a_{n-1}}{a_n}\}
  \]
  \[
    c_n = \{\frac{b_1}{a_1}, \frac{b_2 - b_1}{a_2 - a_1}, \frac{b_3 - b_2}{a_3 - a_2}, \cdots, \frac{b_n - b_{n-1}}{a_n - a_{n-1}}\}
  \]
\end{proof}

\subsection{Cauchy's Proposition}

\begin{proposition}[算术平均值形式]
  If $\lim_{n \to \infty} a_n = a$, then
  \[
    \lim_{n \to \infty} \frac{a_1 + a_2 + \cdots + a_n}{n} = a
  \]
\end{proposition}

\begin{proposition}[算术平均值等价形式]
  If $\lim_{n \to \infty} (a_n - a_{n-1}) = a$, then
  \[
    \lim_{n \to \infty} \frac{a_n}{n} = a
  \]
\end{proposition}

\begin{proposition}[几何平均值形式]
  If $\lim_{n \to \infty} a_n = a > 0$, then
  \[
    \lim_{n \to \infty} \sqrt[n]{a_1 a_2 \cdots a_n} = a
  \]
\end{proposition}

\section{方法与技巧}

\subsection{求数列极限}
\begin{itemize}
  \item 定义法:使用的前提是已知极限值。
  \item 夹逼法
  \item Cauchy Proposition, Stolz Theorem, Topelitz Theorem
  \item 单调有界数列收敛原理
  \item Cauchy's Convergence Criterion
\end{itemize}

\subsection{判定数列发散}
\begin{itemize}
  \item 利用数列发散的定义(数列收敛定义的否命题)
  \item 利用收敛数列性质的逆否命题
  \begin{itemize}
    \item 无界数列一定发散
    \item 有两个收敛到不同极限值的子列的数列一定发散
  \end{itemize}
  \item 考察子列特性: 有一个发散子列的数列一定发散
  \item Cauchy's Convergence Criterion
\end{itemize}

\subsection{考察数列单调性}

\subsection{数列放缩技巧}
\begin{itemize}
  \item 加一项减一项结合三角不等式
\end{itemize}

\end{document}