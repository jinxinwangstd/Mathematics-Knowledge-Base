\documentclass[onecolumn]{ctexart}
\usepackage[utf8]{inputenc}
\usepackage{amsmath}
\usepackage{amssymb}
\usepackage{amsthm}
\usepackage{geometry}
\usepackage{graphicx}
\usepackage{float}
\usepackage{xcolor}
\usepackage{listings}
\usepackage{indentfirst}
\usepackage{bm}
\usepackage{tikz}
\usepackage{hyperref} % For the functionality of creating hypertext links. It should be the last package to declare.
\usetikzlibrary{shapes,arrows}
\geometry{a4paper,scale=0.8}

\newtheorem{definition}{Definition}
\newtheorem{theorem}{Theorem}
\newtheorem{proposition}{Proposition}
\newtheorem{lemma}{Lemma}
\newtheorem{corollary}{Corollary}
\newtheorem{remark}{Remark}
\newtheorem{example}{Example}

\DeclareMathOperator{\rank}{rank}

\title{Notes of "Inequalities"}
\author{Jinxin Wang}
\date{}

\begin{document}

\maketitle

\begin{proposition}
  If $x_1 \geq 0$, $x_2 \geq 0$, then the following inequality holds
  \begin{equation}
    |\sqrt{x_1} - \sqrt{x_2}| \leq \sqrt{|x_1 - x_2|}
  \end{equation}
  and the equality holds if and only if $x_1 = x_2$.
\end{proposition}
\begin{proof}
  
\end{proof}
\begin{remark}
  The geometric interpretation is related to the area of squares. Suppose $x_1$ 
  and $x_2$ denote the area of two squares with one common vertex. Then the 平方 
  of the right side is the absolute difference of the area of the two squares, 
  and the 平方 of the left side is the area of the square with the side length as 
  the absolute difference of the side lengths of the squares. Then the inequality 
  is clear.
\end{remark}

\section{排序不等式}
\href{https://en.wikipedia.org/wiki/Rearrangement_inequality}{Wikipedia about rearrangement inequality}

\begin{example}
  Prove that when $n > 1$ it holds that $n! < (\frac{n+2}{\sqrt{6}})^n$
\end{example}

\end{document}