\documentclass[onecolumn]{ctexart}
\usepackage[utf8]{inputenc}
\usepackage{amsmath}
\usepackage{amssymb}
\usepackage{amsthm}
\usepackage{geometry}
\usepackage{graphicx}
\usepackage{float}
\usepackage{xcolor}
\usepackage{listings}
\usepackage{indentfirst}
\usepackage{bm}
\usepackage{tikz}
\usetikzlibrary{shapes,arrows}
\geometry{a4paper,scale=0.8}

\newtheorem{definition}{Definition}
\newtheorem{theorem}{Theorem}
\newtheorem{proposition}{Proposition}
\newtheorem{lemma}{Lemma}
\newtheorem{corollary}{Corollary}
\newtheorem{remark}{Remark}
\newtheorem{example}{Example}

\DeclareMathOperator{\rank}{rank}

\title{Notes of "Basic Definitions and Examples"}
\author{Jinxin Wang}
\date{}

\begin{document}

\maketitle

\section{Continuity of a Function at a Point}

\begin{definition}[Continuity of a Function at a Point with Domain in a Neighborhood of the Point]
  
\end{definition}

\begin{definition}[Continuity of a Function at a Point with General Domain]
  \[
    f:E \to \mathbb{R} \thickspace \textnormal{is continuous at} \thickspace a \in E := 
    (\forall V(f(a))) \exists U_E(a) (f(U_E(a)) \subset V(f(a)))
  \]
\end{definition}
\begin{remark}
  Depending on the kind of point $a$ of the domain $E$:
  \begin{itemize}
    \item If $a$ is an isolated point of $E$, then there exists $U_E(a) = 
    \lbrace a \rbrace$, and $\forall V(f(a))$, $f(U_E(a)) = \lbrace f(a) \rbrace 
    \subset V(f(a))$. Therefore, $f$ is continuous at any isolated point of its 
    domain $E$.
    \item If $a$ is a limit point of $E$, then we have an equivalent definition 
    of continuity at the point:
    \begin{equation}
      (f:E \to \mathbb{R} \thickspace \textnormal{is continuous at} \thickspace 
      a \in E, \textnormal{where $a$ is a limit point of $E$}) \Leftrightarrow \lim_{E \owns x \to a} f(x) = f(a)
    \end{equation}
    \begin{proof}
      (TODO)
    \end{proof}
  \end{itemize}
\end{remark}
\begin{remark}
  Since we can rewrite
  \begin{equation}
    \lim_{E \owns x \to a} f(x) = f(a) = f(\lim_{E \owns x \to a} x)
  \end{equation}
  It leads to the conclusion that continuous functions and only the continuous 
  ones can commute with the operation of passing to the limit at a point (只有连续函数可以与取极限交换运算顺序).
\end{remark}

\begin{remark}
  By the Cauchy criterion we can give another equivalent definition of 
  continuity at a point with the concept of the oscillation of a function at a 
  point.
  \begin{definition}[The Oscillation of a Function at a Point]
    The oscillation of $f:E \to \mathbb{R}$ at $a$, denoted as $\omega(f;a)$, is defined as
    \begin{equation}
      \omega(f;a) = \lim_{\delta \to 0^+} \omega(f;U_E^\delta(a))
    \end{equation}
  \end{definition}
  Then we have the following statement:
  \[
    (f:E \to \mathbb{R} \thickspace \textnormal{is continuous at} \thickspace a \in E) \Leftrightarrow
    (\omega(f;a) = 0)
  \]
\end{remark}

\begin{definition}[Continuity of a Function on a Set]
  
\end{definition}

\section{Points of Discontinuity}

\begin{definition}[Point of Discontinuity]
  If the function $f: E \to \mathbb{R}$ is not continuous at a point of $E$, 
  this point is called a point of discontinuity or simply a discontinuity of $f$.
\end{definition}
\begin{remark}
  A point of discontinuity of a function must belong to the domain of the 
  definition of the function. Continuity or discontinuity of a function at point 
  is not discussed outside of the domain of the function.
\end{remark}

\begin{definition}[Removable of Discontinuity]
  
\end{definition}

\begin{definition}[Discontinuity of First Kind]
  
\end{definition}

\begin{definition}[Discontinuity of Second Kind]
  
\end{definition}

\begin{example}[The Dirichlet Function]
  
\end{example}

\begin{example}[The Riemann Function]
  
\end{example}

\end{document}