\documentclass[onecolumn]{ctexart}
\usepackage[utf8]{inputenc}
\usepackage{amsmath}
\usepackage{amssymb}
\usepackage{amsthm}
\usepackage{geometry}
\usepackage{graphicx}
\usepackage{float}
\usepackage{xcolor}
\usepackage{listings}
\usepackage{indentfirst}
\usepackage{bm}
\usepackage{tikz}
\usetikzlibrary{shapes,arrows}
\geometry{a4paper,scale=0.8}

\newtheorem{definition}{Definition}
\newtheorem{theorem}{Theorem}
\newtheorem{proposition}{Proposition}
\newtheorem{lemma}{Lemma}
\newtheorem{corollary}{Corollary}
\newtheorem{remark}{Remark}
\newtheorem{example}{Example}

\DeclareMathOperator{\rank}{rank}

\title{Notes of "Properties of Continuous Functions"}
\author{Jinxin Wang}
\date{}

\begin{document}

\maketitle

\section{Local Properties}

\section{Global Properties}

\begin{theorem}[The Bolzano-Cauchy Intermediate-Value Theorem]
  
\end{theorem}

\begin{corollary}
  
\end{corollary}

\begin{theorem}[The Weierstrass Maximum-Value Theorem]
  
\end{theorem}

\begin{definition}[Uniform Continuity]
  
\end{definition}

\begin{theorem}[The Contor-Heine Theorem on Uniform Continuity]
  
\end{theorem}

\begin{proposition}
  A continuous mapping $f:E \to \mathbb{R}$ of a closed interval 
  $E = \lbrack a,b \rbrack$ into $\mathbb{R}$ is injective if and only if the 
  function $f$ is strictly monotonic on $\lbrack a,b \rbrack$.
\end{proposition}

\begin{proposition}
  Each strictly monotonic function $f:X \to \mathbb{R}$ defined on a numerical 
  set $X \subset \mathbb{R}$ has an inverse $f^{-1}:Y \to \mathbb{R}$ defined on 
  the set $Y = f(X)$ of values of $f$, and has the same kind of monotonicity on 
  $Y$ that $f$ has on $X$.
\end{proposition}

\begin{proposition}
  The discontinuities of a function $f:E \to \mathbb{R}$ that is monotonic on 
  the set $E \subset \mathbb{R}$ can be only discontinuities of first kind.
\end{proposition}

\begin{corollary}
  If $a$ is a point of discontinuity of a monotonic function $f:E \to \mathbb{R}$, 
  then at least one of the limits $\lim_{E \owns x \to a^+} = f(a^+)$ or 
  $\lim_{E \owns x \to a^-} = f(a^-)$ exists, and strict inequality holds in at 
  least one of the inequalities $f(a^-) \leq f(a) \leq f(a^+)$ when $f$ is 
  nondecreasing and $f(a^-) \geq f(a) \geq f(a^+)$ when $f$ is nonincreasing. 
  The function assumes no values in the open interval defined by the strict 
  inequality. Open intervals of this kind determined by different points of 
  discontinuity have no points in common.
\end{corollary}

\begin{corollary}
  The set of points of discontinuity of a monotonic function is at most 
  countable.
\end{corollary}

\begin{proposition}[A Criterion for Continuity of a Monotonic Function]
  A monotonic function $f:E \to \mathbb{R}$ defined on a closed interval 
  $E = \lbrack a,b \rbrack$ is continuous if and only if its set of values 
  $f(E)$ is the closed interval with endpoints $f(a)$ and $f(b)$.
\end{proposition}

\begin{theorem}[The Inverse Function Theorem]
  A function $f:X \to \mathbb{R}$ that is strictly monotonic
\end{theorem}

\end{document}