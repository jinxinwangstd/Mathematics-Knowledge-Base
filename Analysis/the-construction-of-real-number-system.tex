\documentclass[onecolumn]{article}
\usepackage[utf8]{inputenc}
\usepackage{amsmath}
\usepackage{amssymb}
\usepackage{amsthm}
\usepackage{geometry}
\usepackage{graphicx}
\usepackage{float}
\usepackage{xcolor}
\usepackage{listings}
\usepackage{indentfirst}
\usepackage{bm}
\usepackage{tikz}
\usetikzlibrary{shapes,arrows}
\geometry{a4paper,scale=0.8}

\makeatletter
\renewcommand\@biblabel[1]{}
\renewenvironment{thebibliography}[1]
{\section*{\refname}%
\@mkboth{\MakeUppercase\refname}{\MakeUppercase\refname}%
\list{\@biblabel{\@arabic\c@enumiv}}%
{\settowidth\labelwidth{\@biblabel{#1}}%
\leftmargin\labelwidth
\advance\leftmargin\labelsep
\advance\leftmargin by 2em%
\itemindent -2em%
\@openbib@code
\usecounter{enumiv}
\let\p@enumiv\@empty
\renewcommand\theenumiv{\@arabic\c@enumiv}}%
\sloppy
\clubpenalty4000
\@clubpenalty \clubpenalty
\widowpenalty4000%
\sfcode`\.\@m}
{\def\@noitemerr
{\@latex@warning{Empty `thebibliography' environment}}%
\endlist}
\makeatother

\newtheorem{definition}{Definition}
\newtheorem{theorem}{Theorem}
\newtheorem{proposition}{Proposition}
\newtheorem{corollary}{Corollary}
\newtheorem{remark}{Remark}

\title{The Construction of the Real Number System with Dedekind Cuts}
\author{Jinxin Wang (1230509005)}
\date{October 2023}

\begin{document}

\maketitle

\section{Abstract}

We are going to construct the real number field $\mathbb{R}$ from the rational 
number field $\mathbb{Q}$ using the Dedekind cuts.

\section{Ordered Set and Upper/Lower Bound}

It is one of the most common thing to compare two elements in a set, which 
relies on an order between each pair of elements. Order is important. From one 
perspective, it connects comparable elements with each other. Hence we want to 
generalize the concept of an order of elements in a set, to bring orders to more 
sets in a valid way. For the number sets that we are already familiar with, such 
as $\mathbb{N}$ or $\mathbb{Q}$, the order of their elements seems natural. 
However, as we shall see, the concept of order is abstract in nature. Depending 
on the operation an order relies, we can define different orders on the same 
set.

\begin{definition}[Order]
  Let S be a set. An order on S is a relation, denoted by <, with the following 
  two properties:
  \begin{itemize}
    \item $\forall x, y \in S$, one and only one of the statements is true:
    \[
      x < y, x = y, y < x
    \]
    \item $\forall x, y, z \in S$, if $x < y$ and $y < z$, then $x < z$.
  \end{itemize}
\end{definition}

\begin{definition}[Ordered Set]
  An ordered set is a set S in which an order is defined.
\end{definition}

Therefore, we can define an order in $\mathbb{Q}$ in the following way based on 
the addition operation on $\mathbb{Q}$: $\forall x, y \in \mathbb{Q}$, if 
$x + (-y) < 0$, then $x < y$. In this way, $\mathbb{Q}$ is an ordered set.

With the definition of order and ordered set, we can define upper/lower bound, 
as well as the least upper bound and the greatest lower bound.

\begin{definition}[Upper\&Lower Bound]
  Let S be an ordered set, and a set $E \subset S$. If there exists an 
  $\alpha \in S$ such that $x \leq \alpha$ for every $x \in E$, we say that $E$ 
  is bounded above, and call $\alpha$ an upper bound of E. If there exists a 
  $\beta \in S$ such that $x \geq \beta$ for every $x \in E$, we say that E is 
  bounded below, and $\beta$ is a lower bound of E.
\end{definition}

\begin{definition}[Least Upper Bound and Greatest Lower Bound]
  Let S be an ordered set, and a set $E \subset S$, and E is bounded above. 
  Suppose that there exists an $\alpha \in S$ with the following properties:
  \begin{itemize}
    \item $\alpha$ is an upper bound of E.
    \item $\forall \beta \in S$ and $\beta < \alpha$, $\beta$ is not an upper 
    bound of E.
  \end{itemize}
  then we say that $\alpha$ is the least upper bound of E, or the supremum of E.
  
  Similarly, a set $E \subset S$, and E is bounded below. Suppose that there 
  exists an $\beta \in S$ with the following properties:
  \begin{itemize}
    \item $\beta$ is a lower bound of E.
    \item $\forall \alpha \in S$ and $\beta < \alpha$, $\alpha$ is not a lower 
    bound of E.
  \end{itemize}
  then we say that $\beta$ is the greatest lower bound of E, or the infimum of E.
\end{definition}

\section{Field and Ordered Field}

We need to introduce the concept of a field, which is a kind of algebraic 
structure. A field specifies the valid way to do the four basic arithmetic 
operation with its elements, and get the result which still belongs to the 
field. This complies with our common sense of number systems such as 
$\mathbb{Q}$.

\begin{definition}[Field]
  A field F is a set with two binary operations defined on it, namely addition and
  multiplication. The addition and multiplication should satisfy the following
  properties:
  \begin{itemize}
    \item For addition
    \begin{itemize}
      \item Closed: $\forall x, y \in F$, $x + y \in F$.
      \item Associative: $\forall x, y, z \in F$, $(x + y) + z = x + (y + z)$.
      \item Identity element: $\exists 0 \in F$, $\forall x \in F$, $0 + x = x + 0 = x$.
      \item Inverse element: $\forall x \in F$, $\exists y \in F$, $x + y = y + x = 0$.
      \item Commutative: $\forall x, y \in F$, $x + y = y + x$.
    \end{itemize}
    \item For multiplication
    \begin{itemize}
      \item Closed: $\forall x, y \in F$, $xy \in F$.
      \item Associative: $\forall x, y, z \in F$, $(xy)z = x(yz)$.
      \item Identity element: $\exists 1 \in F$, $\forall x \in F$, $1x = x1 = x$.
      \item Inverse element: $\forall x \in F \setminus \{0\}$, $\exists x^{-1} \in F$ such that
      \[
        x \cdot x^{-1} = x^{-1} \cdot x = 1
      \]
      \item Commutative: $\forall x, y \in F$, $x \cdot y = y \cdot x$
    \end{itemize}
    \item For mixed operation:
    \begin{itemize}
      \item Distributive: $\forall x, y, z \in F$, $x(y + z) = xy + xz$
    \end{itemize}
  \end{itemize}
\end{definition}

For the purpose of this essay, we focus on fields whose elements are numbers, or
number fields.

It is clear that $\mathbb{Q}$ is a field. 

Since a field is a set, it is natural to think about bringing the concept of 
order to a field, which leads to the definition of an ordered field.

\begin{definition}[Ordered Field]
  An ordered field is a field F that has an order defined in it, which satisfies:
  \begin{itemize}
    \item If $x, y, z \in F$ and $y < z$, then $x + y < x + z$.
    \item If $x, y \in F$ and $x > 0$ and $y > 0$, then $xy > 0$.
  \end{itemize}
\end{definition}

\section{Archimedean Property}
We can discuss the Archimedean property on some number groups with an order and 
the multiplication operation defined in it, such as $\mathbb{Z}$ or $\mathbb{Q}$.
It is stated as follows:
\begin{theorem}
  If $x \in S$, $y \in S$, and $x > 0$, then there is a positive integer n such 
  that
  \[
    nx > y
  \]
\end{theorem}

\begin{proposition}
  $\mathbb{Q}$ has the archimedean property.
\end{proposition}
\begin{proof}
  For every $y \in \mathbb{Q}$ that $y \leq 0$, let $n = 1$, then
  \[
    nx = x > 0 \geq y
  \]
  For every $y \in \mathbb{Q}$ that $y > 0$, $\exists a_1, b_1 \in \mathbb{N^+}$ 
  and $(a_1, b_1) = 1$ such that $y = \frac{a_1}{b_1}$. Similarly, 
  $\exists a_2, b_2 \in \mathbb{N^+}$ such that $x = \frac{a_2}{b_2}$. Let 
  $n = [\frac{a_1}{b_1} \cdot \frac{b_2}{a_2}] + 1$, then
  \[
    \begin{split}
      nx &> \frac{a_1}{b_1} \cdot \frac{b_2}{a_2} x \\
         &= \frac{a_1}{b_1} \cdot \frac{b_2}{a_2} \cdot \frac{a_2}{b_2} \\
         &= \frac{a_1}{b_1} \\
         &= y
    \end{split}
  \]
\end{proof}

\section{Define the Real Number Set with Dedekind Cuts}

\begin{definition}
  A Dedekind cut on $\mathbb{Q}$ is any set $\alpha \subset \mathbb{Q}$ with the
  following three properties:
  \begin{description}
    \item{(I)} $\alpha$ is not empty, and $\alpha \neq \mathbb{Q}$.
    \item{(II)} If $p \in \alpha$, $q \in \mathbb{Q}$, and $q < p$, then 
    $q \in \alpha$.
    \item{(III)} If $p \in \alpha$, then $\exists r \in \alpha$ such that $p < r$.
  \end{description}
\end{definition}

\begin{remark}
  \begin{description}
    \item[{1}] For a Dedekind cut, a rational number $q$ either belongs to it or 
    not so.
    \item[{2}] For a Dedekind cut $\alpha$, if $q \in \mathbb{Q}$ and 
    $q \notin \alpha$, then $\forall p \in \alpha$, $p < q$. This is the 
    contraposition of the property (II).
    \item[{3}] For a Dedekind cut $\alpha$, if $q \notin \alpha$ and $p > q$, 
    then $p \notin \alpha$. The reason is that $\forall r \in \alpha$, 
    $p > q > r$.
    \item[{4}] For a Dedekind cut $\alpha$, the distance between $p \in \alpha$ 
    and $q \notin \alpha$ can be smaller than any positive rational number. We 
    can prove it by constructing a series of $p$ and $q$ with gradually smaller 
    distance. Since $\alpha$ is not empty and not $\mathbb{Q}$, there exists 
    $p \in \alpha$ and $q \notin \alpha$, with their distance 
    $d(p, q) = q - p > 0$. Next we check the rational number $\frac{p + q}{2}$. 
    If $\frac{p + q}{2} \in \alpha$, we replace $p$ with $\frac{p + q}{2}$; 
    otherwise, we replace $q$ with $\frac{p + q}{2}$. We repeat this process of 
    checking the arithmetic average and replacing $p$ or $q$ endlessly. Notice 
    that with each iteration, the distance becomes a half of the previous one. 
    Therefore, it can be anyhow small.
    \item[{5}] The property (III) tells us there is no maximum in a Dedekind 
    cut.
  \end{description}
\end{remark}

We define every member of the real number set $\mathbb{R}$ is a Dedekind cut on 
$\mathbb{Q}$.

\section{The Order in the Real Number Set}

\begin{definition}
  $\forall \alpha, \beta \in \mathbb{R}$, $\alpha < \beta$ is true if and only 
  if $\alpha$ is a property subset of $\beta$.
\end{definition}
We need to prove that the above definition is a valid order in a set.
\begin{proof}
  First, we need to prove that for any $\alpha, \beta \in \mathbb{R}$, only one 
  of the following three relations holds:
  \[
    \alpha < \beta, \alpha = \beta, \alpha > \beta
  \]

  $\alpha$ cannot be both a subset of $\beta$ and not a subset of $\beta$, and 
  vice versa. Therefore, at most one of the above relations holds.

  Then we need to prove at least one of the above relations holds. Suppose the 
  first two relations is false, then $\exists p \in \alpha$ such that 
  $p \notin \beta$, which means $\forall q \in \beta$, $q < p$ (the reason why 
  not $q \leq p$ is that if $\exists q \in \beta$ such that $q = p$, then 
  $\exists r \in \beta$ such that $r > q = p$, then $p \in \beta$ which is 
  contradictory to our condition). Hence $\forall q \in \beta$, $q \in \alpha$. 
  Therefore $\beta$ is a subset of $\alpha$, i.e. $\beta < \alpha$.

  Since a proper subset of a proper subset is still a proper subset, it is 
  obvious that if $\alpha < \beta$, $\beta < \gamma$, then $\alpha < \gamma$.
\end{proof}

Now $\mathbb{R}$ is an ordered set.

\subsection{Least-Upper-Bound Property in the Real Number Set}

We claim that the real numbet set $\mathbb{R}$ defined as above has the 
least-upper-bound property.
\begin{proof}
  Let A be a nonempty subset of $\mathbb{R}$, and $\beta \in \mathbb{R}$ be an 
  upper bound of A. Define $\gamma$ to be the union of all $\alpha \in A$. We 
  shall prove that $\gamma \in \mathbb{R}$ and $\gamma = \sup A$.

  First, we need to prove $\gamma \in \mathbb{R}$.
  \begin{description}
    \item[(I)] Since A is nonempty, A contains at least one element 
    $\alpha_0 \in \mathbb{R}$. $\alpha_0$ is nonempty subset of $\mathbb{Q}$, 
    hence $\gamma$ is a nonempty subset of $\mathbb{Q}$. Since $\alpha < \beta$ 
    or $\alpha \subset \beta$ for every $\alpha \in A$, then 
    $\gamma \subset \beta$, hence $\gamma \neq \mathbb{Q}$.
    \item[(II)] Given $p \in \gamma$, $p$ belongs to at least one element of A, 
    let's say $\alpha_0$, which is a Dedekind cut on $\mathbb{Q}$. Since 
    $p \in \alpha_0$, $q < p$, then $q \in \alpha_0$. Hence $q \in \gamma$.
    \item[(III)] Given $p \in \gamma$, $p$ belongs to at least one element of A, 
    let's say $\alpha_0$, which is a Dedekind cut on $\mathbb{Q}$. Since 
    $p \in \alpha_0$, then $\exists r \in \alpha_0$ such that $p < r$. Therefore,
    $r \in \gamma$ such that $p < r$.
  \end{description}

  Next, we need to prove $\gamma$ is an upper bound of $A$. Since 
  $\gamma = \bigcup_{\alpha \in A} \alpha$, $\forall \alpha \in A$, 
  $\alpha \subset \gamma$, i.e. $\alpha \leq \gamma$.

  Last, we need to prove any $\delta \in \mathbb{R}$ such that $\delta < \gamma$ 
  is not an upper bound of $A$. $\delta < \gamma$ means that 
  $\exists p \in \gamma$ and $p \notin \delta$. $p$ belongs to at least one 
  element of $A$, let's say $\alpha_0$. Therefore $p \in \alpha_0$ and 
  $p \notin \delta$. Therefore, $\delta < \alpha_0$.

  Overall, it is proved that $\gamma = \sup A$.
\end{proof}

\section{The Addition Operation in the Real Number Set}

For any $\alpha \in \mathbb{R}$ and $\beta \in \mathbb{R}$, we define 
$\alpha + \beta$ as the set of all sums $r + s$, where $r \in \alpha$ and 
$s \in \beta$. Besides, we define the identity element $0^*$ (to differentiate 
it from the rational number 0) of the addition operation on $\mathbb{R}$ as the 
set of all negative rational numbers. We shall prove that it holds all 
properties required for the addition operation of a field.
\begin{proof}
  First we need to prove that the addition operation is closed in $\mathbb{R}$. 
  In other words, the result of the addition of two Dedekind cuts is also a 
  Dedekind cut.
  \begin{itemize}
    \item Since $\alpha$ and $\beta$ are both nonempty subset of $\mathbb{Q}$, 
    $\alpha + \beta$ is also nonempty. Pick $r' \notin \alpha$, and 
    $s' \notin \beta$. Hence $\forall r \in \alpha$, $r' > r$, and 
    $\forall s \in \beta$, $s' > s$. Therefore $\forall r \in \alpha$, 
    $\forall s \in \beta$, $r' + s' > r + s$, which means 
    $r' + s' \notin (\alpha + \beta)$. Hence $\alpha + \beta \neq \mathbb{Q}$.
    \item If $q \in \mathbb{Q}$ and $\exists p \in (\alpha + \beta)$ such that 
    $q > p$, $p = r + s > q$ for $r \in \alpha$ and $s \in \beta$. Let 
    $s' = q - r < s$, which follows that $s' \in \beta$. Hence $q = r + s'$ 
    where $r \in \alpha$ and $s' \in \beta$. Therefore, $q \in (\alpha + \beta)$.
    \item $\forall p \in (\alpha + \beta)$, $p = r + s$ for $r \in \alpha$ and 
    $s \in \beta$. Therefore, $\exists r' \in \alpha$ such that $r' > r$, and 
    $\exists s' \in \beta$ such that $s' > s$. Hence $p' = r' + s' > p$ and 
    $p' \in (\alpha + \beta)$.
  \end{itemize}
  It is proved that $\forall x \in \mathbb{R}$ and $\forall y \in \mathbb{R}$, 
  $x + y \in \mathbb{R}$.

  Then we need to prove that the addition operation satisfies the associative 
  property. For $\alpha, \beta, \gamma \in \mathbb{R}$, $\alpha + \beta$ is the 
  set of all $r + s$, with $r \in \alpha$, $s \in \beta$. Then 
  $(\alpha + \beta) + \gamma$ is the set of all $(r + s) + t$, with 
  $r \in \alpha$, $s \in \beta$, $t \in \gamma$. By the same definition, 
  $\alpha + (\beta + \gamma)$ is the set of all $r + (s + t)$, with 
  $r \in \alpha$, $s \in \beta$, $t \in \gamma$. Since the addition operation in 
  $\mathbb{Q}$ satisfies the associative property, $(r + s) + t = r + (s + t)$ 
  for all choices of $r \in \alpha$, $s \in \beta$, $t \in \gamma$. Therefore, 
  $(\alpha + \beta) + \gamma = \alpha + (\beta + \gamma)$.

  Similarly, we can prove that the addition operation satisfies the commutative 
  property. For $\alpha, \beta \in \mathbb{R}$, $\alpha + \beta$ is the set of 
  all $r + s$, with $r \in \alpha$, $s \in \beta$. By the same definition, 
  $\beta + \alpha$ is the set of all $s + r$, with $r \in \alpha$, $s \in \beta$. 
  Since the addition operation in $\mathbb{Q}$ satisfies the commutative 
  property, $r + s = s + r$ for all choices of $r \in \alpha$, $s \in \beta$. 
  Therefore, $\alpha + \beta = \beta + \alpha$.

  Next we prove that $0^*$ we defined serves as an identity element in the 
  addition operation in $\mathbb{R}$.
  \begin{itemize}
    \item It is clear that $0^*$ is not empty and not equal to $\mathbb{Q}$. Any 
    rational number smaller than a negative one is also negative. 
    $\forall r \in 0^*$, which means $r < 0$, $\frac{r}{2} < 0$ and 
    $\frac{r}{2} > r$. Therefore, $0^* \in \mathbb{R}$.
    \item Given $\alpha \in \mathbb{R}$, $\forall r \in \alpha$ and 
    $\forall s \in 0^*$, $r + s < r$, which means $(r + s) \in \alpha$. Hence 
    $\alpha + 0^* \subset \alpha$. Conversely, $\forall r \in \alpha$, 
    $\exists r' \in \alpha$ such that $r' > r$. Then $r = r' + (r - r')$ where 
    $r' \in \alpha$ and $r - r' < 0$ and thus $(r - r') \in 0^*$. Hence 
    $\alpha \subset (\alpha + 0^*)$. Therefore, $\alpha + 0^* = \alpha$.
  \end{itemize}

  Finally, we shall prove that every element $\alpha \in \mathbb{R}$ has an 
  inverse element $-\alpha \in \mathbb{R}$ such that $\alpha + (-\alpha) = 0^*$. 
  We define $-\alpha$ as the set of all rational numbers $p$ with the following 
  properties: there exists $r > 0$ such that $-p - r \notin \alpha$. In other 
  words, all rational numbers with some rational numbers smaller than its 
  negative that fails to be in $\alpha$. Let's first prove that 
  $-\alpha \in \mathbb{R}$.
  \begin{itemize}
    \item Since $\alpha \neq \mathbb{Q}$, $\exists s \notin \alpha$. Let 
    $p = -s - 1$, then $-p - 1 = s \notin \alpha$. Hence $p \in -\alpha$. 
    $-\alpha$ is not empty. Since $\alpha$ is not empty, $\exists s' \in \alpha$. 
    Let $p' = -s'$, then $\forall r > 0$, $-p' - r < s'$ and thus 
    $-p' - r \in \alpha$. Hence $p' \notin -\alpha$. $-\alpha \neq \mathbb{Q}$.
    \item For $p \in (-\alpha)$, $\exists r > 0$ such that $-p - r \notin \alpha$, 
    which means $-p - r > s$ for all $s \in \alpha$. If $q \in \mathbb{Q}$ and 
    $q < p$, then $-q - r > -p - r$. Hence $-q - r \notin \alpha$, which follows 
    that $q \in (-\alpha)$.
    \item  For $p \in (-\alpha)$, $\exists r > 0$ such that 
    $-p - r \notin \alpha$. Let $p' = p + \frac{r}{2} > p$, then 
    $-p' - \frac{r}{2} = -p - r \notin \alpha$. Hence $p' \in (-\alpha)$.
  \end{itemize}
  Following that, we prove that $\alpha + (-\alpha) = 0^*$.
  \begin{itemize}
    \item For $p \in (-\alpha)$, $\exists r > 0$ such that $-p - r \notin \alpha$, 
    which means $-p - r > q$ for all $q \in \alpha$. Hence for all choices of 
    $p \in (-\alpha)$ and $q \in \alpha$, $\exists r > 0$ such that 
    $p + q < -r < 0$. Therefore, $\alpha + (-\alpha) \subset 0^*$.
    \item For any negative rational number $s$, put $s' = -\frac{s}{2} > 0$. 
    Based on the archimedean property of $\mathbb{Q}$, there exists an integer 
    $n$ such that $ns' \in \alpha$ but $(n+1)s' \notin \alpha$. Let 
    $p = -(n+2)s'$, then $-p - s' = (n+1)s' \notin \alpha$, and thus 
    $p \in (-\alpha)$. Besides, $ns' + p = -2s' = s$. Therefore, 
    $0^* \subset (\alpha + (-\alpha))$.
  \end{itemize}
\end{proof}

\begin{remark}
  The above proof of the existence of the inverse element of the addition 
  operation relies on the archimedean property of $\mathbb{Q}$.
\end{remark}

\section{The Multiplication Operation in the Real \\ Number Set}

\subsection{Definition in the Positive Real Number Set}

We first define the multiplication operation in $\mathbb{R^+}$, the set of all 
$\alpha \in \mathbb{R}$ such that $\alpha > 0^*$.

If $\alpha, \beta \in \mathbb{R^+}$, we define $\alpha \beta$ as the set of 
all $p \in \mathbb{Q}$ such that $p < rs$ for some choice of $r \in \alpha$, 
$s \in \beta$, $r > 0$, $s > 0$. As for the identity element $1^*$ (to 
differentiate it from the rational number 1) of the multiplication operation, we 
define it as the set of all $p \in \mathbb{Q}$ such that $p < 1$. We shall prove 
that it holds all properties required for the multiplication operation of a 
field.

\begin{proof}
  First we need to prove that the multiplication operation is closed in 
  $\mathbb{R^+}$, i.e. if $\alpha \in \mathbb{R^+}$, $\beta \in \mathbb{R^+}$, 
  then $\alpha \beta \in \mathbb{R^+}$.
  \begin{itemize}
    \item Since $\alpha > 0^*$, $\beta > 0^*$, there exists $r \in \alpha$ and 
    $s \in \beta$ such that $r > 0$ and $s > 0$. Hence $rs > \frac{rs}{2} > 0$, 
    and $\frac{rs}{2} \in \alpha \beta$. Therefore, $\alpha \beta$ is not empty 
    and $\alpha \beta > 0^*$. Pick $r' \notin \alpha$, and $s' \notin \beta$. 
    Hence $\forall r \in \alpha$, $r' > r$, and $\forall s \in \beta$, $s' > s$. 
    Therefore $\forall r \in \alpha$ such that $r > 0$, $\forall s \in \beta$ 
    such that $s > 0$, $r's' > rs$, which means $r's' \notin \alpha \beta$. 
    Hence $\alpha \beta \neq \mathbb{Q}$.
    \item If $p \in \alpha \beta$, there exists some choice of 
    $r \in \alpha \wedge r > 0$ and $s \in \beta \wedge s > 0$ such that $p < rs$. 
    If $q < p$, then $q < rs$. Hence $q \in \alpha \beta$.
    \item If $p \in \alpha \beta$, there exists some choice of 
    $r \in \alpha \wedge r > 0$ and $s \in \beta \wedge s > 0$ such that $p < rs$.
    Consider $q = \frac{p + rs}{2}$. It is clear that $q \in \mathbb{Q}$, 
    $q > p$, $q < rs$ and thus $q \in \alpha \beta$.
  \end{itemize}
  Therefore, $\alpha \beta$ is a Dedekind cut on $\mathbb{Q}$ and 
  $\alpha \beta > 0^*$.

  Next, we will show that the multiplication operation in $\mathbb{R^+}$ has the 
  associative property. $\forall p \in (\alpha \beta) \gamma$, $p < qt$ for some 
  $q \in \alpha \beta \wedge q > 0$ and $t \in \gamma \wedge t > 0$, where 
  $q < rs$ for some $r \in \alpha \wedge r > 0$ and $s \in \beta \wedge s > 0$. 
  Hence $p < rst$, $p < \frac{p + rst}{2} < rst$. Then $\frac{p + rst}{2r} < st$, 
  which means $\frac{p + rst}{2r} \in \beta \gamma$. Thus 
  $p < r \frac{p + rst}{2r}$ where $r \in \alpha$ and 
  $\frac{p + rst}{2r} \in \beta \gamma$, which follows that 
  $p \in \alpha (\beta \gamma)$. Therefore, 
  $(\alpha \beta) \gamma \subset \alpha (\beta \gamma)$. Conversely, we can 
  prove that $\alpha (\beta \gamma) \subset (\alpha \beta) \gamma$ with the same 
  approach. Therefore, $(\alpha \beta) \gamma = \alpha (\beta \gamma)$.

  What comes next is the commutative property. $\forall p \in \alpha \beta$, 
  $p < rs$ for some $r \in \alpha \wedge r > 0$ and $s \in \beta \wedge s > 0$. 
  Thus $p < sr$, which follows that $p \in \beta \alpha$. Hence
  $\alpha \beta \subset \beta \alpha$. Conversely, we can prove that 
  $\beta \alpha \subset \alpha \beta$ with the same approach. Therefore, 
  $\alpha \beta = \beta \alpha$.

  Then we prove that $1^*$ we defined serves as the identity element in the 
  multiplication operation in $\mathbb{R^+}$.
  \begin{itemize}
    \item It is clear that $1^*$ is not empty and is not equal to $\mathbb{Q}$, 
    and $0^* \subsetneq 1^*$. If $p \in 1^*$ and $q < p$, then $q < p < 1$, 
    which follows that $q \in 1^*$. $\forall p \in 1^*$, $p < 1$, then 
    $\exists r = \frac{p+1}{2}$ such that $p < r < 1$ and thus $r \in 1^*$. 
    Therefore, $1^* \in \mathbb{R^+}$.
    \item Given $\alpha \in \mathbb{R^+}$, $\forall p \in \alpha \cdot 1^*$, 
    $p < qr$ where $q \in \alpha \wedge q > 0$ and $r \in 1^* \wedge r > 0$. 
    Thus $0 < r < 1$, which follows that $p = qr < q$. Hence $p \in \alpha$. 
    Therefore, $\alpha \cdot 1^* \subset \alpha$. On the other hand, 
    $\forall p \in \alpha$, $\exists q, r \in \alpha$ such that $q > 0$ and 
    $p < q < r$. Hence $p < q = r \cdot \frac{q}{r}$ where $0 < \frac{q}{r} < 1$ 
    and thus $\frac{q}{r} \in 1^*$. Therefore, $\alpha \subset \alpha \cdot 1^*$. 
    So we can conclude that $\alpha = \alpha \cdot 1^*$.
  \end{itemize}

  Finally, we shall prove that every element $\alpha \in \mathbb{R^+}$ has an 
  inverse element $\beta \in \mathbb{R^+}$ such that $\alpha \beta = 1^*$. We 
  define $\beta$ as the set of all rational numbers $p$ with the following 
  properties: there exists $r > 0$ such that $\frac{1}{p + r} \notin \alpha$. 
  Let's first prove that $\beta \in \mathbb{R^+}$:
  \begin{itemize}
    \item Since $\alpha \neq \mathbb{Q}$, $\exists p \notin \alpha$. Let 
    $q = \frac{1}{p} - 1$, then $\frac{1}{q + 1} = p \notin \alpha$, which 
    follows that $q \in \beta$. Hence $\beta$ is not empty. Since $\alpha > 0^*$, 
    $\exists s \in \alpha \wedge s > 0$. Let $t = \frac{1}{s} > 0$, then 
    $\forall r > 0$, $t + r > t > 0$, $0 < \frac{1}{t + r} < \frac{1}{t} < s$, 
    and thus $t + r \in \alpha$. Hence $\beta \neq \mathbb{Q}$.
    \item If $p \in \beta$, $\exists r > 0$ such that $\forall s \in \alpha$, 
    $\frac{1}{p + r} > s$. If $q < p$, let $r' = r + p - q > 0$, and then 
    $\forall s \in \alpha$, $\frac{1}{q + r'} = \frac{1}{p + r} > s$. Hence 
    $q \in \beta$.
    \item If $p \in \beta$, $\exists r > 0$ such that $\forall s \in \alpha$, 
    $\frac{1}{p + r} > s$. Let $q = p + \frac{r}{2} > p$, and then 
    $\forall s \in \alpha$, $\frac{1}{q + \frac{r}{2}} = \frac{1}{p + r} > s$. 
    Hence $q \in \beta$.
    \item Pick $p \notin \alpha$ and $p > 0$. Then $q = \frac{1}{p} - 1 \geq 0$, 
    and in the above we showed that $q \in \beta$. Therefore $\beta > 0^*$.
  \end{itemize}
  We verified that $\beta \in \mathbb{R^+}$. Following that, we prove 
  $\alpha \beta = 1^*$.
  \begin{itemize}
    \item $\forall p \in \alpha \wedge p > 0$ and 
    $\forall q \in \beta \wedge q > 0$, $\exists r > 0$ such that 
    $\frac{1}{q + r} \notin \alpha$, which follows that $\frac{1}{q + r} > p$. 
    Hence $pq < p (q + r) < 1$. Therefore, $\alpha \beta \subset 1^*$.
    \item $\forall t \in 1^* \wedge p > 0$, based on the archimedean property of 
    $\mathbb{Q}$, there exists an integer $n_0$ such that $n_0(1 - t) > 2t$, 
    which follows that $t < \frac{n_0}{n_0+2} < \frac{n_0}{n_0+1}$. Since the 
    distance between $p \in \alpha$ and $q \notin \alpha$ can be smaller than 
    any positive rational number, there exist some choice of 
    $p \in \alpha \wedge p > 0$ and $q \notin \alpha$ such that 
    $q - p < \frac{p}{n_0}$. The construction method of $p$ and $q$ is as 
    follows: We start with any choice of $p \in \alpha \wedge p > 0$ and 
    $q \notin \alpha$. If they don't satisfy $q - p < \frac{p}{n_0}$, we check 
    whether $\frac{p+q}{2}$ belongs to $\alpha$. If so, we replace the current 
    value of $p$ with $\frac{p+q}{2}$. Otherwise, we replace the current value 
    of $q$ with it. We repeat this process until $q - p < \frac{p}{n_0}$ holds.
    The reason why the iterations will terminate is that with each iteration 
    $p - q$ becomes a half of its previous value, and thus can be smaller than 
    any positive rational number. On the other hand, $p$ is increasing with each 
    iteration, and so is $\frac{p}{n_0}$. Therefore, the construction method 
    works. Then we put $w = \frac{p}{n_0} > 0$, and thus $n_0 w = p \in \alpha$, 
    $(n_0 + 1)w > q \notin \alpha$. Let $s = \frac{1}{(n_0+2)w}$, 
    $r = \frac{1}{(n_0+1)(n_0+2)w} > 0$, then 
    $\frac{1}{s+r} = (n_0+1)w \notin \alpha$, which follows that $s \in \beta$. 
    Hence $t < \frac{n_0}{n_0+2} = ps$, and $t \in \alpha \beta$. With the 
    property of a Dedekind cut, $\forall t \in 1^* \wedge t \leq 0$, 
    $t \in \alpha \beta$. Therefore, $1^* \subset \alpha \beta$.
  \end{itemize}
  We can conclude that $\alpha \beta = 1^*$.
\end{proof}

Now we have the addition operation and the multiplication operation in 
$\mathbb{R^+}$. We need to prove that they obey the distributive law, which is 
stated as $\forall \alpha, \beta, \gamma \in \mathbb{R^+}$, 
$\alpha (\beta + \gamma) = \alpha \beta + \alpha \gamma$.
\begin{proof}
  $\forall p \in \alpha (\beta + \gamma)$, $p < qr$ for some choice of 
  $q \in \alpha \wedge q > 0$ and $r \in (\beta + \gamma) \wedge r > 0$. 
  Further, $r = s + t$ for some choice of $s \in \beta \wedge s > 0$ and 
  $t \in \gamma \wedge t > 0$ (if $s \leq 0$, we can pick 
  $s' \in \alpha \wedge s' > 0$ and put $t' = t - s' + s < t$, hence 
  $r = s' + t'$). Hence $p < q (s + t) = qs + qt$. There exist $s' \in \beta$ 
  such that $s' > s > 0$ and $t' \in \gamma$ such that $t' > t > 0$. Thus 
  $qs < qs'$ and $qt < qt'$, which follows that $qs \in \alpha \beta$ and 
  $qt \in \alpha \gamma$. Hence $qs + qt \in \alpha \beta + \alpha \gamma$, and 
  thus $p \in \alpha \beta + \alpha \gamma$. Therefore, 
  $\alpha (\beta + \gamma) \subset \alpha \beta + \alpha \gamma$.

  $\forall p \in \alpha \beta + \alpha \gamma$, $p = q_1 + q_2$ for some choice 
  of $q_1 \in \alpha \beta$ and $q_2 \in \alpha \gamma$. Furthermore, 
  $q_1 < r_1 s$ for some choice of $r_1 \in \alpha \wedge r_1 > 0$ and 
  $s \in \beta \wedge s > 0$, and $q_2 = r_2 t$ for some choice of 
  $r_2 \in \alpha \wedge r_2 > 0$ and $t \in \gamma \wedge t > 0$. Hence 
  $p < q_1 s + q_2 t$. Suppose $q_1 \leq q_2$, then 
  $p < q_1 s + q_2 t < q_2 (s + t)$ where $s + t \in \beta + \gamma$. Thus 
  $p \in \alpha (\beta + \gamma)$. Therefore, 
  $\alpha \beta + \alpha \gamma \subset \alpha (\beta + \gamma)$.

  As a result, $\alpha \beta + \alpha \gamma = \alpha (\beta + \gamma)$.
\end{proof}

\subsection{Expand the Definition to the Complete Set}

We expand the multiplication operation in $\mathbb{R^+}$ defined above to 
$\mathbb{R}$ by adding the following definition:
\begin{definition}
  \[
    \alpha 0^* = 0^* \alpha = 0^*
  \]
  \[
    \alpha \beta = 
    \begin{cases}
      (-\alpha)(-\beta), \quad \textnormal{if } \alpha < 0^*, \beta < 0^* \\
      -((-\alpha)\beta), \quad \textnormal{if } \alpha < 0^*, \beta > 0^* \\
      -(\alpha(-\beta)), \quad \textnormal{if } \alpha > 0^*, \beta < 0^*       
    \end{cases}
  \]
\end{definition}

Having proved that the multiplication operation we defined in $\mathbb{R^+}$ 
satisfies the properties required for a field, it is not difficult to prove that 
the properties also hold in $\mathbb{R}$ using the identity equation 
$\alpha = -(-\alpha)$. We shall give the proof of the commutative property as an 
example.
\begin{proof}
  According to the definition of the multiplication operation, the commutative 
  property holds when $0^*$ is involved in a multiplication operation.

  If $\alpha < 0^*$ and $\beta < 0^*$, then $-\alpha > 0^*$ and $-\beta > 0^*$. 
  Hence $\alpha \beta = (-\alpha)(-\beta) = (-\beta)(-\alpha) = \beta \alpha$.

  If $\alpha < 0^*$ and $\beta > 0^*$, then $-\alpha > 0^*$. Hence 
  $\alpha \beta = -((-\alpha)\beta) = -(\beta(-\alpha)) = \beta \alpha$.

  If $\alpha > 0^*$ and $\beta < 0^*$, then $-\beta > 0^*$. Hence 
  $\alpha \beta = -(\alpha(-\beta)) = -((-\beta)\alpha) = \beta \alpha$.

  Therefore, $\forall \alpha, \beta \in \mathbb{R}$, it holds that 
  $\alpha\beta = \beta\alpha$.
\end{proof}

Now we conclude that $\mathbb{R}$ is a field.

\section{Other Properties of the Real Number Field}

\subsection{Map Rational Numbers to Rational Cuts}

We map each rational number $r \in \mathbb{Q}$ to the set $r^*$ which consists 
of all rational numbers $p$ such that $p < r$. It is clear that every $r^*$ is a 
Dedekind cut. These cuts mapped to rational numbers satisfy the following 
relations:
\begin{enumerate}
  \item $r^* + s^* = (r + s)^*$
  \item $r^*s^* = (rs)^*$
  \item $r^* < s^*$ if and only if $r < s$
\end{enumerate}
\begin{proof}
  \begin{enumerate}
    \item $\forall p \in r^* + s^*$, $p = u + v$ where $u < r$ and $v < s$. 
    Hence $p < r + s$, which follows that $p \in (r + s)^*$. Therefore, 
    $r^* + s^* \subset (r+s)^*$. Conversely, $\forall p < r + s$, 
    $p = \frac{p + r - s}{2} + \frac{p + s - r}{2}$, where 
    $\frac{p + r - s}{2} < r$ and $\frac{p + s - r}{2} < s$. Hence 
    $p \in r^* + s^*$. Therefore, $(r + s)^* \subset r^* + s^*$. Then we can 
    conclude that $r^* + s^* = (r + s)^*$.
    \item Suppose $r > 0$ and $s > 0$, $\forall p \in r^*s^*$, $p < uv$ where 
    $0 < u < r$ and $0 < v < s$. Hence $p < rs$, which follows that 
    $p \in (rs)^*$. Therefore $r^*s^* \subset (rs)^*$. Conversely, 
    $\forall p < rs$, $p = \sqrt{\frac{pr}{s}}\sqrt{\frac{ps}{r}}$ where 
    $\sqrt{\frac{pr}{s}} < r$ and $\sqrt{\frac{ps}{r}} < s$. Hence 
    $p \in r^*s^*$. Therefore $(rs)^* \subset r^*s^*$. Thus, $r^*s^* = (rs)^*$ 
    for all choice of $r > 0$ and $s > 0$. For multiplication operations 
    involving $0^*$, based on the definition of the multiplication operation, 
    $r^*0^* = 0^*r^* = 0^* = (r0)^* = (0r)^*$. As for the cases involving the 
    cuts of negative rational numbers, we can apply the identity equation 
    $r^* = -(-r)^*$ with $r < 0$. If $r < 0$ and $s < 0$, then 
    $r^*s^* = (-r)^*(-s)^* = ((-r)(-s))^* = (rs)^*$. If $r < 0$ and $s > 0$, 
    then $r^*s^* = -((-r)^*s^*) = -((-r)s)^* = (rs)^*$. If $r > 0$ and $s < 0$, 
    then $r^*s^* = -(r^*(-s)^*) = -(r(-s))^* = (rs)^*$. Now we can conclude that 
    $r^*s^* = (rs)^*$.
    \item If $r < s$, then $\forall p < r$, it also satisfies $p < s$, whereas 
    $r < \frac{r + s}{2} < s$. Hence $r^* \subsetneq s^*$, which means that 
    $r^* < s^*$. If $r^* < s^*$, then $\exists p \in s^*$ such that 
    $\forall q \in r^*$, $p > q$. Hence $p \geq r$ and $p < s$, which follows 
    that $r < s$. Therefore, $r^* < s^*$ if and only if $r < s$.
  \end{enumerate}
\end{proof}

The above relations mean that the mapping between rational numbers and rational 
cuts doesn't change the algebraic relations involving addition, multiplication 
and comparison. This fact can also be expressed as the ordered field $\mathbb{Q}$ 
is isomorphic to the ordered field $\mathbb{Q^*}$ whose elements are rational 
cuts. The isomorphic relation allows us to regard $\mathbb{Q}$ as a subfield of 
$\mathbb{R}$.

\subsection{An Ordered Field}

We claim that $\mathbb{R}$ is an ordered field with the order we defined above.
\begin{proof}
If $\alpha, \beta, \gamma \in \mathbb{R}$ and $\beta < \gamma$, then 
$\beta \subsetneq \gamma$. $\forall p \in \alpha + \beta$, $p = q + r$ for some 
choice of $q \in \alpha$ and $r \in \beta$. Then $r \in \gamma$. Hence 
$p \in \alpha + \gamma$. Therefore, $\alpha + \beta \subset \alpha + \gamma$. On 
the other hand, $\beta \subsetneq \gamma$ implies that $\exists p \in \gamma$ 
such that $\forall q \in \beta$, $p > q$. Based on the definition of a cut, 
there exists $r > 0$ such that $p + r \in \gamma$. Since the distance between a 
member and a non-member of a cut can be any closer, we can find $s \in \alpha$ 
such that $s + r \notin \alpha$. Since $p > q$ for any $q \in \beta$ and 
$s + r > t$ for any $t \in \alpha$, $p + s + r \notin \alpha + \beta$. As 
$s \in \alpha$ and $p + r \in \gamma$, $p + s + r \in \alpha + \gamma$. Hence 
$\alpha + \beta \subsetneq \alpha + \gamma$. Therefore, 
$\alpha + \beta < \alpha + \gamma$.

If $\alpha, \beta \in \mathbb{R}$ and $\alpha > 0^*$, $\beta > 0^*$, then there 
exist $p \in \alpha p > 0$ and $q \in \beta \wedge q > 0$. Hence $pq > 0$ and 
$pq \in \alpha \beta$. Therefore $\alpha \beta > 0^*$. 
\end{proof}

\subsection{The Archimedean Property}

We claim that $\mathbb{R}$ has the archimedean property.
\begin{proof}
  Suppose $\alpha, \beta \in \mathbb{R}$ and $\alpha > 0^*$. Pick 
  $p \in \alpha \wedge p > 0$ and $q \notin \beta$. Since $p, q \in \mathbb{Q}$ 
  and $p > 0$, based on the archimedean property of $\mathbb{Q}$, there exists 
  an integer $n$ such that $np > q$, which follows that $\forall r \in \beta$, 
  $np > r$. Since $np \in (n+1)^* \alpha$, $(n+1)\alpha = (n+1)^*\alpha > \beta$.
\end{proof}

\section{Conclusion}

We can conclude that the Real Number system defined with the Dedekind cuts is an 
ordered field with the least-upper-bound property, the archimedean property, and 
$\mathbb{Q}$ as its subfield.

\begin{thebibliography}{}
\bibliographystyle{ieee}
  
\bibitem{Ref1}[1]
Rudin, Walter. "Principles of Mathematical Analysis, International Series in 
Pure and Applied Mathematics, McGraw-Hil." (1976).
\end{thebibliography}
\end{document}