\documentclass[onecolumn]{ctexart}
\usepackage[utf8]{inputenc}
\usepackage{amsmath}
\usepackage{amssymb}
\usepackage{amsthm}
\usepackage{geometry}
\usepackage{graphicx}
\usepackage{float}
\usepackage{xcolor}
\usepackage{listings}
\usepackage{indentfirst}
\usepackage{bm}
\usepackage{tikz}
\usetikzlibrary{shapes,arrows}
\geometry{a4paper,scale=0.8}

\newtheorem{definition}{Definition}
\newtheorem{theorem}{Theorem}
\newtheorem{proposition}{Proposition}
\newtheorem{corollary}{Corollary}
\newtheorem{remark}{Remark}

\title{Notes of "The Algebra of Vectors"}
\author{Jinxin Wang}
\date{}

\begin{document}

\maketitle

\section{Vectors and Linear Operations}

\subsection{Decomposition of Vectors}

\begin{theorem}[Theorem of Decomposition of Vectors in Plane and Space]
  
\end{theorem}

\subsection{Application in Geometry Problems}

\subsubsection{Vectors in a Plane}

\begin{proposition}
  $\alpha, \beta, \gamma$ lie in a plane $\Leftrightarrow$ $\exists \lambda, \mu, 
  \nu$ that at least one of them is not equal to 0 such that
  \begin{equation}
    \lambda \alpha + \mu \beta + \nu \gamma = 0
  \end{equation}
\end{proposition}

\subsubsection{Points in a Line}

\begin{proposition}
  $A, B, C$ lie in a line $\Leftrightarrow$ $\exists s, t \in \mathbb{R}$ such 
  that
  \begin{equation}
    \vec{OC} = s \vec{OA} + t \vec{OB} \wedge s + t = 1
  \end{equation}
\end{proposition}

\begin{definition}
  $(A, B, C) = \frac{\vec{AC}}{\vec{CB}}$
\end{definition}
\begin{remark}
  $(A, B, C) \in (-\infty, -1) \cup (-1, \infty)$
\end{remark}

\subsubsection{Ceva's Theorem}

\subsubsection{Menelaus' Theorem}

\section{Affine Coordinate System}



\section{Dot Product of Vectors}

\subsection{Definition of Dot Product}

\subsection{Properties of Dot Product}

\subsection{Dot Product in Coordinate Systems}

\subsection{应用}

\subsubsection{余弦定理}

\subsubsection{三条高线交于一点}

\section{Cross Product of Vectors}

\subsection{三个不共面向量的定向}

两个不共线向量$\alpha, \beta$的定向:当从$\alpha$转到$\beta$满足右手螺旋,则为右手系;满足左手螺旋,则为左手系。

三个不同面向量$\alpha, \beta, \gamma$的定向:$\alpha, \beta$决定的平面将空间分成两个半空间。从$\gamma$指向的那一侧
观察,若$\alpha, \beta$组成右手系,则这三个向量组成右手系;若$\alpha, \beta$组成左手系,则这三个向量组成左手系。

Properties of Orientation:
\begin{itemize}
  \item When two vectors have a transposition, the orientation changes.
  \item When a vectors is replaced by its negative one, the orientation changes.
\end{itemize}

\begin{remark}
  It seems to be related to determinant.
\end{remark}

\subsection{Definition of Cross Products of Vectors}

\subsection{Properties of Cross Products of Vectors}

\begin{itemize}
  \item Skew-symmetric
  \begin{equation}
    \alpha \times \beta = - \beta \times \alpha
  \end{equation}
  \item Double linearity
  \begin{equation}
    (\lambda \alpha) \times \beta = \lambda (\alpha \times \beta) = \alpha \times (\lambda \beta)
  \end{equation}
  \begin{equation}
    \alpha \times (\beta + \gamma) = \alpha \times \beta + \alpha \times \gamma, (\alpha + \beta) \times \gamma = \alpha \times \gamma + \beta \times \gamma
  \end{equation}
\end{itemize}

\subsection{Cross Products of Vectors in Coordinate Systems}

\section{Mixed Product of Vectors}

\subsection{Double Cross Product}

\begin{equation}
  (\alpha \times \beta) \times \gamma = (\alpha \cdot \gamma) \beta - (\beta \cdot \gamma) \alpha
\end{equation}
\begin{equation}
  \alpha \times (\beta \times \gamma) = (\alpha \cdot \gamma) \beta - (\alpha \cdot \beta) \gamma
\end{equation}
\begin{proof}
  Hint: 利用一个方便的坐标系来通过坐标进行验证。
\end{proof}
\begin{remark}
  虽然证明过程依赖于坐标系的选取,但最终得到的等式并不依赖于坐标系的选取,甚至不依赖于坐标系,完全是向量之间的关系。
\end{remark}

\subsection{Mixed Product}

几何意义:绝对值是以$\alpha, \beta, \gamma$为边的平行六面体的体积。符号为三个向量组成的基的定向。

Properties of Mixed Product:
\begin{itemize}
  \item $\alpha, \beta, \gamma$ in the same plane $\Leftrightarrow$ $(\alpha, \beta, \gamma) = 0$
  \item $(\alpha, \beta, \gamma) = (\beta, \gamma, \alpha) = (\gamma, \alpha, \beta)$
  \item $\alpha \times \beta \cdot \gamma = (\alpha, \beta, \gamma) = \alpha \cdot \beta \times \gamma$
  \item Triple linearity
\end{itemize}

\subsection{Mixed Product in Coordinate System}

\begin{equation}
  (\alpha, \beta, \gamma) = 
  \begin{vmatrix}
    a_1 & b_1 & c_1 \\
    a_2 & b_2 & c_2 \\
    a_3 & b_3 & c_3
  \end{vmatrix}
  (e_1, e_2, e_3)
\end{equation}

Specially, in a Cartesian coordinate system, $(e_1, e_2, e_3) = 1$. Therefore,
\begin{equation}
  (\alpha, \beta, \gamma) = 
  \begin{vmatrix}
    a_1 & b_1 & c_1 \\
    a_2 & b_2 & c_2 \\
    a_3 & b_3 & c_3
  \end{vmatrix}
\end{equation}

\subsection{Corollary and Application}

\subsubsection{Jacobi Identity}

\begin{equation}
  \alpha \times (\beta \times \gamma) + \beta \times (\gamma \times \alpha) + \gamma \times (\alpha \times \beta) = 0
\end{equation}

\subsubsection{Lagrange Identity}
\begin{equation}
  \alpha \times \beta \cdot \gamma \times \delta = 
  \begin{vmatrix}
    \alpha \cdot \gamma & \alpha \cdot \delta \\
    \beta \cdot \gamma & \beta \cdot \delta
  \end{vmatrix}
\end{equation}
\begin{proof}
  Hint: Use mixed product.
\end{proof}

\subsubsection{Geometric Proof of Cramer Criterion in $3 \times 3$ Linear Systems}


\end{document}