\documentclass[onecolumn]{ctexart}
\usepackage[utf8]{inputenc}
\usepackage{amsmath}
\usepackage{amssymb}
\usepackage{amsthm}
\usepackage{geometry}
\usepackage{graphicx}
\usepackage{float}
\usepackage{xcolor}
\usepackage{listings}
\usepackage{indentfirst}
\usepackage{bm}
\usepackage{tikz}
\usetikzlibrary{shapes,arrows}
\geometry{a4paper,scale=0.8}

\newtheorem{definition}{Definition}
\newtheorem{theorem}{Theorem}
\newtheorem{proposition}{Proposition}
\newtheorem{lemma}{Lemma}
\newtheorem{corollary}{Corollary}
\newtheorem{remark}{Remark}
\newtheorem{example}{Example}

\DeclareMathOperator{\rank}{rank}

\title{Notes of "Analytic Geometry in $\mathbb{R}^3$"}
\author{Jinxin Wang}
\date{}

\begin{document}

\maketitle

\section{图形与方程}

\section{平面方程}

\subsection{Equation of a Plane}

\begin{theorem}
  Given the equation of a plane in the general form: $Ax + By + Cz + D = 0$, a 
  vector $\vec{u} = (u_x, u_y, u_z)$ is parallel to the plane if and only if
  \begin{equation}
    Au_x + Bu_y + Cu_z = 0
  \end{equation}
\end{theorem}
\begin{proof}
  Hint:
\end{proof}

\subsection{Positional Relationship between Planes}
There are only two kinds of positional relationships between planes:
\begin{itemize}
  \item Parallel
  \item Intersected
\end{itemize}

\begin{example}
  The intersection of three planes $\pi_1: $, $\pi_2: $, $\pi_3$ is unique $\Leftrightarrow$
  $\begin{vmatrix}
    A_1 & B_1 & C_1 \\
    A_2 & B_2 & C_2 \\
    A_3 & B_3 & C_3
  \end{vmatrix} \neq 0$
\end{example}

\subsection{Geometric Interpretation of Linear Inequality with 3 Unknowns}

\section{直线方程}

\subsection{Two Types of Equation of a Line}

\subsubsection{The Point-Direction Form of Line Equation}

\subsubsection{The General Form of Line Equation}

\subsubsection{Transformation between Two Types of Line Equations}

Given $\frac{x - x_0}{u_x} = \frac{y - y_0}{u_y} = \frac{z - z_0}{u_z}$, 
the equivalent general form is
\[
  \begin{cases}
    \frac{x - x_0}{u_x} = \frac{y - y_0}{u_y} \\
    \frac{y - y_0}{u_y} = \frac{z - z_0}{u_z}
  \end{cases}
\]
\begin{equation}
  \begin{cases}
    u_y x - u_x y + u_x y_0 - u_y x_0 = 0 \\
    u_z y - u_y z + u_y z_0 - u_z y_0 = 0
  \end{cases} 
\end{equation}

Given $\begin{cases}
  A_1 x + B_1 y + C_1 z + D_1 = 0 \\
  A_2 x + B_2 y + C_2 z + D_2 = 0
\end{cases}$, we can find a point $(x_0, y_0, z_0)$ on the line, and the 
direction vector it is parallel to:
\begin{equation}
  (
  \begin{vmatrix}
    B_1 & C_1 \\
    B_2 & C_2
  \end{vmatrix},
  \begin{vmatrix}
    C_1 & A_1 \\
    C_2 & A_2
  \end{vmatrix},
  \begin{vmatrix}
    A_1 & B_1 \\
    A_2 & B_2
  \end{vmatrix})
\end{equation}

\subsection{直线与平面的位置关系}
位置关系:
\begin{itemize}
  \item 平行不重合
  \item 重合
  \item 相交
\end{itemize}

\subsubsection{The Point-Direction Form}


\subsubsection{The General Form}

Suppose the equation in the general form of a line $l$ is $
\begin{cases}
  A_1 + B_1 + C_1 + D_1 = 0 \\
  A_2 + B_2 + C_2 + D_2 = 0
\end{cases}$, and the equation of a plane $\pi$ is $A_3 + B_3 + C_3 + D_3 = 0$, 
then to determine the positional relationship between them, our method is to 
transform the problem into studying the relationship between the three planes:
\begin{proposition}
  \begin{itemize}
    \item $l \parallel \pi \wedge l \nsubseteq \pi $ $\Leftrightarrow$ $
    \begin{cases}
      A_1 + B_1 + C_1 + D_1 = 0 \\
      A_2 + B_2 + C_2 + D_2 = 0 \\
      A_3 + B_3 + C_3 + D_3 = 0
    \end{cases}$ has no solution.
    \item $l$ intersected with $\pi$ $\Leftrightarrow$ $
    \begin{cases}
      A_1 + B_1 + C_1 + D_1 = 0 \\
      A_2 + B_2 + C_2 + D_2 = 0 \\
      A_3 + B_3 + C_3 + D_3 = 0
    \end{cases}$ has a unique solution $\Leftrightarrow$ $
    \begin{vmatrix}
      A_1 & B_1 & C_1 \\
      A_2 & B_2 & C_2 \\
      A_3 & B_3 & C_3 
    \end{vmatrix} \neq 0$
    \item $l \subset \pi$ $\Leftrightarrow$ $
    \begin{cases}
      A_1 + B_1 + C_1 + D_1 = 0 \\
      A_2 + B_2 + C_2 + D_2 = 0 \\
      A_3 + B_3 + C_3 + D_3 = 0
    \end{cases}$ has infinitely many solutions.
  \end{itemize}
\end{proposition}

\subsection{共轴平面系}
We can approach the relationship that a line belongs to a plane in another 
direction:
\begin{proposition}
  Suppose the equation in the general form of a line $l$ is $
  \begin{cases}
    A_1 + B_1 + C_1 + D_1 = 0 \\
    A_2 + B_2 + C_2 + D_2 = 0
  \end{cases}$, and the equation of a plane $\pi$ is $A_3 + B_3 + C_3 + D_3 = 0$, 
  then $l \subset \pi$ $\Leftrightarrow$ $\exists \lambda, \mu$ which at least 
  one of them are non-zero such that
  \begin{equation}
    \lambda (A_1 + B_1 + C_1 + D_1) + \mu (A_2 + B_2 + C_2 + D_2 = 0) = A_3 + B_3 + C_3 + D_3
  \end{equation}
\end{proposition}

\begin{definition}[共轴平面系]
  
\end{definition}

\subsection{直线与直线的位置关系}
两直线间的位置关系总览,同时也是一种判断两条直线的位置关系的流程:
\begin{itemize}
  \item 方向向量共线
  \begin{itemize}
    \item 平行
    \item 重合
  \end{itemize}
  \item 方向向量不共线
  \begin{itemize}
    \item 相交
    \item 异面
  \end{itemize}
\end{itemize}

\subsubsection{The Point-Direction Form}

$l_1 \parallel l_2$ $\Leftrightarrow$ $u_1 \parallel u_2$

$l_1$与$l_2$异面 $\Leftrightarrow$ $(\vec{M_1M_2}, u_1, u_2) \neq 0$

\begin{proof}
  
\end{proof}

\subsubsection{The General Form}

$l_1 \parallel l_2$ $\Leftrightarrow$ $
\begin{vmatrix}
  A_1 & B_1 & C_1 \\
  A_2 & B_2 & C_2 \\
  A_3 & B_3 & C_3 
\end{vmatrix} = 
\begin{vmatrix}
  A_1 & B_1 & C_1 \\
  A_2 & B_2 & C_2 \\
  A_4 & B_4 & C_4  
\end{vmatrix} = 0$

$l_1$与$l_2$异面 $\Leftrightarrow$ $
\begin{vmatrix}
  A_1 & B_1 & C_1 & D_1 \\
  A_2 & B_2 & C_2 & D_2 \\
  A_3 & B_3 & C_3 & D_3 \\
  A_4 & B_4 & C_4 & D_4
\end{vmatrix} \neq 0$

\begin{proof}
  
\end{proof}

\subsection{Application: Find the Equation of Lines or Planes with Certain Positional Relationships}

\begin{example}
  在一个仿射坐标系中,直线$l_1$有一般方程$
  \begin{cases}
    x - y + z - 1 = 0 \\
    y + z = 0
  \end{cases}$,直线$l_2$过点$M(0, 0, -1)$,平行于向量$\vec{u}(2, 1, -2)$。平面$\pi$的方程为$x + 
  y + z = 0$。求由全体与$l_1, l_2$都相交,并且平行于$\pi$的直线所构成的曲面$S$的方程。

  Solution:
  \begin{itemize}
    \item 平面参数法
    \item 直线参数法
    \item 双直线参数法
    \item 轨迹法
  \end{itemize}
\end{example}

\section{有关平面和直线的度量关系}

\subsection{度量的基础}
\begin{itemize}
  \item 向量的内积运算给出空间中两点之间距离的度量。
  \item 向量的内积和外积运算给出空间中直线之间的角度的度量。
\end{itemize}

\subsection{直角坐标系中图形方程的几何意义}

\subsubsection{平面的一般方程}

\subsubsection{直线的一般方程}

\subsubsection{共轴平面系}

\subsection{距离的度量}
点到直线的距离和点到平面的距离是基础,其它距离都可转化为这两种距离之一。

\subsubsection{点到直线的距离}

\subsubsection{点到平面的距离}

\subsubsection{平行直线之间的距离}

\subsubsection{直线到平行平面的距离}

\subsubsection{平行平面之间的距离}

\subsubsection{异面直线的距离与公垂线的方程}

\begin{definition}
  两异面直线的公垂线定义为与这两条直线都相交且垂直的直线。

  两异面直线之间的距离定义为它们的公垂线与这两条直线的交点的距离。
\end{definition}

求两异面直线之间的距离的方法:
\begin{itemize}
  \item 转化为两点之间的距离:利用参数设出公垂线与两直线的交点,利用公垂线的几何特性(垂直)求出交点坐标,则异面直线的距离等于两交点的距离。
  \item 转化为直线到平行平面的距离:求出过其中一条直线与另一条直线平行的平面,则异面直线距离等于直线到所求平行平面的距离。
\end{itemize}

求两异面直线的公垂线方程的方法:
\begin{itemize}
  \item 利用公垂线与异面直线的交点:同上面利用两点之间的距离求异面直线的距离的方法。
  \item 利用公垂线与异面直线决定的平面:公垂线的方向向量易求,则可求出公垂线与两异面直线各自决定的平面,则这两个平面的交线即为公垂线。
\end{itemize}

\begin{remark}
  \begin{enumerate}
    \item 利用参数设公垂线与异面直线的交点需要直线的点向式方程。一般情况下计算量较小,优先使用。
    \item 利用共轴平面系求各种平面的方法虽然不需要将直线的一般方法转化为点向式方程,但一般情况下计算量仍比利用公垂线交点的方法要大,因此一般不优先使
    用。但其中体现的几何特征还是值得掌握。
  \end{enumerate}
\end{remark}

\subsection{夹角的度量}

\subsubsection{平面与平面之间的夹角}

\subsubsection{直线与直线之间的夹角}

\subsubsection{直线与平面之间的夹角}

\section{旋转面、柱面和锥面}

Generally speaking, 建立一个几何图形的一般方程,要找出图形上点的几何特征,然后将它转化为坐标所要满足的条件,就得到了图形的一般方程。

\subsection{Surface of Revolution旋转面}

\subsubsection{Definition of Surface of Revolution and its Geometric Elements}

\subsubsection{General Method to Find the Equation of a Surface of Revolution}

Given the axis of rotation $l$ and a generatrix $\varGamma$ of a surface of revolution $S$:
\begin{equation}
  M \in S \Leftrightarrow \exists M' \in \varGamma \textnormal{s.t.} \vec{MM'} \perp l \wedge d(M, l) = d(M', l)
\end{equation}

\subsubsection{Surface of Revolution with Straight Generatrix}
\begin{itemize}
  \item cylinder
  \begin{itemize}
    \item Generation: A straight generatrix rotating by an axis parallel to the generatrix
    \item Geometric elements: the axis of rotation and the radius.
  \end{itemize}
  \item cone
  \begin{itemize}
    \item Generation: A straight generatrix rotating by an axis intersecting the generatrix
    \item Geometric elements: the axis of rotation, the vertex, and the half-angle
  \end{itemize}
  \item Rotated hyperboloid of one sheet
  \begin{itemize}
    \item Generation: A straight generatrix rotating by an axis skew with the generatrix and not perpendicular with it.
    \item Geometric elements: the axis of rotation and the generatrix
  \end{itemize}
  \item plane with a circular hole
  \begin{itemize}
    \item Generation: A straight generatrix rotating by an axis skew and perpendicular with it.
  \end{itemize}
\end{itemize}

\subsection{Cylindrical Surface柱面}

\subsubsection{General Method to Find the Equation of a Cylindrical Surface}
Given the vector $\vec{u}(u_x, u_y, u_z)$ to which the axis of rotation is parallel and the directrix $\varGamma$ whose equation is $
\begin{cases}
  F(x, y, z) = 0 \\
  G(x, y, z) = 0
\end{cases}$ of a cylindrical surface $S$, then for any point $M(x, y, z)$
\begin{equation}
  \begin{split}
    M \in S &\Leftrightarrow \exists M' \in \varGamma \thickspace \textnormal{s.t.} \thickspace \vec{MM'} \parallel \vec{u} \\
            &\Leftrightarrow \exists M' \in \varGamma \thickspace \textnormal{s.t.} \thickspace \vec{MM'} = t\vec{u}, t \in \mathbb{R} \\
            &\Leftrightarrow \exists t \in \mathbb{R} \thickspace \textnormal{s.t.} \thickspace 
            \begin{cases}
              F(x + t u_x, y + t u_y, z + t u_z) = 0 \\
              G(x + t u_x, y + t u_y, z + t u_z) = 0
            \end{cases}
  \end{split}
\end{equation}


\subsection{Conical Surface锥面}

\begin{definition}[Geometric Definition of Conical Surfaces]
  A conical surface consists of a series of lines passing through a same point. 
  Each line is called a generatrix. The point is called the apex of the conical 
  surface. A line intersecting with all generatrix is called a directrix of the 
  conical surface.
\end{definition}

The elements to determine a conical surface: the apex and the directrix.

\subsubsection{General Method to Find the Equation of a Conical Surface}
Given the apex $M_0(x_0, y_0, z_0)$ and the directrix $\varGamma$ whose equation is $
\begin{cases}
  F(x, y, z) = 0 \\
  G(x, y, z) = 0
\end{cases}$ of a conical surface $S$, then
\begin{equation}
  \begin{split}
    M \in S &\Leftrightarrow \exists M' \in \varGamma \thickspace \textnormal{s.t.} \thickspace M, M', M_0 \textnormal{in the same line} \\
            &\Leftrightarrow \exists M' \in \varGamma \thickspace \textnormal{s.t.} \thickspace \vec{OM'} = t\vec{OM} + (1 - t)\vec{OM_0}, t \in \mathbb{R} \\
            &\Leftrightarrow \exists t \in \mathbb{R} \thickspace \textnormal{s.t.} \thickspace
            \begin{cases}
              F(tx + (1 - t)x_0, ty + (1 - t)y_0, tz + (1 - t)z_0) = 0 \\
              G(tx + (1 - t)x_0, ty + (1 - t)y_0, tz + (1 - t)z_0) = 0
            \end{cases}
  \end{split}
\end{equation}

\begin{remark}
  The equation derived by the above process doesn't include the apex in its zero 
  set.
\end{remark}
\begin{remark}
  The equation derived by the above process usually has some fractions (分式) with 
  variables in their denominators, e.g. $\frac{4x^2}{z^2} - \frac{y^2}{z^2} = 1$, 
  due to the elimination of parameters. It is tempting to transform it into an 
  equation without fractions, with the benefit of including the apex in its zero 
  set. However, the transformation also has possible risk of including more 
  points not belonging to the conical surface in the zero set.

  Ex1: The equation of a conical surface $\frac{4x^2}{z^2} - \frac{y^2}{z^2} = 1$ 
  with the origin as its apex, can be transformed to $4x^2 - y^2 - z^2 = 0$. 
  Compared with the previous equation, the transformation adds $(0, 0, 0)$ (the 
  apex), and the points in the lines
  $\begin{cases}
    2x - y = 0 \\
    z = 0
  \end{cases}$, and
  $\begin{cases}
    2x + y = 0 \\
    z = 0
  \end{cases}$ to the point set of the conical surface, which is wrong.

  Ex2: The equation of a conical surface $\frac{x^2}{z^2} + \frac{y^2}{z^2} = 3$ 
  with the origin as its apex, can be transformed to $x^2 + y^2 - 3z^2 = 0$. 
  Compared with the previous equation, the transformation adds $(0, 0, 0)$ (the 
  apex) to the point set of the conical surface, which is fine.

  To determine whether such transformation is valid, wen need to work out the 
  points added to the zero set by the elimination of denominators by solving the 
  transformed equation with the denominators set to 0.
\end{remark}

\begin{proposition}
  The graph of a homogeneous equation of degree $n$ is a conical surface with 
  the origin as its apex.
\end{proposition}
\begin{proof}
  Hint:
  \begin{itemize}
    \item Since the equation is a homogeneous equation of degree $n$ (it is 
    clear that $n \geq 1$ otherwise we will have a trivial equation of $0 = 0$), 
    the origin is in its point set.
    \item For any point $M(x, y, z) \in S$, it is true that $M'(\lambda x, 
    \lambda y, \lambda z) \in S, \lambda \in \mathbb{R}$. All these points form 
    a line through the origin.
  \end{itemize}
\end{proof}

\section{Quadric Surface}

\section{Ruled Quadric Surface}

Geometric interpretation of ruled surface: 由一簇直线构成的曲面叫做直纹面。

\subsection{Quadric Cylindrical Surface二次柱面}

Based on the geometric interpretation of cylindrical surfaces, they are a kind 
of ruled surface.

\subsection{Quadric Conical Surface二次锥面}

Based on the geometric interpretation of conical surfaces, they are a kind of 
ruled surface.

\subsection{Hypobolic Paraboloid双曲抛物面}

\subsubsection{Equation of Generatrix in Hypobolic Paraboloid}

Given the standard equation of a hypobolic paraboloid:
\begin{equation}
  \frac{x^2}{a^2} - \frac{y^2}{b^2} = 2z
\end{equation}
\[
  (\frac{x}{a} + \frac{y}{b})(\frac{x}{a} - \frac{y}{b}) = 2z
\]
Let $c = \frac{x}{a} + \frac{y}{b}$, then the line with the equation
\[
  l_c = 
  \begin{cases}
    \frac{x}{a} + \frac{y}{b} = c \\
    c(\frac{x}{a} - \frac{y}{b}) = 2z
  \end{cases}
\]
belongs to the hypobolic paraboloid.

Let $c = \frac{x}{a} - \frac{y}{b}$, then the line with the equation
\[
  l_c' = 
  \begin{cases}
    \frac{x}{a} - \frac{y}{b} = c \\
    c(\frac{x}{a} + \frac{y}{b}) = 2z
  \end{cases}
\]
belongs to the hypobolic paraboloid.

Therefore, we find two collection of straight generatrix in the hypoboloic 
paraboloid:
\begin{equation}
  \begin{split}
    I = \lbrace l_c | c \in \mathbb{R} \rbrace
    I' = \lbrace l_c' | c \in \mathbb{R} \rbrace
  \end{split}
\end{equation}

Rewrite the equations of the two collections of straight generatrix in the
point-direction form.

$l_c$ passes through the point $M_c(ac, 0, \frac{c^2}{2})$, and is parallel to 
the vector $\vec{u_c}(-a, b, -c)$.

$l_c'$ passes through the point $M_c'(ac, 0, \frac{c^2}{2})$, and is parallel to 
the vector $\vec{u_c}(a, b, c)$.

\subsubsection{Properties of Generatrix in Hypobolic Paraboloid}
\begin{itemize}
  \item For every point $P \in S$, each collection of straight generatrix has 
  exactly one generatrix passing through it.
  \item Generatrix from the same collection is parallel to a plane.
  \item Two generatrix from the same collection are skew with each other.
  \item Generatrix from different collections intersect.
  \item There is no straight generatrix belonging to both collections.
  \item All straight generatrix in $S$ belong to either collection.
\end{itemize}

\subsection{Hyperboloid of One Sheet单叶双曲面}

\subsubsection{Equation of Generatrix in Hyperboloid of One Sheet}

Given the standard equation of a hyperboloid of one sheet:
\begin{equation}
  \frac{x^2}{a^2} + \frac{y^2}{b^2} - \frac{z^2}{c^2} = 1
\end{equation}
\[
  (\frac{x}{a} + \frac{z}{c})(\frac{x}{a} - \frac{z}{c}) = (1 + \frac{y}{b})(1 - \frac{y}{b})
\]
\begin{equation}
  l_{s:t} =
  \begin{cases}
    s(\frac{x}{a} + \frac{z}{c}) = t(1 + \frac{y}{b}) \\
    t(\frac{x}{a} - \frac{z}{c}) = s(1 - \frac{y}{b})
  \end{cases}
\end{equation}
\begin{equation}
  l_{s:t}' =
  \begin{cases}
    s(\frac{x}{a} + \frac{z}{c}) = t(1 - \frac{y}{b}) \\
    t(\frac{x}{a} - \frac{z}{c}) = s(1 + \frac{y}{b})
  \end{cases}
\end{equation}

It is apparent that each straight generatrix one-to-one corresponds to a value of $s:t$ where

\begin{equation}
  l_{\theta} =
  \begin{cases}
    \cos\theta(\frac{x}{a} + \frac{z}{c}) = \sin\theta(1 + \frac{y}{b}) \\
    \sin\theta(\frac{x}{a} - \frac{z}{c}) = \cos\theta(1 - \frac{y}{b})
  \end{cases}
\end{equation}
\begin{equation}
  l_{\theta}' =
  \begin{cases}
    \cos\theta(\frac{x}{a} + \frac{z}{c}) = \sin\theta(1 - \frac{y}{b}) \\
    \sin\theta(\frac{x}{a} - \frac{z}{c}) = \cos\theta(1 + \frac{y}{b})
  \end{cases}
\end{equation}

\begin{remark}
  
\end{remark}

Rewrite the equations of the two collections of straight generatrix in the
point-direction form. Take $l_{\theta}$ as an example:
\[
  l_{\theta} =
  \begin{cases}
    \frac{\cos\theta}{a} x - \frac{\sin\theta}{b} y + \frac{\cos\theta}{c} z = \sin\theta \\
    \frac{\sin\theta}{a} x + \frac{\cos\theta}{b} y - \frac{\sin\theta}{c} z = \cos\theta
  \end{cases} 
\]
Let $z = 0$, then we solve the following linear system with two equations and two 
unknowns
\begin{equation} \label{eq:15}
  \begin{cases}
    \frac{\cos\theta}{a} x - \frac{\sin\theta}{b} y = \sin\theta \\
    \frac{\sin\theta}{a} x + \frac{\cos\theta}{b} y = \cos\theta
  \end{cases}  
\end{equation}
and find the only solution $x = a\sin2\theta, y = b\cos2\theta$. Therefore, the 
straight generatrix $l_{\theta}$ pass through the point 
$M_{\theta}(a\sin2\theta, b\cos2\theta, 0)$.

Besides, $l_{\theta}$ is parallel to the vector $\vec{u_{\theta}}(
\begin{vmatrix}
  -\frac{\sin\theta}{b} & \frac{\cos\theta}{c} \\
  \frac{\cos\theta}{b} & -\frac{\sin\theta}{c}
\end{vmatrix},
\begin{vmatrix}
  \frac{\cos\theta}{c} & \frac{\cos\theta}{a} \\
  -\frac{\sin\theta}{c} & \frac{\sin\theta}{a}
\end{vmatrix},
\begin{vmatrix}
  \frac{\cos\theta}{a} & -\frac{\sin\theta}{b} \\
  \frac{\sin\theta}{a} & \frac{\cos\theta}{b}
\end{vmatrix}
)$, which is \\ $\vec{u_{\theta}}(-\frac{\cos2\theta}{bc}, 
\frac{\sin2\theta}{ac}, \frac{1}{ab})$, or in better form $\vec{u_{\theta}}
(-a\cos2\theta, b\sin2\theta, c)$.

\begin{remark}
  Previously I had this doubt that why $l_\theta$ passes through the point 
  $M_\theta$ instead of any other points or more points in the $xOy$ plane. 
  Based on the equation (\ref{eq:15}), there is only one solution, which means 
  $l_\theta$ has only one intersection point with the $xOy$ plane. Geometrically 
  it also makes sense since $l_\theta$ is not parallel to the $xOy$ plane given 
  its $z$-component is non-zero.
\end{remark}

Similarly, we can find that $l_\theta'$ passes through the point $M_\theta'
(a\sin2\theta, -b\cos2\theta, 0)$, and is parallel to the vector 
$\vec{u_\theta'}(a\cos2\theta, b\sin2\theta, -c)$.

\subsubsection{Properties of Generatrix in Hyperboloid of One Sheet}
\begin{itemize}
  \item For every point $P \in S$, each collection of straight generatrix has 
  exactly one generatrix passing through it.
  \item Two generatrix from the same collection are skew with each other.
  \item Three generatrix from the same collection are not parallel to a plane.
  \item Generatrix from different collections are coplanar.
  \item There is no straight generatrix belonging to both collections.
  \item All straight generatrix in $S$ belong to either collection.
\end{itemize}

\subsection{Identification of Ruled Quadric Surface}

There are only four kinds of ruled quadric surface except those within a plane:
\begin{itemize}
  \item Quadric Cylindrical Surface
  \item Quadric Conical Surface
  \item Hypobolic Paraboloid
  \item Hyperboloid of One Sheet
\end{itemize}

We can use the characteristics of these four kinds of ruled quadric surface to 
differentiate them and identify a random ruled quadric surface:
\begin{itemize}
  \item Parallelism of generatrix: There exist parallel generatrix in 
  hyperboloid of one sheet, but any two generatrix in hypobolic paraboloid are 
  not parallel.
  \item Parallelism between generatrix and planes: 
\end{itemize}

\section{方法与技巧}

\subsection{求平面方程}
\begin{itemize}
  \item 直线的一般方程+直线外一点坐标:共轴平面系
  \item 直线的点向式方程+直线外一点坐标:行列式
\end{itemize}

\subsection{求二次直纹面方程}
\begin{itemize}
  \item 表示为动平面交线,然后消参
  \item 表示为动直线上的点,然后消参
\end{itemize}

\end{document}