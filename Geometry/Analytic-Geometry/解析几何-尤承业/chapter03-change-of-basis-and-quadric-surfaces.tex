\documentclass[onecolumn]{ctexart}
\usepackage[utf8]{inputenc}
\usepackage{amsmath}
\usepackage{amssymb}
\usepackage{amsthm}
\usepackage{geometry}
\usepackage{graphicx}
\usepackage{float}
\usepackage{xcolor}
\usepackage{listings}
\usepackage{indentfirst}
\usepackage{bm}
\usepackage{tikz}
\usetikzlibrary{shapes,arrows}
\geometry{a4paper,scale=0.8}

\newtheorem{definition}{Definition}
\newtheorem{theorem}{Theorem}
\newtheorem{proposition}{Proposition}
\newtheorem{lemma}{Lemma}
\newtheorem{corollary}{Corollary}
\newtheorem{remark}{Remark}
\newtheorem{example}{Example}

\DeclareMathOperator{\rank}{rank}

\title{Notes of "Change of Basis and Quadric Surfaces"}
\author{Jinxin Wang}
\date{}

\begin{document}

\maketitle

\section{The General Theory of Change of Basis}

\subsection{Transition Matrix and Change of Basis Formula of Points and Vectors}

\begin{remark}
  由于坐标变换前后的方程描述的是同一个几何对象,因此许多几何性质在坐标变换前后保持不变,比如直线在变换后还是直线。但某些性质也会改变,如由于坐标轴单位的不同,
  不同方向上的比例关系发生了改变。
\end{remark}

\subsection{Change of Basis Formula of a Graph}

\begin{example}[Change of Basis of a Plane]
  
\end{example}

\begin{example}[Change of Basis of a Line]
  Suppose the transition matrix from $I$ to $I'$ is \\ $C = 
  \begin{pmatrix}
    c_{11} & c_{12} & c_{13} \\
    c_{21} & c_{22} & c_{23} \\
    c_{31} & c_{32} & c_{33} \\
  \end{pmatrix}$. The standard equation of a line in $I$ is $\frac{x - x_0}{u_x} 
  = \frac{y - y_0}{u_y} = \frac{z - z_0}{u_z}$. To find the equation of the line 
  in $I'$, we have two methods:
  \begin{itemize}
    \item To utilize the method of change of basis of a plane, we transform the 
    standard equation of the line to the general form:
    \[
      \begin{cases}
        \frac{x - x_0}{u_x} = \frac{y - y_0}{u_y} \\
        \frac{y - y_0}{u_y} = \frac{z - z_0}{u_z}
      \end{cases}
    \]
  \end{itemize}
\end{example}

\subsection{Properties of Transition Matrix}

\subsection{Transition Matrix of Change of Basis Between Cartesian Systems}

\section{Types of Quadric Curve}

\section{Determine the Type and Invariants of a Quadric Curve}

\end{document}