\documentclass[onecolumn]{ctexart}
\usepackage[utf8]{inputenc}
\usepackage{amsmath}
\usepackage{amssymb}
\usepackage{amsthm}
\usepackage{geometry}
\usepackage{graphicx}
\usepackage{float}
\usepackage{xcolor}
\usepackage{listings}
\usepackage{indentfirst}
\usepackage{bm}
\usepackage{tikz}
\usetikzlibrary{shapes,arrows}
\geometry{a4paper,scale=0.8}

\newtheorem{definition}{Definition}
\newtheorem{theorem}{Theorem}
\newtheorem{proposition}{Proposition}
\newtheorem{lemma}{Lemma}
\newtheorem{corollary}{Corollary}
\newtheorem{remark}{Remark}
\newtheorem{example}{Example}

\DeclareMathOperator{\rank}{rank}

\title{Notes of "Types of Quadric Curves"}
\author{Jinxin Wang}
\date{}

\begin{document}

\maketitle

The problem we are going to deal with in this section is to classify quadric 
curves into different types based on their graphs. We already studied three 
types of quadric curves with standard equations before: ellipse, hyperbola, and 
parabola. Are there any other types of quadric curves with different graphs?

From the previous studying experience of the three types of quadric curves, it 
is clear that a standard equation, or a simple equation, reduces the difficulty 
of graphing an equation and thus classifying it into different types of quadric 
curves. Now with the tool of coordinate transformation, given a quadric equation 
in a coordinate system, we can always transform it into another coordinate 
system to make the equation simpler. This is our main method in this section.

Since the standard equations of the three quadric curves are in a right-handed 
Cartesian coordinate system, we are going to transform a general quadric 
equation into a right-handed Cartesian coordinate system. If the equation is in 
an affine coordinate system, the information of the system (such as the vectors 
in the basis) is needed to transform it into another coordinate system. To make 
our problem simpler, in this section we only deal with general quadric equations 
in a right-handed Cartesian coordinate system.

\[
  a_{11}x^2 + 2a_{12} xy + a_{22} y^2 + 2b_1 x + 2b_2 y + c = 0
\]

\section{用转轴变化消去交叉项}

\section{用移轴变换进一步简化方程}

\end{document}