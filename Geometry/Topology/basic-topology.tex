\documentclass[onecolumn]{ctexart}
\usepackage[utf8]{inputenc}
\usepackage{amsmath}
\usepackage{amssymb}
\usepackage{amsthm}
\usepackage{geometry}
\usepackage{graphicx}
\usepackage{float}
\usepackage{xcolor}
\usepackage{listings}
\usepackage{indentfirst}
\usepackage{bm}
\usepackage{tikz}
\usetikzlibrary{shapes,arrows}
\geometry{a4paper,scale=0.8}

\newtheoremstyle{break}
  {\topsep}{\topsep}%
  {\itshape}{}%
  {\bfseries}{}%
  {\newline}{}%
\theoremstyle{break}

\newtheorem{definition}{Definition}
\newtheorem{theorem}{Theorem}
\newtheorem{proposition}{Proposition}
\newtheorem{corollary}{Corollary}
\newtheorem{remark}{Remark}
\newtheorem{example}{Example}

\title{Notes of "Basic Topology"}
\author{Jinxin Wang}
\date{}

\begin{document}

\maketitle

\section{Metric Spaces}

\begin{definition}[Euclidean Space]
  
\end{definition}

\begin{theorem}[Properties of Elements in Euclidean Space]
  
\end{theorem}

\begin{definition}[Distance Function and Metric Space]
  
\end{definition}

\begin{definition}[Segment, Interval, and $k$-Cell]
  
\end{definition}

\section{Point Set}
\begin{definition}[Neighborhood, Limit Point, Isolated Point, and Interior Point]
  
\end{definition}
\begin{remark}
  A point $p$ is a limit point of a set $E$ doesn't require $p \in E$.
\end{remark}

\begin{definition}[Closed, Open, Perfect, Bounded, and Dense Set]
  
\end{definition}

\begin{theorem}
  Every neighborhood is an open set.
\end{theorem}
\begin{proof}
  Hint: Recall the definition of a neighborhood.
\end{proof}

\begin{theorem}[Property of Neighborhoods of a Limit Point]
  If $p$ is a limit point of a set $E$, then every neighborhood of $p$ contains 
  infinitely many points of $E$.
\end{theorem}
\begin{proof}
  Hint: Proof by contradiction.
\end{proof}
The converse is easy to prove by the definition of a limit point. Then it leads 
to another definition of a limit point.
\begin{definition}[The Second Definition of a Limit Point]
  A point $p$ is a limit point of a set E if every neighborhood of $p$ contains 
  infinitely many points of $E$.
\end{definition}
\begin{corollary}
  A finite point set has no limit points.
\end{corollary}

\begin{theorem}[]
  Let \{$E_\alpha$\} be a (finite or infinite) collection of sets $E_\alpha$. Then
  \begin{equation}
    (\bigcup_\alpha E_\alpha)^c = \bigcap_\alpha (E_\alpha^c)
  \end{equation}
\end{theorem}
\begin{proof}
  Hint: To prove the subset relation of both directions, consider a random 
  element in the corresponding side.
\end{proof}

\begin{theorem}[Open and Close for a Set and its Complement]
  A set $E$ is open if and only if its complement is closed.
\end{theorem}
\begin{corollary}
  A set $E$ is closed if and only if its complement is open.
\end{corollary}

\begin{theorem}[Open and Close for Unions and Intersections]
  Suppose $\{G_\alpha\}$ is a collection of open sets and $\{F_\alpha\}$ is a 
  collection of closed sets.
  \begin{itemize}
    \item $\bigcup_\alpha G_\alpha$ is open.
    \item $\bigcap_\alpha F_\alpha$ is closed.
    \item If $\{G_\alpha\}$ has finite number of elements, then 
    $\bigcap_{i=1}^n G_i$ is open.
    \item If $\{F_\alpha\}$ has finite number of elements, then 
    $\bigcup_{i=1}^n F_i$ is closed.
  \end{itemize}
\end{theorem}
\begin{proof}
  Hint:(TODO)
\end{proof}

\begin{definition}[Closure]
  If $X$ is a metric space, $E \subset X$, and $E'$ denotes the set of all limit 
  points of $E$ in $X$, then the closure of $E$ is the set $\bar{E} = E \cup E'$.
\end{definition}

\begin{theorem}[Closure and Closed Set]
  If $X$ is a metric space and $E \subset X$, then
  \begin{itemize}
    \item $\bar{E}$ is closed.
    \item $E = \bar{E}$ if and only if $E$ is closed.
    \item $\bar{E} \subset F$ for every closed set $F \subset X$ and $E \subset 
    F$.
    \item $\bar{E}$ is the smallest closed set that contains $E$.
  \end{itemize}
\end{theorem}
\begin{proof}
  Hint:(TODO)
\end{proof}

\begin{theorem}[Least Upper Bound and Limit Point]
  Let $E$ be a nonempty set of real numbers which is bounded above. Let $y = 
  \sup E$. Then $y \in \bar{E}$. Hence $y \in E$ if $E$ is closed.
\end{theorem}

\begin{definition}[Relative Open Subset]
  Suppose $E \subset Y \subset X$, where $X$ is a metric space. We say that $E$ 
  is open relative to $Y$ if for each $p \in E$ there is an associated $r > 0$ 
  such that $q \in E$ whenever $d(p, q) < r$ and $q \in Y$.
\end{definition}
\begin{remark}
  A set $E$ may be open relative to $Y$ without being an open subset in $X$. For 
  example, a segment $(a, b)$ is not an open subset in $\mathbb{R}^2$ but it is 
  open relative to segments $Y$ containing it, such as $(a-1, b+1)$. In other 
  words, relative open subset is a weaker property than open subset.
\end{remark}

\begin{theorem}[Relation Between Relative Open Subset and Open Subset]
  Suppose $Y \subset X$. A subset $E$ of $Y$ is open relative to $Y$ if and only 
  if $E = Y \cap G$ for some open subset $G$ of $X$.
\end{theorem}
\begin{proof}
  Hint:(TODO)
\end{proof}

\section{Compact Set}
\end{document}
