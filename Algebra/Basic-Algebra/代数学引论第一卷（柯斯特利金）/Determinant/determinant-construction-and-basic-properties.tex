\documentclass[onecolumn]{ctexart}
\usepackage[utf8]{inputenc}
\usepackage{amsmath}
\usepackage{amssymb}
\usepackage{amsthm}
\usepackage{geometry}
\usepackage{graphicx}
\usepackage{float}
\usepackage{xcolor}
\usepackage{listings}
\usepackage{indentfirst}
\usepackage{bm}
\usepackage{tikz}
\usepackage{hyperref} % For the functionality of creating hypertext links. It should be the last package to declare.
\usetikzlibrary{shapes,arrows}
\geometry{a4paper,scale=0.8}

\newtheorem{definition}{Definition}
\newtheorem{theorem}{Theorem}
\newtheorem{proposition}{Proposition}
\newtheorem{lemma}{Lemma}
\newtheorem{corollary}{Corollary}
\newtheorem{remark}{Remark}
\newtheorem{example}{Example}

\DeclareMathOperator{\rank}{rank}

\title{Notes of "Determinant: Construction and Basic Properties"}
\author{Jinxin Wang}
\date{}

\begin{document}

\maketitle

\section{Geometric Background}

\section{Combination-Analytic Method}

\begin{definition}[Determinant]
  The determinant of a matrix $A$ of order $n$ is a number decided by the matrix 
  that is
  \begin{equation}
    \det A = \Sigma_{\sigma \in S_n} \epsilon_\sigma a_{1\sigma(1)} a_{2\sigma(2)} \cdots a_{n\sigma(n)}
  \end{equation}
\end{definition}
\begin{remark}
  For a matrix of order $n$, its determinant is a sum with $n!$ terms, since 
  $S_n$ has $n!$ elements.
\end{remark}

\section{Basic Properties of Determinants}

For the convenience of description, some conventions of notations:
\begin{itemize}
  \item The $i$-th row of a matrix $A$ is denoted by $A_{(i)}$.
  \item The $j$-th column of a matrix $A$ is denoted by $A^{(j)}$.
  \item A matrix of order $n$ can be represented as a column of row vectors: 
  $A = \lbrack A_{(1)}, A_{(2)}, \ldots, A_{(n)} \rbrack$
  \item A matrix of order $n$ can be represented as a row of column vectors: 
  $A = (A^{(1)}, A^{(2)}, \ldots, A^{(n)})$
\end{itemize}

According to the definition of a determinant, $\det$ is a mapping from a square 
matrix $A$ to a number $|A|$. Our mission is to study how the mapping changes 
when we change rows or columns of a matrix.

\begin{definition}[多重线性函数]
  
\end{definition}

\begin{definition}[斜对称函数]
  
\end{definition}

\begin{theorem}[Determinants of a Square Matrix and its Transpose]
  \[
    \det A = \det A^T
  \]
\end{theorem}
\begin{proof}
  Suppose $A = (a_{ij})_{n \times n}$ and $A^T = (a_{ij}')_{n \times n}$, then 
  $a_{ij} = a_{ji}'$.
  \[
    \det A = \Sigma_{\sigma \in S_n} \epsilon_\sigma a_{1\sigma(1)} a_{2\sigma(2)} \cdots a_{n\sigma(n)}
  \]
  If we rearrange the order of factors in each term of $\det A$ by its column index, then
  \[
    \det A = \Sigma_{\sigma \in S_n} \epsilon_\sigma a_{\sigma^{-1}(1)1} a_{\sigma^{-1}(2)2} \cdots a_{\sigma^{-1}(n)n}
  \]
  Recall that $\epsilon_\sigma = \epsilon_{\sigma^{-1}}$. Besides, since $S_n 
  \to S_n: \sigma \mapsto \sigma^{-1}$ is a bijection, 
  \[
    \lbrace \sigma \mid \sigma \in S_n \rbrace = \lbrace \sigma^{-1} \mid \sigma \in S_n \rbrace
  \]
  Therefore,
  \[
    \begin{split}
      \det A &= \Sigma_{\sigma \in S_n} \epsilon_{\sigma^{-1}} a_{\sigma^{-1}(1)1} a_{\sigma^{-1}(2)2} \cdots a_{\sigma^{-1}(n)n} \\
             &= \Sigma_{\sigma \in S_n} \epsilon_{\sigma^{-1}} a'_{1\sigma^{-1}(1)} a'_{2\sigma^{-1}(2)} \cdots a'_{n\sigma^{-1}(n)} \\
             &= \Sigma_{\sigma^{-1} \in S_n} \epsilon_{\sigma^{-1}} a'_{1\sigma^{-1}(1)} a'_{2\sigma^{-1}(2)} \cdots a'_{n\sigma^{-1}(n)} \\
             &= \det A^T
    \end{split}
  \]
\end{proof}
\begin{remark}
  The above property of determinants shows a kind of symmetry between the rows 
  and columns of a determinant. More specifically, if there is a property of 
  determinant with regards of rows, it also holds for columns, and vice versa.
\end{remark}
\begin{remark}
  The symmetry between rows and columns of a determinant is sometimes used to 
  calculate the value of the determinant.
\end{remark}

\begin{theorem}[行列式的多重线性和反对称性]
  The function defined on the set $M_n(\mathbb{R})$ $\det: A \mapsto \det A$ has 
  the following properties:
  \begin{description}
    \item[D1] $\det$ is skew-symmetric with regards of the rows of a square 
    matrix.
    \item[D2] $\det$ is multilinear with regards of the rows of a square matrix.
    \item[D3] $\det E = 1$
  \end{description}
\end{theorem}
\begin{proof}
  
\end{proof}

The function $\det: A \mapsto \det A$ has some more properties that can be 
derived from the above theorem, but we want to show that they holds for any 
function $D: M_n(\mathbb{R}) \to \mathbb{R}$ that is skew-symmetric and 
multilinear with regards of rows of a square matrix (in other words, satisfying 
the above property \textbf{D1} and \textbf{D2}).

\end{document}