\documentclass[onecolumn]{ctexart}
\usepackage[utf8]{inputenc}
\usepackage{amsmath}
\usepackage{amssymb}
\usepackage{amsthm}
\usepackage{geometry}
\usepackage{graphicx}
\usepackage{float}
\usepackage{xcolor}
\usepackage{listings}
\usepackage{indentfirst}
\usepackage{bm}
\usepackage{tikz}
\usetikzlibrary{shapes,arrows}
\geometry{a4paper,scale=0.8}

\newtheorem{definition}{Definition}
\newtheorem{theorem}{Theorem}
\newtheorem{proposition}{Proposition}
\newtheorem{lemma}{Lemma}
\newtheorem{corollary}{Corollary}
\newtheorem{remark}{Remark}
\newtheorem{example}{Example}

\DeclareMathOperator{\rank}{rank}

\title{Notes of "Rank of Matrices"}
\author{Jinxin Wang}
\date{}

\begin{document}

\maketitle

\section{Retrospective of Linear System}

With the notion of linear span, we have a new perspective of a linear system:

In a $m$-dimensional column vector space $\mathbb{R}^m$, consider a subset A with $n$ vectors
\[
  A^{(j)} = \lbrack a_{1j}, a_{2j}, \ldots, a_{mj} \rbrack , j = 1, 2, \ldots, n
\]
and their linear span
\[
  V = \langle A \rangle = \langle A^{(1)}, A^{(2)}, \ldots, A^{(n)} \rangle
\]
For a given vector $B \in \mathbb{R}^m$, two natural questions are as follows:
\begin{enumerate}
  \item Whether it is true that $B \in V$?
  \item If so, how to express $B$ as a linear combination of vectors in $A$?
\end{enumerate}
\begin{remark}
  If $\dim V = n$, then $A = \lbrace A^{(1)}, A^{(2)}, \ldots, A^{(n)} \rbrace$ 
  is a base of $V$. Then the second question above is equivalent to find the 
  coordinate of $B$ under the base $A$.
\end{remark}

Formulate the above two questions in equations:
\begin{equation}
  x_1 \begin{pmatrix}
    a_{11} \\
    a_{21} \\
    \vdots \\
    a_{m1}
  \end{pmatrix} + 
  x_2 \begin{pmatrix}
    a_{12} \\
    a_{22} \\
    \vdots \\
    a_{m2}   
  \end{pmatrix} + 
  \cdots + 
  x_n \begin{pmatrix}
    a_{1n} \\
    a_{2n} \\
    \vdots \\
    a_{mn}      
  \end{pmatrix} = 
  \begin{pmatrix}
    b_{1} \\
    b_{2} \\
    \vdots \\
    b_{m}         
  \end{pmatrix}
\end{equation}
which is equivalent to a linear system:
\begin{equation}
  \begin{cases}
    a_{11} x_1 + a_{12} x_2 + \cdots + a_{1n} x_n = b_1 \\
    a_{21} x_1 + a_{22} x_2 + \cdots + a_{2n} x_n = b_2 \\
    \cdots \\
    a_{m1} x_1 + a_{m2} x_2 + \cdots + a_{mn} x_n = b_m 
  \end{cases}
\end{equation}

Another perspective of a linear system that we already know is the matrix form:
\begin{equation}
  AX = B
\end{equation}
with the coefficient matrix
\begin{equation}
  A =
  \begin{pmatrix}
    a_{11} & a_{12} & \cdots & a_{1n} \\
    a_{21} & a_{22} & \cdots & a_{2n} \\
    \cdots
    a_{m1} & a_{m2} & \cdots & a_{mn} \\
  \end{pmatrix}
\end{equation}
and the augmented matrix
\begin{equation}
  (A | B)
\end{equation}

\section{Rank of Matrices}
\begin{definition}[Column Space, Column Rank, Row Space, and Row Rank of a Matrix]
  Suppose $A$ is a $m \times n$ matrix
  $V_c(A) = \langle A^{(1)}, A^{(2)}, \ldots, A^{(n)}\rangle$

  $r_c(A) = dim V_c(A)$

  $V_r(A) = \langle A_{(1)}, A_{(2)}, \ldots, A_{(n)}\rangle $

  $r_r(A) = dim V_r(A)$
\end{definition}
\begin{remark}
  According to Theorem 2 in the Notes of "Vector Space of Rows and Columns", the 
  dimension of a linear span is determined, then our notion of column rank and 
  row rank of a matrix is well defined, because the definitions of column rank 
  and row rank of a matrix rely on the linear span of column vectors and row 
  vectors of the matrix. In other words, these properties of a 
  matrix exists.
\end{remark}

前面学到过,我们可以对系数矩阵进行两种初等变换,将其化为阶梯型矩阵。注意到关于初等变化的一个重要事实:两类初等变换都是inversible. More 
specifically, the matrix $A'$ transformed from a matrix $A$ with elementary row 
operations can be transformed back to $A$ with the same type of elementary row 
operations.

The above fact leads to the following important observation about the effect of 
elementary row operations on the column rank and row rank of a matrix.
\begin{lemma}
  If a matrix $A'$ is transformed from a matrix $A$ through finite number of 
  elementary row operations, it holds that
  \[
    r_c(A') = r_c(A)
  \]
  \[
    r_r(A') = r_r(A)
  \]
\end{lemma}
\begin{proof}
  Hint: 同解性
\end{proof}

The above lemma helps us answer a natural and important question: whether the 
row rank and the column rank of a matrix is equal or not?
\begin{theorem}
  For a $m \times n$ matrix $A$, it always holds that
  \[
    r_c(A) = r_r(A)
  \]
\end{theorem}
\begin{proof}
  Hint: 化为阶梯型
\end{proof}

\begin{definition}[Rank of a Matrix]
  The rank of a matrix $A$, denoted by $\rank A$, is equal to its column rank, 
  as well as its row rank.
  \[
    \rank A = r_c(A) = r_r(A)
  \]
\end{definition}
\begin{remark}
  一个矩阵的秩是唯一确定的,是它的内在特征,不依赖于任何外界情况。
\end{remark}

\section{Application: Solvable Criterion of a Linear System}

Through transforming a linear system to its row reduce echelon form, we can 
already answer the solvable question of a linear system.

\begin{corollary}
  The number of the major unknowns in a homogenous linear system doesn't rely on 
  the way how it reaches the row reduced echelon form. It is always equal to the 
  rank of its coefficient matrix.
\end{corollary}
\begin{proof}
  Hint: The number of non-zero rows in the row reduced echelon form.
\end{proof}

\begin{theorem}[Rouché–Capelli theorem or Kronecker–Capelli theorem]
  A linear system is solvable if and only if the rank of its coefficient matrix 
  is equal to the rank of its augmented matrix.
\end{theorem}
\end{document}