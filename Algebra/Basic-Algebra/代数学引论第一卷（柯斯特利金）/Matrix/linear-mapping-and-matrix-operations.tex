\documentclass[onecolumn]{ctexart}
\usepackage[utf8]{inputenc}
\usepackage{amsmath}
\usepackage{amssymb}
\usepackage{amsthm}
\usepackage{geometry}
\usepackage{graphicx}
\usepackage{float}
\usepackage{xcolor}
\usepackage{listings}
\usepackage{indentfirst}
\usepackage{bm}
\usepackage{tikz}
\usetikzlibrary{shapes,arrows}
\geometry{a4paper,scale=0.8}

\newtheorem{definition}{Definition}
\newtheorem{theorem}{Theorem}
\newtheorem{proposition}{Proposition}
\newtheorem{lemma}{Lemma}
\newtheorem{corollary}{Corollary}
\newtheorem{remark}{Remark}
\newtheorem{example}{Example}

\DeclareMathOperator{\rank}{rank}

\title{Notes of "Linear Mapping and Matrix Operations"}
\author{Jinxin Wang}
\date{}

\begin{document}

\maketitle

\section{Linear Mapping and Matrix}

\section{Matrix Multiplication}

\subsection{Block Matrix Multiplication}

\section{Matrix Transposition}

\section{Rank of a Product of Matrices}

\section{Square Matrix}

\subsection{Commuting Matrices}

\begin{definition}[Diagonal Matrix and Scalar Matrix]
  A diagonal matrix is a a matrix in which all entries outside the main diagonal 
  are all zero.
\end{definition}

\begin{definition}[Identity Matrix]
  \begin{equation}
    E = (\delta_{ij}), \delta_{ij} = 
    \begin{cases}
      1, i = j \\
      0, i \neq j \\
    \end{cases}
  \end{equation}
\end{definition}

\begin{definition}[Scalar Matrix]
  A scalar matrix is the result of the identity matrix multiplied with a scalar.
\end{definition}

\begin{definition}[Matrix Unit]
  A matrix unit is a matrix with only one nonzero entry with value 1. The matrix 
  unit with the nonzero entry in the $i$-th row and $j$-th column is denoted as 
  $E_{ij}$
\end{definition}
\begin{remark}
  A matrix unit is not necessarily a square matrix.
\end{remark}

\begin{definition}[Commuting Matrices]
  Given two matrices $A, B \in M_n(\mathbb{R})$, $A$ and $B$ are said to commute 
  if $AB = BA$, or equivalently their commutator $\lbrack A, B \rbrack = AB - BA 
  = 0$.
\end{definition}
\begin{remark}
  Notice that commuting matrices must be square matrix, because if $A \in M_{s 
  \times n}, B \in M_{n \times r}, s \neq r$, then $AB \in M_{s \times r}, BA 
  \in M_{r \times s}$, hence it is invalid to compare them.
\end{remark}

\begin{theorem}
  If a matrix $A \in M_n(\mathbb{R})$ commutes with $\forall B \in 
  M_n(\mathbb{R})$, then $A$ is a scalar matrix.
\end{theorem}
\begin{proof}
  Hint:
  \begin{itemize}
    \item If $AE_{12} = E_{12}A$, then $\forall k \neq 1, a_{k1} = 0$, and 
    $\forall k \neq 2, a_{2k} = 0$, and $a_{11} = a_{22}$.
    \item If $AE_{ij} = E_{ij}A$, then $\forall k \neq i, a_{ki} = 0$, and 
    $\forall k \neq j, a_{jk} = 0$, and $a_{ii} = a_{jj}$.
    \item Consider $\forall B \in M_n{\mathbb{R}}, AB = BA$.
  \end{itemize}
\end{proof}
\begin{remark}
  When proving a property applies to any matrix in $M_{m \times n}$, one method 
  is to consider all matrix units in $M_{m \times n}$, since the set of matrix 
  units is a basis of $M_{m \times n}$.
\end{remark}

\subsection{Inverse Matrix}
\begin{lemma}[Uniqueness of Inverse Matrix]
  \[
    A' = A'E = A'(AA'') = (A'A)A'' = A''
  \]
\end{lemma}
\begin{definition}[Inverse Matrix]
  
\end{definition}

\begin{definition}[Non-degenerate Matrix and Degenerate Matrix]
  A matrix $A \in M_n(\mathbb{R})$ is non-degenerate if $\rank A = n$. $A$ is 
  degenerate if $\rank A < n$.
\end{definition}
\begin{remark}
  We only talk about non-degenerate matrices and degenerate matrices when it 
  comes to square matrices.
\end{remark}

\begin{theorem}
  $A \in M_n(\mathbb{R})$ is non-degenerate if and only if $A$ is inversible.
\end{theorem}
\begin{proof}
  Hint: \\
  $\Leftarrow$: Use the rank of the product of two matrices. \\
  $\Rightarrow$: Use the unqiueness of the solution of $AX = 0$, and the 
  transpose of the product of two matrices.
\end{proof}

\begin{corollary}
  If $A \in M_n{\mathbb{R}}$ is inversible, then $A^T$ is inversible, and 
  $(A^T)^{-1} = (A^{-1})^T$.
\end{corollary}

\begin{corollary}
  If $B \in M_m(\mathbb{R})$ and $C \in M_n(\mathbb{R})$ are non-degenerate, 
  $\forall A \in M_{m \times n}(\mathbb{R})$, it holds that
  \[
    \rank BAC = \rank A
  \]
\end{corollary}

\begin{corollary}
  If $A, B \in M_n(\mathbb{R})$, and $AB = E \vee BA = E$, then $B = A^{-1}$.
\end{corollary}

\begin{corollary}
  If $A, B, \ldots, C, D \in M_n(\mathbb{R})$ are non-degenerate, then 
  $AB \cdots CD$ is non-degenerate, and its inverse matrix is
  \[
    (AB \cdots CD)^{-1} = D^{-1} C^{-1} \cdots B^{-1} A^{-1}
  \]
\end{corollary}

\subsection{Calculation of Powers of a Matrix}

\begin{example}[Powers of a Scalar Matrix]
  
\end{example}

\begin{example}
  \[
    A = 
    \begin{pmatrix}
      a & c \\
      0 & b \\
    \end{pmatrix}
  \]
\end{example}

\begin{example}
  \[
    A = 
    \begin{pmatrix}
      0 & 1 \\
      1 & 1 \\
    \end{pmatrix}
  \]
\end{example}

\section{Equivalence Class of Matrices}

\begin{definition}[Elementary Matrix]
  
\end{definition}

\begin{theorem}
  $M_{m \times n}(\mathbb{R})$ has a partition of $\min(m, n) + 1$ equivalence 
  classes. All matrices with its rank as $r$ including the representative 
  element is in a equivalence class.
\end{theorem}

\begin{corollary}
  Every non-degenerate matrix $A \in M_n(\mathbb{R})$ can be expressed as the 
  product of elementary matrices.
\end{corollary}

\section{Calculation of Inverse Matrix}

\section{Solution Space}

\end{document}