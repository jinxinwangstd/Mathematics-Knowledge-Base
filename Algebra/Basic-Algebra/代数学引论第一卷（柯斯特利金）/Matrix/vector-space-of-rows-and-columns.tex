\documentclass[onecolumn]{ctexart}
\usepackage[utf8]{inputenc}
\usepackage{amsmath}
\usepackage{amssymb}
\usepackage{amsthm}
\usepackage{geometry}
\usepackage{graphicx}
\usepackage{float}
\usepackage{xcolor}
\usepackage{listings}
\usepackage{indentfirst}
\usepackage{bm}
\usepackage{tikz}
\usetikzlibrary{shapes,arrows}
\geometry{a4paper,scale=0.8}

\newtheorem{definition}{Definition}
\newtheorem{theorem}{Theorem}
\newtheorem{proposition}{Proposition}
\newtheorem{lemma}{Lemma}
\newtheorem{corollary}{Corollary}
\newtheorem{remark}{Remark}
\newtheorem{example}{Example}

\DeclareMathOperator{\rank}{rank}

\title{Notes of "Vector Space of Rows and Columns"}
\author{Jinxin Wang}
\date{}

\begin{document}

\maketitle

\section{Basic Definitions}

\begin{definition}[Row (Vector) Space and Row Vectors]
  Given $n \in \mathbb{N}$, 
\end{definition}

Properties of Vector Space
\begin{enumerate}
  \item Associative property of addition
  \item Commutative property of addition
  \item Zero element
  \item Inverse element of addition
  \item Unit element in scalars
  \item $\alpha(\beta X) = (\alpha \beta)X$
  \item $(\alpha + \beta)X = \alpha X + \beta X$
  \item $\alpha (X + Y) = \alpha X + \alpha Y$
\end{enumerate}

There are also column vector space and column vectors.

\section{Linear Combination and Linear Span}

\begin{definition}[Linear Combinations of Vectors]
  
\end{definition}

Linear combinations have an interesting property:
\begin{proposition}
  A linear combination of linear combinations of vectors $X_1, X_2, \ldots, X_k 
  \in \mathbb{R}^n$ is also a linear combination of vectors $X_1, X_2, \ldots, 
  X_k \in \mathbb{R}^n$.
\end{proposition}
\begin{proof}
  Hint: Express the linear combination by vectors $X_1, X_2, \ldots, X_k \in 
  \mathbb{R}^n$
\end{proof}

Therefore, if we consider the set $V$ consisting of all linear combinations of 
vectors $X_1, X_2, \ldots, X_k \in \mathbb{R}^n$, it has the following property:
\[
  X, Y \in V \Rightarrow \alpha X + \beta Y \in V, \forall \alpha \beta \in \mathbb{R}
\]

\begin{definition}[Linear Span (Or Linear Hull)]
  
\end{definition}

\begin{definition}[Linear Span of a Subset in $\mathbb{R}^n$]
  Given a subset $S \subset \mathbb{R}^n$, the linear span of $S$, denoted by 
  $\langle S \rangle$, is the set of all linear combinations of any finite numbers of vectors 
  in $S$.
\end{definition}
Some interesting properties of linear spans:
\begin{proposition}
  Suppose $V$ is a linear span in $\mathbb{R}^n$, then
  \begin{itemize}
    \item $\langle V \rangle = V$.
    \item If $S \subset V$, then $\langle S \rangle \subset V$.
  \end{itemize}
\end{proposition}
The second property leads to another definition of the linear span of a subset in $\mathbb{R}$:
\begin{definition}[Linear Span (Or Linear Hull)]
  The linear span of a subset $S \subset \mathbb{R^n}$ is the intersection of 
  all linear spans in $\mathbb{R}^n$ that contains $S$:
  \[
    \langle S \rangle = \bigcap_{S \subset V} V
  \]
\end{definition}
\begin{proof}
  Hint:
  \begin{itemize}
    \item $\langle S \rangle \subset \bigcap_{S \subset V} V$
    \item $\bigcap_{S \subset V} V \subset \langle S \rangle$
    \item $\bigcap_{S \subset V} V$ is a linear span
  \end{itemize}
\end{proof}

\section{Linear Dependence}
The concept of linear combination leads to a kind of relationship between a set 
of vectors:
\begin{definition}[Linear Independent \& Linear Dependent]

\end{definition}
\begin{remark}
  The order of the vectors doesn't affect linear independence because the 
  addition operation in the vector space holds the commutative property.
\end{remark}

\begin{theorem}
  
\end{theorem}

\section{Basis and Dimension}

\begin{definition}[Basis]
  Suppose that $V$ is a non-zero linear span in $\mathbb{R}^n$. A set of vectors
  $X_1, X_2, \ldots, X_r$ is said to be a basis of $V$, if they are linear 
  independent, and their linear span is the same as $V$: $\langle X_1, X_2, 
  \ldots, X_r \rangle = V$.
\end{definition}
\begin{remark}
  The basis of a vector space or a linear span is not unique.
\end{remark}
\begin{remark}
  As we proved before, the linear span of $\lbrace E_{(1)}, E_{(2)}, \ldots, 
  E_{(n)} \rbrace$ is $\mathbb{R}^n$, hence $\lbrace E_{(1)}, E_{(2)}, \ldots,\\ 
  E_{(n)} \rbrace$ is a basis of $\mathbb{R}^n$, and is said to be the standard 
  basis of $\mathbb{R}^n$.
\end{remark}

\begin{proposition}
  Given a basis of a vector space, the linear combination of a vector in the 
  space with the basis is unique, and we call the coefficients of the linear 
  combination the coordinate of the vector under the basis.
\end{proposition}

\begin{lemma}
  Let $V$ be a linear span in $\mathbb{R}$ with a basis of $X_1, X_2, \ldots, 
  X_r$, and $Y_1, Y_2, \ldots, Y_s$ be a set of linear independent vectors in 
  $V$, then $s \leq r$.
\end{lemma}
\begin{proof}
  Hint: Proof by contradiction.
  \begin{itemize}
    \item Consider the definition of $Y_1, Y_2, \ldots, Y_s$ are linear 
    independent.
    \item Expand the definition into a homogeneous linear system.
    \item Discuss the number of solutions of the linear system.
  \end{itemize}
\end{proof}

\begin{theorem}
  
\end{theorem}
\begin{definition}[Dimension of a Linear Span and Maximal Linearly Independent Subset]
  
\end{definition}

\begin{definition}[Rank of a Set of Vectors]
  
\end{definition}
\end{document}