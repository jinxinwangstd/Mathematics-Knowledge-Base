\documentclass[onecolumn]{ctexart}
\usepackage[utf8]{inputenc}
\usepackage{amsmath}
\usepackage{amssymb}
\usepackage{amsthm}
\usepackage{geometry}
\usepackage{graphicx}
\usepackage{float}
\usepackage{xcolor}
\usepackage{listings}
\usepackage{indentfirst}
\usepackage{bm}
\usepackage{tikz}
\usetikzlibrary{shapes,arrows}
\geometry{a4paper,scale=0.8}

\newtheorem{definition}{Definition}
\newtheorem{theorem}{Theorem}
\newtheorem{proposition}{Proposition}
\newtheorem{lemma}{Lemma}
\newtheorem{corollary}{Corollary}
\newtheorem{remark}{Remark}
\newtheorem{example}{Example}

\DeclareMathOperator{\rank}{rank}

\title{Notes of "Polynomials"}
\author{Jinxin Wang}
\date{}

\begin{document}

\maketitle

\section{一元多项式}

\begin{definition}[数域P上的一元多项式]
  
\end{definition}

\begin{definition}[一元多项式相等及零多项式]
  
\end{definition}

\begin{definition}[数域P上的一元多项式环]
  
\end{definition}

\section{整除的概念}

\begin{proposition}[带余除法的成立]
  \begin{equation}
    f(x) = q(x)g(x) + r(x)
  \end{equation}
  $q(x)$ and $r(x)$ are uniquely determined. $r(x ) = 0$ or $\deg(r(x)) < \deg(g(x))$.
\end{proposition}
\begin{remark}[综合除法]
  对于除式是次数为1的多项式的情况,我们有一个快速进行带余除法的技巧,叫做综合除法。它基于以下观察
  \[
    f(x) = \Sigma_{i=0}^n a_i x^i, g(x) = x - c, q(x) = \Sigma_{i=0}^{n-1} b_i x^i, r(x) = r
  \]
  \[
    q(x)g(x) + r(x) = b_{n-1} x^n + (b_{n-2} - c b_{n-1}) x^{n-1} + (b_{n-3} - c b_{n-2}) x^{n-2} + \cdots + (b_0 - c b_1) x + r - c b_0 = \Sigma_{i=0}^n a_i x^i = f(x)
  \]
  则系数间有如下关系:
  \[
    \begin{cases}
      b_{n-1} = a_n \\
      b_{n-2} = a_{n-1} + c b_{n-1} \\
      b_{n-3} = a_{n-2} + c b_{n-2} \\
      \ldots \\
      b_1 = a_2 + c b_2 \\
      b_0 = a_1 + c b_1 \\
      r = a_0 + c b_0
    \end{cases}
  \]
  根据上述关系,我们可以快速得到带余除法的结果。

  注意上述关系只在除式$g(x) = x + c$即一次项系数为$1$才成立,对于一次项系数不是$1$的除式,则有$g(x) = c_1 x + c_0 = c_1 (x + \frac{c_0}{c_1}) = c_1 g'(x)$.
\end{remark}


\begin{definition}[整除]
  
\end{definition}

Properties of Divisors:
\begin{itemize}
  \item 对偶等价于相差常数倍
  \item 传递性
  \item 整除组合
\end{itemize}

\section{最大公因式}

\begin{definition}[Common Divisor and Greatest Common Divisor]
  If $\phi(x)$ is a divisor of both $f(x)$ and $g(x)$, we say it is a common 
  divisor of $f(x)$ and $g(x)$.

  Suppose $f(x) \in P\lbrack x \rbrack$ and $g(x) \in P\lbrack x \rbrack$. A 
  polynomial $d(x) \in P\lbrack x \rbrack$ is the greatest common divisor of 
  $f(x)$ and $g(x)$ if the following conditions are true:
  \begin{itemize}
    \item $d(x)$ is a common divisor of $f(x)$ and $g(x)$.
    \item Every common divisor of $f(x)$ and $g(x)$ is a divisor of $d(x)$.
  \end{itemize}
\end{definition}
\begin{remark}
  Since $\forall f(x) \in P\lbrack x \rbrack$ is a divisor of the zero 
  polynomial, in other words $0 = 0 \cdot f(x)$, the greatest common divisor of 
  $f(x)$ and $0$ is $0$. Especially, the greatest common divisor of $0$ and $0$ 
  is $0$, which is in accordance with the definition of GCD.
\end{remark}

\begin{lemma}[Common Divisors in Euclidean Division]
  If it holds that $f(x) = q(x)g(x) + r(x)$ for $f(x) \in P\lbrack x \rbrack$ 
  and $g(x) \in P\lbrack x \rbrack$, then the two pairs of polynomials $(f(x), 
  g(x))$ and $(g(x), r(x))$ have the same common divisors.
\end{lemma}
\begin{proof}
  
\end{proof}

\begin{theorem}[Theorem of Polynomial Greatest Common Divisor]
  $\forall f(x) \in P\lbrack x \rbrack$ and $\forall g(x) \in P\lbrack x \rbrack$, 
  there exists $d(x) \in P\lbrack x \rbrack$ that is the greatest common divisor 
  of $f(x)$ and $g(x)$, and $d(x)$ can be expressed as a combination of $f(x)$ 
  and $g(x)$, which is $\exists u(x) \in P\lbrack x \rbrack, v(x) \in P\lbrack x 
  \rbrack$ such that
  \[
    d(x) = u(x)f(x) + v(x)g(x)
  \]
\end{theorem}
\begin{proof}
  (TODO)
\end{proof}

\begin{remark}[Euclid's Algorithm]
  
\end{remark}

\begin{definition}[Coprime]
  (f(x), g(x)) = 1
\end{definition}

\begin{theorem}[互素的等价条件]
  \[
    ((f(x), g(x)) = 1) \Leftrightarrow u(x)f(x) + v(x)g(x) = 1
  \]
\end{theorem}
\begin{remark}
  This theorem contains a kind of symmetry because both $f(x)$ and $g(x)$ and $u(x)$ and $v(x)$ are coprime.
\end{remark}

\begin{theorem}
  If $(f(x), g(x)) = 1$, and $f(x) | g(x)h(x)$, then $f(x) | h(x)$.
\end{theorem}

\begin{corollary}
  If $f_1(x) | g(x)$, $f_2(x) | g(x)$, and $(f_1(x), f_2(x)) = 1$, then $f_1(x)f_2(x) | g(x)$.
\end{corollary}

上述结论均可推广到多个多项式。

\section{因式分解定理}
\begin{definition}[不可约多项式]
  
\end{definition}
\begin{remark}
  由定义可知,$p(x) \in P\lbrack x \rbrack$ $\Leftrightarrow$ $p(x)$的因式只有$c \neq 0$和
  $c p(x) (c \neq 0)$. 因此,$\forall f(x) \in P\lbrack x \rbrack$, $(p(x), f(x)) = 1$ or 
  $(p(x), f(x)) = p(x)$. In other words, either $p(x)$ and $f(x)$ are coprime, or $p(x)$ 
  is a divisor of $f(x)$.
\end{remark}

\begin{theorem}[不可约多项式作因式]
  Suppose $p(x) \in P\lbrack x \rbrack$ is a irreducible polynomial. Given two 
  polynomials $f(x), g(x) \in P\lbrack x \rbrack$, if $p(x) | f(x)g(x)$, then 
  $p(x) | f(x)$ or $p(x) | g(x)$.
\end{theorem}
\begin{proof}
  
\end{proof}

\begin{theorem}[因式分解定理]
  
\end{theorem}
\begin{proof}
  Hint: 存在性证明利用对被分解的多项式的次数进行数学归纳法。唯一性证明利用对分解的不可约多项式的个数进行数学归纳法。
\end{proof}

\begin{definition}[标准分解式]
  
\end{definition}

\section{重因式}

\begin{definition}[k重因式、重因式和单因式]
  
\end{definition}

\begin{theorem}
  如果不可约多项式$p(x)$是$f(x)$的k重因式,那么它是$f'(x)$的$k-1$重因式。
\end{theorem}
\begin{remark}
  这个定理的逆命题并不成立,即若不可约多项式$p(x)$是$f'(x)$的$k$重因式,并不能推出$p(x)$是$f(x)$的$k+1$重因式。
  反例:Given $f(x) = \frac{1}{3}x^3 - x$, then $f'(x) = x^2 - 1$, and $(x - 1)$ is a divisor of 
  $f'(x)$, but it is not a divisor of $f(x)$.
\end{remark}

\begin{corollary}
  If irreducible polynomial $p(x)$ 是$f(x)$的$k$重因式,则$p(x)$是$f^{(1)}(x), f^{(2)}(x), \ldots, 
  f^{(k-1)}(x)$的因式,但不是$f^{(k)}(x)$的因式。
\end{corollary}

\begin{corollary}
  $p(x)$是$f(x)$的重因式 $\Leftrightarrow$ $p(x)$是$f(x)$和$f'(x)$的公因式
\end{corollary}

\begin{corollary}
  $f(x)$没有重因式 $\Leftrightarrow$ $f(x)$和$f'(x)$互素
\end{corollary}
\begin{remark}
  这个推论给出了一个判别一个多项式$f(x)$是否有重因式的方法,即判断$f(x)$和$f'(x)$是否互素,具体方法是
  Euclid's Algorithm. The process is even mechanical.
\end{remark}

\section{多项式函数}

\begin{definition}[函数的根和零点]
  
\end{definition}

\begin{theorem}[余数定理]
  
\end{theorem}
\begin{corollary}
  $x = \alpha$ is a root of $f(x)$ $\Leftrightarrow$ $(x - \alpha) | f(x)$
\end{corollary}

\begin{definition}[重根]
  
\end{definition}

\begin{theorem}[非零多项式根的个数的最大值]
  A non-zero polynomial $f(x) \in P\lbrack x \rbrack$ with the degree of $n (n 
  \geq 0)$ has at most $n$ roots.
\end{theorem}

\begin{theorem}[多项式共点定理]
  如果多项式$f(x), g(x)$的次数都不超过$n$,而它们对$n+1$个不同的数
  $\alpha_1, \alpha_2, \ldots, \alpha_{n+1}$有相同的值,
  即$f(\alpha_i) = g(\alpha_i), i=1,2,\ldots,n+1$,则$f(x) = g(x)$。
\end{theorem}
\begin{proof}
  
\end{proof}

\begin{example}[Lagrange Interpolating Polynomial]
  
\end{example}

\section{复系数与实系数多项式的因式分解}

\subsection{复系数多项式}

\begin{theorem}[Fundamental Theorem of Algebra]
  每个次数大于等于1的复系数多项式在复数域中有一根。
\end{theorem}

\begin{theorem}[复系数多项式因式分解定理]
  每一个次数大于等于1的复系数多项式在复数域上都可以唯一地分解为一次因式的乘积。
\end{theorem}

\subsection{实系数多项式}

\begin{lemma}[实系数多项式根的共轭性]
  在复数域上,若$x = \alpha$是一个实系数多项式的根,则$\bar{x} = \bar{\alpha}$也是这个实系数多项式的一个根。
\end{lemma}

\begin{theorem}[实系数多项式因式分解定理]
  每一个次数大于等于1的实系数多项式在实数域上都可以唯一地分解为一次因式和二次不可约因式的乘积。
\end{theorem}
\section{有理系数多项式}

\section{多元多项式}

\section{对称多项式}
\end{document}