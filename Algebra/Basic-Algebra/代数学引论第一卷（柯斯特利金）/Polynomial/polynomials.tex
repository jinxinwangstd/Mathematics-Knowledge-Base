\documentclass[onecolumn]{ctexart}
\usepackage[utf8]{inputenc}
\usepackage{amsmath}
\usepackage{amssymb}
\usepackage{amsthm}
\usepackage{geometry}
\usepackage{graphicx}
\usepackage{float}
\usepackage{xcolor}
\usepackage{listings}
\usepackage{indentfirst}
\usepackage{bm}
\usepackage{tikz}
\usetikzlibrary{shapes,arrows}
\geometry{a4paper,scale=0.8}

\newtheorem{definition}{Definition}
\newtheorem{theorem}{Theorem}
\newtheorem{proposition}{Proposition}
\newtheorem{lemma}{Lemma}
\newtheorem{corollary}{Corollary}
\newtheorem{remark}{Remark}
\newtheorem{example}{Example}

\DeclareMathOperator{\rank}{rank}

\title{Notes of "Polynomials"}
\author{Jinxin Wang}
\date{}

\begin{document}

\maketitle

\section{一元多项式}

\begin{definition}[数域P上的一元多项式]
  
\end{definition}

\begin{definition}[一元多项式相等及零多项式]
  
\end{definition}

\begin{definition}[数域P上的一元多项式环]
  
\end{definition}

\section{整除的概念}

\begin{proposition}[带余除法的成立]
  \begin{equation}
    f(x) = q(x)g(x) + r(x)
  \end{equation}
  $q(x)$ and $r(x)$ are uniquely determined. $r(x ) = 0$ or $\deg(r(x)) < \deg(g(x))$.
\end{proposition}
\begin{remark}[综合除法]
  
\end{remark}


\begin{definition}[整除]
  
\end{definition}

Properties of Divisors:
\begin{itemize}
  \item 对偶等价于相差常数倍
  \item 传递性
  \item 整除组合
\end{itemize}

\section{最大公因式}

\begin{definition}[Common Divisor and Greatest Common Divisor]
  If $\phi(x)$ is a divisor of both $f(x)$ and $g(x)$, we say it is a common 
  divisor of $f(x)$ and $g(x)$.

  Suppose $f(x) \in P\lbrack x \rbrack$ and $g(x) \in P\lbrack x \rbrack$. A 
  polynomial $d(x) \in P\lbrack x \rbrack$ is the greatest common divisor of 
  $f(x)$ and $g(x)$ if the following conditions are true:
  \begin{itemize}
    \item $d(x)$ is a common divisor of $f(x)$ and $g(x)$.
    \item Every common divisor of $f(x)$ and $g(x)$ is a divisor of $d(x)$.
  \end{itemize}
\end{definition}
\begin{remark}
  Since $\forall f(x) \in P\lbrack x \rbrack$ is a divisor of the zero 
  polynomial, in other words $0 = 0 \cdot f(x)$, the greatest common divisor of 
  $f(x)$ and $0$ is $0$. Especially, the greatest common divisor of $0$ and $0$ 
  is $0$, which is in accordance with the definition of GCD.
\end{remark}

\begin{lemma}[Common Divisors in Euclidean Division]
  If it holds that $f(x) = q(x)g(x) + r(x)$ for $f(x) \in P\lbrack x \rbrack$ 
  and $g(x) \in P\lbrack x \rbrack$, then the two pairs of polynomials $(f(x), 
  g(x))$ and $(g(x), r(x))$ have the same common divisors.
\end{lemma}
\begin{proof}
  
\end{proof}

\begin{theorem}[Theorem of Polynomial Greatest Common Divisor]
  $\forall f(x) \in P\lbrack x \rbrack$ and $\forall g(x) \in P\lbrack x \rbrack$, 
  there exists $d(x) \in P\lbrack x \rbrack$ that is the greatest common divisor 
  of $f(x)$ and $g(x)$, and $d(x)$ can be expressed as a combination of $f(x)$ 
  and $g(x)$, which is $\exists u(x) \in P\lbrack x \rbrack, v(x) \in P\lbrack x 
  \rbrack$ such that
  \[
    d(x) = u(x)f(x) + v(x)g(x)
  \]
\end{theorem}
\begin{proof}
  (TODO)
\end{proof}

\begin{remark}[Euclid's Algorithm]
  
\end{remark}

\end{document}