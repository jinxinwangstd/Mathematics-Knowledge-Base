\documentclass[onecolumn]{ctexart}
\usepackage[utf8]{inputenc}
\usepackage{amsmath}
\usepackage{amssymb}
\usepackage{amsthm}
\usepackage{geometry}
\usepackage{graphicx}
\usepackage{float}
\usepackage{xcolor}
\usepackage{listings}
\usepackage{indentfirst}
\usepackage{bm}
\usepackage{tikz}
\usetikzlibrary{shapes,arrows}
\geometry{a4paper,scale=0.8}

\newtheorem{definition}{Definition}
\newtheorem{theorem}{Theorem}
\newtheorem{proposition}{Proposition}
\newtheorem{lemma}{Lemma}
\newtheorem{corollary}{Corollary}
\newtheorem{remark}{Remark}
\newtheorem{example}{Example}

\DeclareMathOperator{\rank}{rank}

\title{Notes of "Polynomials"}
\author{Jinxin Wang}
\date{}

\begin{document}

\maketitle

\section{一元多项式}

\begin{definition}[数域P上的一元多项式]
  
\end{definition}

\begin{definition}[一元多项式相等及零多项式]
  
\end{definition}

\begin{definition}[数域P上的一元多项式环]
  
\end{definition}

\section{整除的概念}

\begin{proposition}[带余除法的成立]
  \begin{equation}
    f(x) = q(x)g(x) + r(x)
  \end{equation}
  $q(x)$ and $r(x)$ are uniquely determined. $r(x ) = 0$ or $\deg(r(x)) < \deg(g(x))$.
\end{proposition}
\begin{remark}[综合除法]
  对于除式是次数为1的多项式的情况,我们有一个快速进行带余除法的技巧,叫做综合除法。它基于以下观察
  \[
    f(x) = \Sigma_{i=0}^n a_i x^i, g(x) = x - c, q(x) = \Sigma_{i=0}^{n-1} b_i x^i, r(x) = r
  \]
  \[
    q(x)g(x) + r(x) = b_{n-1} x^n + (b_{n-2} - c b_{n-1}) x^{n-1} + (b_{n-3} - c b_{n-2}) x^{n-2} + \cdots + (b_0 - c b_1) x + r - c b_0 = \Sigma_{i=0}^n a_i x^i = f(x)
  \]
  则系数间有如下关系:
  \[
    \begin{cases}
      b_{n-1} = a_n \\
      b_{n-2} = a_{n-1} + c b_{n-1} \\
      b_{n-3} = a_{n-2} + c b_{n-2} \\
      \ldots \\
      b_1 = a_2 + c b_2 \\
      b_0 = a_1 + c b_1 \\
      r = a_0 + c b_0
    \end{cases}
  \]
  根据上述关系,我们可以快速得到带余除法的结果。

  注意上述关系只在除式$g(x) = x + c$即一次项系数为$1$才成立,对于一次项系数不是$1$的除式,则有$g(x) = c_1 x + c_0 = c_1 (x + \frac{c_0}{c_1}) = c_1 g'(x)$.
\end{remark}


\begin{definition}[整除]
  
\end{definition}

Properties of Divisors:
\begin{itemize}
  \item 对偶等价于相差常数倍
  \item 传递性
  \item 整除组合
\end{itemize}

\section{最大公因式}

\begin{definition}[Common Divisor and Greatest Common Divisor]
  If $\phi(x)$ is a divisor of both $f(x)$ and $g(x)$, we say it is a common 
  divisor of $f(x)$ and $g(x)$.

  Suppose $f(x) \in P\lbrack x \rbrack$ and $g(x) \in P\lbrack x \rbrack$. A 
  polynomial $d(x) \in P\lbrack x \rbrack$ is the greatest common divisor of 
  $f(x)$ and $g(x)$ if the following conditions are true:
  \begin{itemize}
    \item $d(x)$ is a common divisor of $f(x)$ and $g(x)$.
    \item Every common divisor of $f(x)$ and $g(x)$ is a divisor of $d(x)$.
  \end{itemize}
\end{definition}
\begin{remark}
  Since $\forall f(x) \in P\lbrack x \rbrack$ is a divisor of the zero 
  polynomial, in other words $0 = 0 \cdot f(x)$, the greatest common divisor of 
  $f(x)$ and $0$ is $0$. Especially, the greatest common divisor of $0$ and $0$ 
  is $0$, which is in accordance with the definition of GCD.
\end{remark}

\begin{lemma}[Common Divisors in Euclidean Division]
  If it holds that $f(x) = q(x)g(x) + r(x)$ for $f(x) \in P\lbrack x \rbrack$ 
  and $g(x) \in P\lbrack x \rbrack$, then the two pairs of polynomials $(f(x), 
  g(x))$ and $(g(x), r(x))$ have the same common divisors.
\end{lemma}
\begin{proof}
  
\end{proof}

\begin{theorem}[Theorem of Polynomial Greatest Common Divisor]
  $\forall f(x) \in P\lbrack x \rbrack$ and $\forall g(x) \in P\lbrack x \rbrack$, 
  there exists $d(x) \in P\lbrack x \rbrack$ that is the greatest common divisor 
  of $f(x)$ and $g(x)$, and $d(x)$ can be expressed as a combination of $f(x)$ 
  and $g(x)$, which is $\exists u(x) \in P\lbrack x \rbrack, v(x) \in P\lbrack x 
  \rbrack$ such that
  \[
    d(x) = u(x)f(x) + v(x)g(x)
  \]
\end{theorem}
\begin{proof}
  (TODO)
\end{proof}

\begin{remark}[Euclid's Algorithm]
  
\end{remark}

\begin{definition}[Coprime]
  (f(x), g(x)) = 1
\end{definition}

\begin{theorem}[互素的等价条件]
  \[
    ((f(x), g(x)) = 1) \Leftrightarrow u(x)f(x) + v(x)g(x) = 1
  \]
\end{theorem}
\begin{remark}
  This theorem contains a kind of symmetry because both $f(x)$ and $g(x)$ and $u(x)$ and $v(x)$ are coprime.
\end{remark}

\begin{theorem}
  If $(f(x), g(x)) = 1$, and $f(x) | g(x)h(x)$, then $f(x) | h(x)$.
\end{theorem}

\begin{corollary}
  If $f_1(x) | g(x)$, $f_2(x) | g(x)$, and $(f_1(x), f_2(x)) = 1$, then $f_1(x)f_2(x) | g(x)$.
\end{corollary}

上述结论均可推广到多个多项式。

\section{因式分解定理}
\begin{definition}[不可约多项式]
  
\end{definition}

\begin{theorem}[不可约多项式作因式]
  
\end{theorem}

\begin{theorem}[因式分解定理]
  
\end{theorem}

\section{重因式}

\section{多项式函数}
\end{document}