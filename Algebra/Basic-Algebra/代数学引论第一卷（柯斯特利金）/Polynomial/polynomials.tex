\documentclass[onecolumn]{ctexart}
\usepackage[utf8]{inputenc}
\usepackage{amsmath}
\usepackage{amssymb}
\usepackage{amsthm}
\usepackage{geometry}
\usepackage{graphicx}
\usepackage{float}
\usepackage{xcolor}
\usepackage{listings}
\usepackage{indentfirst}
\usepackage{bm}
\usepackage{tikz}
\usetikzlibrary{shapes,arrows}
\geometry{a4paper,scale=0.8}

\newtheorem{definition}{Definition}
\newtheorem{theorem}{Theorem}
\newtheorem{proposition}{Proposition}
\newtheorem{lemma}{Lemma}
\newtheorem{corollary}{Corollary}
\newtheorem{remark}{Remark}
\newtheorem{example}{Example}

\DeclareMathOperator{\rank}{rank}

\title{Notes of "Polynomials"}
\author{Jinxin Wang}
\date{}

\begin{document}

\maketitle

\section{一元多项式}

\begin{definition}[数域P上的一元多项式]
  
\end{definition}

\begin{definition}[一元多项式相等及零多项式]
  
\end{definition}

\begin{definition}[数域P上的一元多项式环]
  
\end{definition}

\section{整除的概念}

\begin{proposition}[带余除法的成立]
  \begin{equation}
    f(x) = q(x)g(x) + r(x)
  \end{equation}
  $q(x)$ and $r(x)$ are uniquely determined. $r(x ) = 0$ or $\deg(r(x)) < \deg(g(x))$.
\end{proposition}

\begin{definition}[整除]
  
\end{definition}

Properties of Divisors:
\begin{itemize}
  \item 对偶等价于相差常数倍
  \item 传递性
  \item 整除组合
\end{itemize}

\section{最大公因式}

\begin{definition}[Common Divisor and Greatest Common Divisor]
  
\end{definition}

\begin{lemma}[Common Divisors in Euclidean Division]
  $f(x)$ and $g(x)$, $g(x)$ and $r(x)$ have the same common divisors.
\end{lemma}

\begin{theorem}[Theorem of Polynomial Greatest Common Divisor]
  
\end{theorem}

\begin{example}[Euclid's Algorithm]
  
\end{example}

\end{document}