\documentclass{article}
\usepackage[utf8]{inputenc}
\usepackage{amsmath}
\usepackage{amssymb}
\usepackage{amsthm}
\usepackage{bm}
\usepackage{tikz}
\setlength{\parindent}{0pt}

\newtheorem{theorem}{Theorem}
\newtheorem{definition}{Definition}
\newtheorem{lemma}{Lemma}
\newtheorem{corollary}{Corollary}
\newtheorem{example}{Example}
\newtheorem{trick}{Trick}
\newtheorem{question}{Question}

\newcommand{\uvec}[1]{\boldsymbol{\hat{\textbf{#1}}}}

\title{Solving Linear Equations}
\author{Jinxin Wang}
\date{}

\begin{document}
    
\maketitle

The fundamental problem in linear algebra is to solve n linear equations in n 
unknown variables.

How do we understand a system of linear equations?

There are multiple geometric viewpoints of a system of linear equations.
\begin{itemize}
  \item Row picture: If we look at each row (equation) individually, each row 
  represents a set of points satisfying the equation, and the intersection of 
  all sets of points specified by the system of equations represents the 
  solution to the system.
  \item Column picture.
\end{itemize}

\end{document}