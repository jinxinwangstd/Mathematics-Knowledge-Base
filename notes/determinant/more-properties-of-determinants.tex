\documentclass[onecolumn]{ctexart}
\usepackage[utf8]{inputenc}
\usepackage{amsmath}
\usepackage{amssymb}
\usepackage{amsthm}
\usepackage{geometry}
\usepackage{graphicx}
\usepackage{float}
\usepackage{xcolor}
\usepackage{listings}
\usepackage{indentfirst}
\usepackage{bm}
\usepackage{tikz}
\usepackage{hyperref} % For the functionality of creating hypertext links. It should be the last package to declare.
\usetikzlibrary{shapes,arrows}
\geometry{a4paper,scale=0.8}

\newtheorem{definition}{Definition}
\newtheorem{theorem}{Theorem}
\newtheorem{proposition}{Proposition}
\newtheorem{lemma}{Lemma}
\newtheorem{corollary}{Corollary}
\newtheorem{remark}{Remark}
\newtheorem{example}{Example}

\DeclareMathOperator{\rank}{rank}

\title{Notes of "More Properties of Determinants"}
\author{Jinxin Wang}
\date{}

\begin{document}

\maketitle

\section{行列式按一行或一列的元素展开}

\begin{example}
\[
  \begin{vmatrix}
    a & b & b & \cdots & b \\
    b & a & b & \cdots & b \\
    b & b & a & \cdots & b \\
    \vdots & \vdots & \vdots & & \vdots \\
    b & b & b & \cdots & a \\
  \end{vmatrix} = (a + (n - 1)b)(a - b)^{n-1}
\]
\end{example}

\begin{example}[奇数阶反对称(斜对称)行列式]
  
\end{example}

\begin{example}[Verdemond Determinant]
\[
  D_n = 
  \begin{vmatrix}
    1 & 1 & 1 & \cdots & 1 \\
    a_1 & a_2 & a_3 & \cdots & a_n \\
    a_1^2 & a_2^2 & a_3^2 & \cdots & a_n^2 \\
    \vdots & \vdots & \vdots & & \vdots \\
    a_1^{n-1} & a_2^{n-1} & a_3^{n-1} & \cdots & a_n^{n-1} \\
  \end{vmatrix} = \Pi_{1 \leq i < j \leq n} (a_j - a_i)
\]

Pf: By mathematical induction
\[
  \begin{split}
    D_n &= 
    \begin{pmatrix}
      1 & 1 & 1 & \cdots & 1 \\
      a_1 - a_1 & a_2 - a_1 & a_3 - a_1 & \cdots & a_n - a_1 \\
      a_1^2 - a_1 a_1 & a_2^2 - a_2 a_1 & a_3^2 - a_3 a_1 & \cdots & a_n^2 - a_n a_1 \\
      \vdots & \vdots & \vdots & & \vdots \\
      a_1^{n-2} - a_1^{n-3}a_1 & a_2^{n-2} - a_2^{n-3}a_1 & a_3^{n-2} - a_3^{n-3}a_1 & \cdots & a_n^{n-2} - a_n^{n-3}a_1 \\
      a_1^{n-1} - a_1^{n-2}a_1 & a_2^{n-1} - a_2^{n-2}a_1 & a_3^{n-1} - a_3^{n-2}a_1 & \cdots & a_n^{n-1} - a_n^{n-2}a_1 \\
    \end{pmatrix} \\
    &=
    \begin{pmatrix}
      1 & 1 & 1 & \cdots & 1 \\
      0 & a_2 - a_1 & a_3 - a_1 & \cdots & a_n - a_1 \\
      0 & a_2(a_2 - a_1) & a_3(a_3 - a_1) & \cdots & a_n(a_n - a_1) \\
      \vdots & \vdots & \vdots & & \vdots \\
      0 & a_2^{n-3}(a_2 - a_1) & a_3^{n-3}(a_3 - a_1) & \cdots & a_n^{n-3}(a_n - a_1) \\
      0 & a_2^{n-2}(a_2 - a_1) & a_3^{n-2}(a_3 - a_1) & \cdots & a_n^{n-2}(a_n - a_1) \\     
    \end{pmatrix} \\
    &= (a_2 - a_1)(a_3 - a_1)\cdots(a_n - a_1)
    \begin{pmatrix}
      1 & 1 & 1 & \cdots & 1 \\
      a_2 & a_3 & a_4 & \cdots & a_n \\
      a_2^2 & a_3^2 & a_4^2 & \cdots & a_n^2 \\
      \vdots & \vdots & \vdots & & \vdots \\
      a_2^{n-2} & a_3^{n-2} & a_4^{n-2} & \cdots & a_n^{n-2} \\ 
    \end{pmatrix} \\
    &= \Pi_{j = 2}^n(a_j - a_1)D_{n-1}
  \end{split}
\]
\end{example}

\begin{example}[三对角行列式]
\[
  A = 
  \begin{vmatrix}
    a & b & 0 & \cdots & 0 & 0 \\
    c & a & b & \cdots & 0 & 0 \\
    0 & c & a & \cdots & 0 & 0 \\
    \vdots & \vdots & \vdots & & \vdots & \vdots \\
    0 & 0 & 0 & \cdots & a & b \\
    0 & 0 & 0 & \cdots & c & a \\
  \end{vmatrix}
\]

If $a^2 - 4bc = 0$, then let $\alpha$ be the only solution of the equation 
$x^2 - ax + bc = 0$, and thus $\det A = \frac{(n+1)\alpha^n}{2^n}$.

If $a^2 - 4bc \neq 0$, then let $\alpha, \beta$ be the two solutions of the 
equation $x^2 - ax + bc = 0$, and thus $\det A = 
\frac{\alpha^{n+1} - \beta^{n+1}}{\alpha - \beta}$.
\end{example}

\begin{example}[循环行列式]
\[
  \begin{vmatrix}
    a_0 & a_1 & a_2 & \cdots & a_{n-2} & a_{n-1} \\
    a_{n-1} & a_0 & a_1 & \cdots & a_{n-3} & a_{n-2} \\
    a_{n-2} & a_{n-1} & a_0 & \cdots & a_{n-4} & a_{n-3} \\
    \vdots & \vdots & \vdots & & \vdots & \vdots \\
    a_2 & a_3 & a_4 & \cdots & a_0 & a_1 \\
    a_1 & a_2 & a_3 & \cdots & a_{n-1} & a_0 \\
  \end{vmatrix} = f(1)f(\omega)f(\omega^2) \cdots f(\omega^{n-1})
\]
in which
\[
  f(x) = \Sigma_{i=0}^{n-1} a_i x^i, \omega = \exp(\frac{2\pi i}{n})
\]
\end{example}

\section{特殊矩阵的行列式}

\end{document}