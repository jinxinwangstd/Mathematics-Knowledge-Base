\documentclass[onecolumn]{ctexart}
\usepackage[utf8]{inputenc}
\usepackage{amsmath}
\usepackage{amssymb}
\usepackage{amsthm}
\usepackage{mathtools}
\usepackage{geometry}
\usepackage{graphicx}
\usepackage{float}
\usepackage{xcolor}
\usepackage{listings}
\usepackage{indentfirst}
\usepackage{bm}
\usepackage{tikz}
\usepackage{hyperref} % For the functionality of creating hypertext links. It should be the last package to declare.
\usetikzlibrary{shapes,arrows}
\geometry{a4paper,scale=0.8}

\newtheorem{definition}{Definition}
\newtheorem{theorem}{Theorem}
\newtheorem{proposition}{Proposition}
\newtheorem{lemma}{Lemma}
\newtheorem{corollary}{Corollary}
\newtheorem{remark}{Remark}
\newtheorem{example}{Example}

\DeclareMathOperator{\rank}{rank}

\title{Notes of "Applications of Determinants"}
\author{Jinxin Wang}
\date{}

\begin{document}

\maketitle

\section{Overview}
\begin{itemize}
  \item The Criterion for Non-Degenerate Matrices
  \item Cramer's Rule
  \item 加边子式法
  \begin{itemize}
    \item Definition: A k-th order minor of a matrix
    \item Theorem: 加边子式法
    \item Corollary: A practical way of computing the rank of a matrix
  \end{itemize}
\end{itemize}

\section{The Criterion for a Non-Degenerate Matrix}

\section{Cramer's Rule}

\section{加边子式法}

\begin{definition}[A k-th order minor of a matrix]
  
\end{definition}

\begin{theorem}[加边子式法]
  
\end{theorem}
\begin{remark}
  加边子式法 is practically useful especially when we want to find not only the rank, 
  but also a maximal linearly independent subset of rows or columns of a matrix. 
  Compared with it, transformation to row echelon form by elementary operations 
  will lose the information of maximal linearly independent subsets because 
  elementary operations will modify the elements of the matrix.
\end{remark}

\begin{corollary}[A practical way of computing the rank of a matrix]
  The rank of a matrix equals to the maximum order of its non-zero minors.
\end{corollary}

\end{document}