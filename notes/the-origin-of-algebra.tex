\documentclass[onecolumn]{ctexart}
\usepackage[utf8]{inputenc}
\usepackage{amsmath}
\usepackage{amssymb}
\usepackage{amsthm}
\usepackage{geometry}
\usepackage{graphicx}
\usepackage{float}
\usepackage{xcolor}
\usepackage{listings}
\usepackage{indentfirst}
\usepackage{bm}
\usepackage{tikz}
\usetikzlibrary{shapes,arrows.meta,matrix,topaths}
\geometry{a4paper,scale=0.8}

\newtheorem{definition}{Definition}
\newtheorem{theorem}{Theorem}
\newtheorem{proposition}{Proposition}
\newtheorem{lemma}{Lemma}
\newtheorem{corollary}{Corollary}
\newtheorem{remark}{Remark}
\newtheorem{example}{Example}

\DeclareMathOperator{\rank}{rank}
\DeclareMathOperator{\sgn}{sgn}

\title{Small Determinant}
\author{Jinxin Wang}
\date{}

\begin{document}

\maketitle

\section{Small Determinant}

\begin{tikzpicture}
[
  auto,
  block/.style = {rectangle, draw=black, thick, fill=white, text centered, text 
  width=20em, minimum height=4em},
  line/.style = {draw, thick, -latex'}
]
  \matrix [column sep=5mm, row sep=3mm] {
    \node [block] (p1) {Solve a linear system with two unknowns and two equations}; \\
    \node [block] (p2) {Find that the unknowns can be expressed with determinants consisting of coefficients}; \\
    \node [block] (p3) {Building from solving linear systems with two unknowns 
    and two equations, solve homogeous linear system with three unknowns and two
    equations}; \\
    \node [block] (p4) {Find that the unknowns can also be expressed with 
    determinants consisting of coefficients}; \\
    \node [block] (p5) {Building from here, }
  };
  \begin{scope}[every path/.style=line]
    \path (p1) -- (p2);
    \path (p2) -- (p3);
  \end{scope}
\end{tikzpicture}

\end{document}