\documentclass[onecolumn]{ctexart}
\usepackage[utf8]{inputenc}
\usepackage{amsmath}
\usepackage{amssymb}
\usepackage{amsthm}
\usepackage{geometry}
\usepackage{graphicx}
\usepackage{float}
\usepackage{xcolor}
\usepackage{listings}
\usepackage{indentfirst}
\usepackage{bm}
\usepackage{tikz}
\usetikzlibrary{shapes,arrows}
\geometry{a4paper,scale=0.8}

\newtheorem{definition}{Definition}
\newtheorem{theorem}{Theorem}
\newtheorem{proposition}{Proposition}
\newtheorem{corollary}{Corollary}
\newtheorem{remark}{Remark}

\title{Notes of "Permutations"}
\author{Jinxin Wang}
\date{}

\begin{document}

\maketitle

\section{The Concept of Permutations}

1. definition of permutations

2. notation of permutations

3. permutation group

3.1 the set of permutations of n elements

3.2 the operations on permutations

3.3 check it is a group

3.4 the cardinal of the permutation group

\section{The Cyclic Structure of Permutations}

1. Motivation: with operation, we try to turn a permutation into the result of 
some smaller permutations.

2. the concept of cycles and disjoint cycles

3. the definition of the power of a permutation

4. the order of a permutation

5. the equivalence relation determined by a permutation, and the orbits of a 
permutation

6. any permutation can be expressed as the product of disjoint cycles, and this 
expression is unique up to the order of the cycles.

7. the concept of a transposition

8. any permutation can be expressed as the product of transpositions.

\section{The Parity of Permutations}

1. the first definition of the parity of a permutation based on the number of 
transpositions.

2. proposition: the parity of a permutation is unique and independent from the 
expression of transpositions.

3. the concept of odd permutations and even permutations.

4. derive the parity of a permutation from its expression of the product of 
cycles.

\section{Permutation Group's Action on Functions}

1. the definition of a permutation's action on a function.

2. the definition of a skew-symmetric function

3. the proof of the uniqueless and independence of the parity of a permutation 
by the action of a permutation and the skew-symmetric function.

\end{document}