\documentclass[onecolumn]{ctexart}
\usepackage[utf8]{inputenc}
\usepackage{amsmath}
\usepackage{amssymb}
\usepackage{amsthm}
\usepackage{mathtools}
\usepackage{geometry}
\usepackage{graphicx}
\usepackage{float}
\usepackage{xcolor}
\usepackage{listings}
\usepackage{indentfirst}
\usepackage{bm}
\usepackage{tikz}
\usetikzlibrary{shapes,arrows}
\geometry{a4paper,scale=0.8}

\newtheorem{definition}{Definition}
\newtheorem{theorem}{Theorem}
\newtheorem{proposition}{Proposition}
\newtheorem{lemma}{Lemma}
\newtheorem{corollary}{Corollary}
\newtheorem{remark}{Remark}
\newtheorem{example}{Example}

\title{Notes of "Binary Operation"}
\author{Jinxin Wang}
\date{}

\begin{document}

\maketitle

\section{Overview}

\begin{itemize}
  \item Operations, semigroups and monoids
  \begin{itemize}
    \item Def: An n-ary operation on a set
    \item Examples of n-ary operations on a set
    \begin{itemize}
      \item Examples of unary operations: 
      \item Examples of binary operations: 
    \end{itemize}
    \item Def: A semigroup
    \item Def: An identity element and a monoid
    \begin{itemize}
      \item Rmk: The uniqueness of the identity element in a monoid
    \end{itemize}
    \item Examples of semigroups and monoids
    \begin{itemize}
      \item Eg: The set of all transformations on a set
    \end{itemize}
    \item Def: A subsemigroup and A submonoid
    \begin{itemize}
      \item Rmk: The equality between the identity element of a monoid and the one of its submonoid
    \end{itemize}
    \item Examples of subsemigroups and submonoids
    \begin{itemize}
      \item Eg: The set of all even integers with the addition in $(\mathbb{Z}, +)$
      \item Eg: The set of all inversible $M_n(\mathbb{R})$ with the multiplication of matrices in $(M_n(\mathbb{R}), \cdot)$
    \end{itemize}
  \end{itemize}
  \item Properties of associative operations
  \begin{itemize}
    \item Prop: The result of an associative binary operation of multiple operands is independent from the order of carrying out the operations
    \item Rmk: Examples of binary operations on a set that is not associative
  \end{itemize}
  \item Exponentions and multiples
  \begin{itemize}
    \item Def: Exponentions with non-negative integer exponents
    \item Rmk: Properties of exponentions with non-negative integer exponents
    \item Def: Multiples with non-negative integer multipliers
    \item Rmk: Properties of multiples with non-negative integer multipliers
  \end{itemize}
  \item Inversible elements
  \begin{itemize}
    \item Def: An inversible element and its inverse element
    \begin{itemize}
      \item Rmk: The prerequisite of discussing the inversibility of an element is having the identity element
      \item Rmk: The uniqueness of the inverse element of an inversible element in a monoid
      \item Rmk: The inversibility of the operation result of two inversible elements
    \end{itemize}
    \item Def: Negative exponentions and negative multiples
    \item Prop: Properties of exponentions with integer exponents
  \end{itemize}
\end{itemize}

\section{Operations, semigroups and monoids}

\begin{remark}[The equality between the identity element of a monoid and the one of its submonoid]
  The identity element of a monoid doesn't necessarily equal to the one in a submonoid, if the submonoid doesn't contain the identity of the monoid.

  One example is that the monoid is $(M_2(\mathbb{R}), \cdot)$, whose identity is $
  \begin{pmatrix}
    1 & 0 \\
    0 & 1 \\
  \end{pmatrix}$, and the submonoid is the set of real-valued matrices of order 2 with the form like $
  \begin{pmatrix}
    a & 0 \\
    0 & 0 \\
  \end{pmatrix}$, whose identity is $
  \begin{pmatrix}
    1 & 0 \\
    0 & 0 \\
  \end{pmatrix}$
\end{remark}

\section{Properties of associative operations}

\section{Exponentions and multiples}

\section{Inversible elements}

\end{document}