\documentclass[onecolumn]{ctexart}
\usepackage[utf8]{inputenc}
\usepackage{amsmath}
\usepackage{amssymb}
\usepackage{amsthm}
\usepackage{mathtools}
\usepackage{geometry}
\usepackage{graphicx}
\usepackage{float}
\usepackage{xcolor}
\usepackage{listings}
\usepackage{indentfirst}
\usepackage{bm}
\usepackage{tikz}
\usetikzlibrary{shapes,arrows}
\geometry{a4paper,scale=0.8}

\newcommand*\textinmath[1]{\thickspace\textnormal{#1}\thickspace}

\newtheorem{definition}{Definition}
\newtheorem{theorem}{Theorem}
\newtheorem{proposition}{Proposition}
\newtheorem{lemma}{Lemma}
\newtheorem{corollary}{Corollary}
\newtheorem{remark}{Remark}
\newtheorem{example}{Example}

\title{Notes of "Group"}
\author{Jinxin Wang}
\date{}

\begin{document}

\maketitle

\section{Overview}

\begin{itemize}
  \item Group and subgroup
  \begin{itemize}
    \item Def: A group
    \begin{itemize}
      \item Rmk: The uniqueness of the identity element in a group
      \item Rmk: The uniqueness of the inverse element of an element in a group
      \item Rmk: The definition of a group does not specify the uniqueness of the identity element and the inverse element of each element
      \item Rmk: The default notation of an abstract group is with multiplication, such as the operator and the identity element
    \end{itemize}
    \item Def: A subgroup of a group
    \begin{itemize}
      \item Rmk: Trivial subgroups and proper subgroups
      \item Rmk: Is the identity element of a subgroup always the same as the one of its parent group?
    \end{itemize}
    \item Examples of groups and subgroups
    \begin{itemize}
      \item Eg: $(\mathbb{Q}, +)$ and $(\mathbb{Z}, +)$
      \item Eg: $(\mathbb{R} \slash \lbrace 0 \rbrace, \cdot)$ and $(\mathbb{R}^+, \cdot)$
      \item Eg: $S_n$ and the set of even permutations of order $n$
      \item Eg: $GL_n(\mathbb{R})$ and $SL_n(\mathbb{R})$
      \item Eg: The set of inversible elements in a monoid
      \item Eg: $(\lbrace 1, -1 \rbrace, \cdot)$
    \end{itemize}
    \item Def: An abelian group
    \item Examples of abelian groups
    \begin{itemize}
      \item Eg: $(\mathbb{Z} \slash n\mathbb{Z}, +)$
      \item Eg: $A(X, Y) = \lbrace f: X \to Y \rbrace$ where $Y$ is a abelian group
      \item Eg: $L(X, Y) = \lbrace f: X \to Y \mid f \textinmath{is an additive map} \rbrace$
    \end{itemize}
    \item Def: The order (cardinality) of a group, a finite group and an infinite group
    \item Examples of finite groups and infinite groups
    \begin{itemize}
      \item Examples of finite groups: $S_n$, $(\lbrace 1, -1 \rbrace, \cdot)$
      \item Examples of infinite groups: $(\mathbb{Q}, +)$
    \end{itemize}
  \end{itemize}
  \item Cyclic groups
  \begin{itemize}
    \item Rmk: An element in a group generates a subgroup of the group
    \item Def: A cyclic group and its generator(s)
    \begin{itemize}
      \item Rmk: The uniqueness of the generator(s) of a cyclic group
    \end{itemize}
    \item Examples of cyclic groups
    \begin{itemize}
      \item Eg: $(\mathbb{Z}, +)$ can be generated by $1$ or $-1$
      \item Eg: $(\lbrace 1, -1 \rbrace, \cdot)$ can be generated by $-1$
    \end{itemize}
    \item Rmk: The order of a generated cyclic group by an element in a finite group
  \end{itemize}
  \item The order of an element in a group
  \begin{itemize}
    \item Def: An element of infinite order or finite order in a group
    \item Examples of elements of infinite order and elements of finite order in groups
    \begin{itemize}
      \item Examples of elements of finite order in groups: A permutation in $S_n$
      \item Examples of elements of infinite order in groups: $1$ and $-1$ in $(\mathbb{Z}, +)$
    \end{itemize}
    \item Prop: The relationship between the order of an element in a group and the order of the cyclic group generated by it
  \end{itemize}
  \item Subgroups of a cyclic group
  \begin{itemize}
    \item Prop: The form of subgroups of a cyclic group
    \begin{itemize}
      \item Rmk: For a subgroup of a finite cyclic group, the factor $k$ is not unique?
    \end{itemize}
    \item Prop: The relationship between different generators of a finite cyclic group
    \item Eg: An application of the form of subgroups of a cyclic group to $(\mathbb{Z}, +)$
  \end{itemize}
  \item Homomorphisms and isomorphisms
  \begin{itemize}
    \item Def: A group homomorphism
    \item Def: A group isomorphism
    \begin{itemize}
      \item Rmk: Both a homomorphism and an isomorphism refer to a mapping rather than a relation between two algebraic structures
    \end{itemize}
    \item Prop: Some basic properties of a group homomorphism
    \item Prop: Some basic properties of a group isomorphism
  \end{itemize}
  \item Examples and conclusions of group homomorphisms and group isomorphisms
  \begin{itemize}
    \item Prop: A necessary and sufficient condition of two cyclic groups to be isomorphic in terms of the orders of them
  \end{itemize}
\end{itemize}

\section{Group and subgroup}

\begin{definition}[A group]
  A set $G$ is called a group if a binary operation $\cdot$ is defined on it, and for any $a, b, c \in G$, it holds that
  \begin{description}
    \item[D1] Associative law: $a \cdot (b \cdot c) = (a \cdot b) \cdot c$
    \item[D2] Identity element: There exists $e \in G$ such that $e \cdot a = a \cdot e = a$ for each $a \in G$
    \item[D3] Inverse element: For each $a \in G$, there exists $b \in G$ such that $ab = ba = e$
  \end{description}
\end{definition}
\begin{remark}[The uniqueness of the identity element in a group]
  Suppose there are $e \in G$ and $e' \in G$ which satisfy the definition of the identity element, then
  \[
    e = e \cdot e' = e'
  \]
\end{remark}
\begin{remark}[The uniqueness of the inverse element of an element in a group]
  For each $a \in G$, suppose there are two inverse elements $b \in G$ and $b' \in G$, then
  \[
    b = b \cdot e = b \cdot (a \cdot b') = (b \cdot a) \cdot b' = e \cdot b' = b'
  \]
\end{remark}
\begin{remark}[The definition of a group does not specify the uniqueness of the identity element and the inverse element of each element]
  As we can see, the definition of a group does not require the uniqueness of 
  the identity element and the inverse element of each element in a group. The 
  reason is that the uniqueness is a natural property of the identity element 
  and the inverse element of each element once a set satisfies the definition 
  of a group. Hence, we don't need and aren't supposed to add such conditions 
  to the definition.
\end{remark}
\begin{remark}[The default notation of an abstract group is with multiplication, such as the operator and the identity element]
  
\end{remark}

\begin{definition}[A subgroup of a group]
  
\end{definition}
\begin{remark}[Is the identity element of a subgroup always the same as the one of its parent group?]
  The identity element of a group is also the identity of its every subgroup.

  Proof: Suppose $U$ is a group, $V \subset U$ is a subgroup of $U$, and $e_U$ 
  and $e_V$ are the identity elements of them respectively. For $e_V$, we have 
  $e_V^2 = e_V$. Since $e_V \in U$, the equation also holds in $U$. Then we have 
  $e_V^{-1} e_V^2 = e_V^{-1} e_V$, which is $e_V = e_U$.

  Recall that the similar conclusion doesn't hold for monoids. We can see the 
  changes brought by the additional axioms of a group compared with a monoid.
\end{remark}

\begin{example}[$(\mathbb{Q}, +)$ and $(\mathbb{Z}, +)$]
  $(\mathbb{Q}, +)$ is a group, and $(\mathbb{Z}, +)$ is a subgroup of it.
\end{example}

\begin{example}[$(\mathbb{R} \slash \lbrace 0 \rbrace, \cdot)$ and $(\mathbb{R}^+, \cdot)$]
  $(\mathbb{R} \slash \lbrace 0 \rbrace, \cdot)$ is a group, and $(\mathbb{R}^+, \cdot)$ is a subgroup of it.
\end{example}

\begin{example}[$S_n$ and the set of even permutations of order $n$]
  The set of permutations of order $n$ $S_n$ is a group, and the set of even permutations of order $n$ is a subgroup of $S_n$.
\end{example}

\begin{example}[$GL_n(\mathbb{R})$ and $SL_n(\mathbb{R})$]
  The set of inversible real-valued matrices of order $n$, denoted by 
  $GL_n(\mathbb{R})$, forms a group. The set of inversible real-valued matrices 
  of order $n$ whose determinant is $1$, denoted by $SL_n(\mathbb{R})$, is a 
  subgroup of $GL_n(\mathbb{R})$.
\end{example}

\section{Cyclic groups}

\section{The order of an element in a group}

\begin{definition}[An element of infinite order or finite order in a group]
  Given an element $a$ in a group, we check the sequence of its integer 
  exponentions. If all of them are different, we say that the order of the 
  element $a$ is infinite, and $a$ is called an element of infinite order in the 
  group. If there are same elements, then there exists integers $k_i$ such that 
  $a^{k_i} = e$. The minimal integer in the set $\lbrace k_i \rbrace$, denoted 
  by $q$, is called the order of the element $a$ in the group, and $a$ is called 
  an element of finite order in the group, or an element of order $q$ in the 
  group.
\end{definition}

\section{Subgroups of a cyclic group}

\section{Homomorphisms and isomorphisms}

\section{Examples and conclusions of group homomorphisms and group isomorphisms}

\end{document}